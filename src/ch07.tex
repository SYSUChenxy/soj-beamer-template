%------------------------------------------------------------
\title[07 - 循环与分支综合]
{07 - 循环与分支综合}

\subtitle{C++ 程序设计基础}

\author[Beiyu Li]
{Beiyu Li\\
\texttt{<sysulby@gmail.com>}}

% \institute[SOJ]
% {Sicily Online Judge}

\date[\today]
{\number\year 年 \number\month 月 \number\day 日}
%------------------------------------------------------------


\begin{document}

\author[sysulby]
{SOJ 信息学竞赛教练组}

\begin{frame}
    \titlepage
\end{frame}
\setcounter{framenumber}{0} % 标题页不编号


\section{复习回顾}

%------------------------------------------------------------
\begin{frame}[fragile]
    \frametitle{分支结构}

    \begin{columns}
        \column{.01\textwidth}

        \column{.49\textwidth}
        \begin{itemize}[<+->]
            \item 根据不同的情况,执行不同的操作
                \begin{itemize}
                    \item 单分支
                    \item 双分支
                    \item 多分支
                    \item 分支嵌套
                \end{itemize}
        \end{itemize}

        \column{.50\textwidth}

        \only<2>{\lstinputlisting[language=C++,name=if]{ch04/if.cc}}
        \only<3>{\lstinputlisting[language=C++,name=if_else]{ch04/if_else.cc}}
        \only<4>{\lstinputlisting[language=C++,name=if_else_if]{ch04/if_else_if.cc}}
        \only<5>{\lstinputlisting[language=C++,name=if_nest]{ch04/if_nest.cc}}


    \end{columns}

\end{frame}
%------------------------------------------------------------

%------------------------------------------------------------
\begin{frame}[fragile]
    \frametitle{例题回顾:闰年判断}

    \alt<2-2>{
        \lstinputlisting[language=C++,name=leap2]{ch04/leap2.cc}
    }{
        \begin{exampleblock}{编程题}

            \begin{itemize}
                \item 闰年分为世纪闰年和普通闰年。\\
                    普通闰年的年份是 $4$ 的倍数,且不是 $100$ 的倍数;世纪闰年的年份是 $400$ 的倍数。\\
                    编写程序,输入一个整数 $year$ ($1000 \le year \le 3000$),表示一个年份,判断该年份是否为闰年,输出对应的判断结果。

                \item 样例输入

                    \lstinline|2010|

                \item 样例输出

                    \lstinline|2010 is not a leap year|

            \end{itemize}

        \end{exampleblock}
    }
\end{frame}
%------------------------------------------------------------


%------------------------------------------------------------
\begin{frame}[fragile]
    \frametitle{循环结构}

    \begin{itemize}[<+->]
        \item 可反复执行某些指令
            \begin{itemize}
                \item \lstinline|while| 循环:先判断再执行
                \item \lstinline|do..while| 循环:先执行再判断,至少执行一次
                \item \lstinline|for| 循环:执行初始动作后,先判断再执行
            \end{itemize}

            \only<2>{\lstinputlisting[language=C++,name=while]{ch05/while.cc}}
            \only<3>{\lstinputlisting[language=C++,name=do_while]{ch05/do_while.cc}}
            \only<4>{\lstinputlisting[language=C++,name=for]{ch06/for.cc}}
    \end{itemize}

\end{frame}
%------------------------------------------------------------


%------------------------------------------------------------
\begin{frame}[fragile]
    \frametitle{例题回顾:输出从 1 到 n 的整数之和}

    \alt<2>{
        \lstinputlisting[language=C++,name=for_sum_1_to_n]{ch06/for_sum_1_to_n.cc}
    }{
        \begin{exampleblock}{编程题}

            \begin{itemize}
                \item 编写程序,输入一个整数 $n$ ($1 \le n \le 1000$),输出从 $1$ 到 $n$ 之间的整数之和。

                \item 样例输入

                    \lstinline|5|

                \item 样例输出

                    \lstinline|15|

                \item 样例说明

                    $1+2+3+4+5=15$

            \end{itemize}

        \end{exampleblock}
    }
\end{frame}
%------------------------------------------------------------


\section{循环嵌套分支}

%------------------------------------------------------------
\begin{frame}[fragile]
    \frametitle{例 7.1:加减数列求和}

    \alt<2>{
        \lstinputlisting[language=C++,name=sum]{ch07/sum.cc}
    }{
        \begin{exampleblock}{编程题}

            \begin{itemize}
                \item 编写程序,输入一个整数 $n$ ($1 \le n \le 1000$),输出 $1-2+3-4+5...$,一直到 $n$ 的和。

                \item 样例输入

                    \lstinline|7|

                \item 样例输出

                    \lstinline|4|

                \item 样例说明

                    $1-2+3-4+5-6+7=4$

            \end{itemize}

        \end{exampleblock}
    }
\end{frame}
%------------------------------------------------------------

%------------------------------------------------------------
\begin{frame}[fragile]
    \frametitle{例 7.2:优秀学生}

    \alt<2-9>{
        \begin{columns}
            \column{.01\textwidth}

            \column{.72\textwidth}
            \lstinputlisting[language=C++,name=student]{ch07/student.cc}

            \column{.27\textwidth}
            \begin{itemize}
                    \uncover<3-> {
                    \item 允许边输入边输出

                    \item<4-> 运行窗口

                        \uncover<5->{\lstinline|2|\\}
                        \uncover<6->{\lstinline|98 92|\\}
                        \uncover<7->{\lstinline|Good Job|\\}
                        \uncover<8->{\lstinline|80 95|\\}
                        \uncover<9->{\lstinline|Try Harder|}
                    }
            \end{itemize}
        \end{columns}

    }{
        \begin{exampleblock}{编程题}

            \begin{itemize}
                \item 编写程序,输入一个整数 $n$ ($1 \le n \le 1000$),表示有 $n$ 个学生,接下来输入每个学生的语文成绩 $x$ ($1 \le x \le 100$) 和数学成绩 $y$ ($1 \le y \le 100$)。对于每个学生,如果其两科成绩均不低于 $90$ 分,则输出“Good Job”,否则输出“Try Harder”。

                \item 样例输入

                    \lstinline|2|\\
                    \lstinline|98 92|\\
                    \lstinline|80 95|

                \item 样例输出

                    \lstinline|Good Job|\\
                    \lstinline|Try Harder|


            \end{itemize}

        \end{exampleblock}
    }
\end{frame}
%------------------------------------------------------------


%------------------------------------------------------------
\begin{frame}[fragile]
    \frametitle{例 7.3:输出 n 的所有因子}

    \alt<2-3>{
        \lstinputlisting[language=C++,name=factors]{ch07/factors.cc}
        \begin{tikzpicture}[remember picture,overlay]
            \uncover<3>{\redbox{factors-9}{8}{10}}
        \end{tikzpicture}
    }{
        \begin{exampleblock}{编程题}

            \begin{itemize}
                \item 编写程序,输入一个整数 $n$ ($1 \le n \le 1000$),输出 $n$ 的所有因子。

                \item 样例输入

                    \lstinline|8|

                \item 样例输出

                    \lstinline|1 2 4 8|

            \end{itemize}

        \end{exampleblock}
    }
\end{frame}
%------------------------------------------------------------

%------------------------------------------------------------
\begin{frame}[fragile]
    \frametitle{例 7.4:求 n 的所有因子之和}

    \alt<2-3>{
        \lstinputlisting[language=C++,name=factors_sum]{ch07/factors_sum.cc}

        \begin{tikzpicture}[remember picture,overlay]
            \uncover<3>{\redbox{factors_sum-10}{6}{9}}
        \end{tikzpicture}
    }{
        \begin{exampleblock}{编程题}

            \begin{itemize}
                \item 编写程序,输入一个整数 $n$ ($1 \le n \le 1000$),求 $n$ 的所有因子之和。

                \item 样例输入

                    \lstinline|8|

                \item 样例输出

                    \lstinline|15|

                \item 样例说明

                    $1+2+4+8=15$

            \end{itemize}

        \end{exampleblock}
    }
\end{frame}
%------------------------------------------------------------


%------------------------------------------------------------
\begin{frame}[fragile]
    \frametitle{例 7.5:求 n 个数的最大值}

    \alt<2-10>{
        \alt<7-10> {
            \alt<9-10>{
                \begin{itemize}
                        \lstinputlisting[language=C++,name=max2]{ch07/max2.cc}

                    \item 把擂主的初始值设置为最小值,保证第一个挑战者能攻擂成功,则无需特判是否为第一个挑战者
                \end{itemize}
            }{
                \begin{itemize}
                        \lstinputlisting[language=C++,name=max1]{ch07/max1.cc}

                    \item 如果 $x$ 是第一个挑战者,那 $x$ 直接成为擂主;或者新来的挑战者 $x$ 大于擂主 $Max$,那挑战者 $x$ 成为擂主
                \end{itemize}
            }

            \begin{tikzpicture}[remember picture,overlay]
                \uncover<8>{\redbox{max1-7}{7}{10}}
                \uncover<8>{\redbox{max1-9}{8}{6}}
                \uncover<10>{\redbox{max2-5}{12}{11}}
                \uncover<10>{\redbox{max2-9}{8}{7}}
            \end{tikzpicture}
        }{
            \begin{itemize}
                \item 打擂台找最大值
            \end{itemize}				

            \vspace*{1em}
            \centering
            \begin{tikzpicture}[node distance=1.25cm]
                \node (start) [process] {8};
                \node (second) [process, right of=start, xshift = 1.5cm] {7};
                \node (third) [process, right of=second, xshift = 1.5cm] {9};
                \node (stop) [process, right of=third, xshift = 1.5cm] {5};
                \only<2>{\node (max) [startstop, below of=second, xshift = 1.5cm] {擂主};}
				\only<3-4>{\node (max1) [startstop, below of=second, xshift = 1.5cm] {擂主: 8};}
				\only<5-6>{\node (max2) [startstop, below of=second, xshift = 1.5cm] {擂主: 9};}
				
				\only<3>{\draw [arrow] (start) --  node[left, xshift = -0.3cm, yshift = -0.1cm] {攻擂成功} (max1);}
				\only<4>{\draw [arrow] (second) --  node[left, xshift = -0.1cm, yshift = -0.1cm] {攻擂失败} (max1);}
				\only<5>{\draw [arrow] (third) -- node[left, xshift = -0.1cm] {攻擂成功} (max2);}
				\only<6>{\draw [arrow] (stop) --  node[left, xshift=-0.3cm] {攻擂失败} (max2);}
            \end{tikzpicture}	
        }
    }{
        \begin{exampleblock}{编程题}

            \begin{itemize}
                \item 编写程序,输入一个整数 $n$ ($1 \le n \le 1000$),接下来输入 $n$ 个整数 $x$ ($-10^9 \le x \le 10^9$),求这 $n$ 个整数中的最大值。

                \item 样例输入

                    \lstinline|4|\\
                    \lstinline|8 7 9 5|

                \item 样例输出

                    \lstinline|9|

            \end{itemize}

        \end{exampleblock}
    }
\end{frame}
%------------------------------------------------------------


%------------------------------------------------------------
\begin{frame}[fragile]
    \frametitle{例 7.6:求 n 个数的最大值及位置}

    \alt<2-3>{
        \lstinputlisting[language=C++,name=max_pos]{ch07/max_pos.cc}

        \begin{tikzpicture}[remember picture,overlay]
            \uncover<3>{\redbox{max_pos-7}{2}{10}}
            \uncover<3>{\redbox{max_pos-13}{6}{10}}
        \end{tikzpicture}
    }{
        \begin{exampleblock}{编程题}

            \begin{itemize}
                \item 编写程序,输入一个整数 $n$ ($1 \le n \le 1000$),接下来输入 $n$ 个整数 $x$ ($-10^9 \le x \le 10^9$),求这 $n$ 个整数中的最大值及位置。

                \item 样例输入

                    \lstinline|4|\\
                    \lstinline|8 7 9 5|

                \item 样例输出

                    \lstinline|9 3|

            \end{itemize}

        \end{exampleblock}
    }
\end{frame}
%------------------------------------------------------------

%------------------------------------------------------------
\begin{frame}[fragile]
    \frametitle{例 7.7:反向输出非负整数}

    \alt<2-3>{
        \lstinputlisting[language=C++,name=output_digits]{ch07/output_digits.cc}

        \begin{tikzpicture}[remember picture,overlay]
            \uncover<3>{\redbox{output_digits-9}{4}{15}}
            \uncover<3>{\redbox{output_digits-10}{4}{8}}
        \end{tikzpicture}
    }{
        \begin{exampleblock}{编程题}

            \begin{itemize}
                \item 编写程序,输入一个整数 $n$ ($1 \le n \le 10^9$),请从低位到高位输出 $n$ 的每个数位。

                \item 样例输入

                    \lstinline|856|

                \item 样例输出

                    \lstinline|6 5 8|

            \end{itemize}

        \end{exampleblock}
    }
\end{frame}
%------------------------------------------------------------


%------------------------------------------------------------
\begin{frame}[fragile]
    \frametitle{例 7.8:求一个整数中有多少个 3}

    \alt<2-3>{
        \lstinputlisting[language=C++,name=num_of_digit_3]{ch07/num_of_digit_3.cc}

        \begin{tikzpicture}[remember picture,overlay]
            \uncover<3>{\redbox{num_of_digit_3-9}{4}{15}}
            \uncover<3>{\redbox{num_of_digit_3-11}{8}{6}}
        \end{tikzpicture}
    }{
        \begin{exampleblock}{编程题}

            \begin{itemize}
                \item 编写程序,输入一个整数 $n$ ($1 \le n \le 10^9$),统计 $n$ 的数位上有多少个 $3$。

                \item 样例输入

                    \lstinline|1353|

                \item 样例输出

                    \lstinline|2|

            \end{itemize}

        \end{exampleblock}
    }
\end{frame}
%------------------------------------------------------------


\section{break 和 continue 语句}

%------------------------------------------------------------
\begin{frame}[fragile]
    \frametitle{break 和 continue 语句}

    \begin{itemize}
        \item \lstinline|break|:用于循环体中,表示跳出本层循环
    \end{itemize}

    \begin{columns}
        \column{.05\textwidth}

        \column{.43\textwidth}
        \uncover<1-4>{\lstinputlisting[language=C++,name=while_break]{ch07/while_break.cc}}

        \column{.52\textwidth}
        \uncover<3-4>{\lstinputlisting[language=C++,name=for_break]{ch07/for_break.cc}}
    \end{columns}

    \begin{tikzpicture}[remember picture,overlay]
        \uncover<2->{\redbox{while_break-3}{11}{6}}
		\uncover<2->{\draw[red, very thick, ->] (3.8,1.25) -- (4.5,1.25) -- (4.5,0.45) -- (1.0,0.45);}
		\uncover<4>{\redbox{for_break-3}{11}{6}}
		\uncover<4>{\draw[red, very thick, ->] (8.95,1.25) -- (9.7,1.25) -- (9.7,0.45) -- (6.1,0.45);}
    \end{tikzpicture}
\end{frame}
%------------------------------------------------------------

%------------------------------------------------------------
\begin{frame}[fragile]
    \frametitle{break 和 continue 语句}

    \begin{itemize}
        \item \lstinline|continue|:用于循环体中,表示跳过本次循环(循环体)中余下未执行的语句,进行下一次循环
    \end{itemize}

    \begin{columns}
        \column{.05\textwidth}

        \column{.43\textwidth}
        \uncover<1-4>{\lstinputlisting[language=C++,name=while_continue]{ch07/while_continue.cc}}

        \column{.52\textwidth}
        \uncover<3-4>{\lstinputlisting[language=C++,name=for_continue]{ch07/for_continue.cc}}
    \end{columns}

    \begin{tikzpicture}[remember picture,overlay]
        \uncover<2->{\redbox{while_continue-3}{11}{9}}
        \uncover<2->{\draw[red, very thick, ->] (4.35,1.25) -- (5.0,1.25) -- (5.0,2) -- (3.1,2);}
        \uncover<4>{\redbox{for_continue-3}{11}{9}}
        \uncover<4>{\draw[red, very thick, ->] (9.5,1.25) -- (10.4,1.25) -- (10.4,1.85);}
    \end{tikzpicture}

\end{frame}
%------------------------------------------------------------

%------------------------------------------------------------
\begin{frame}[fragile]
    \frametitle{break 语句 - 示例}

    \begin{itemize}
        \item 输入一个整数 $n$ ($1 \le n \le 10^6$),输出 $1 \sim n$ 中的每个整数,增加判断当这个数为 $5$ 时,使用 \lstinline|break| 语句,观察输出结果。
    \end{itemize}

    \begin{columns}
        \column{.01\textwidth}

        \column{.43\textwidth}
        \uncover<2-5>{\lstinputlisting[language=C++,name=while_1_n_break]{ch07/while_1_n_break.cc}}

        \column{.56\textwidth}
        \uncover<4-5>{\lstinputlisting[language=C++,name=for_1_n_break]{ch07/for_1_n_break.cc}}
    \end{columns}

    \begin{tikzpicture}[remember picture,overlay]
        \uncover<3->{\redbox{while_1_n_break-4}{14}{6}}
		\uncover<3->{\draw[red, very thick, ->] (3.97,1.65) -- (4.5,1.65) -- (4.5,0.4) -- (0.6,0.4);}
		\uncover<5>{\redbox{for_1_n_break-4}{14}{6}}
		\uncover<5>{\draw[red, very thick, ->] (9.1,1.45) -- (9.7,1.45) -- (9.7,0.6) -- (5.7,0.6);}
    \end{tikzpicture}
\end{frame}
%------------------------------------------------------------


%------------------------------------------------------------
\begin{frame}[fragile]
    \frametitle{continue 语句 - 示例}

    \begin{itemize}
        \item 输入一个整数 $n$ ($1 \le n \le 10^6$),输出 $1 \sim n$ 中的每个整数,增加判断当这个数为 $5$ 时,使用 \lstinline|continue| 语句,观察输出结果。
    \end{itemize}

    \begin{columns}
        \column{.01\textwidth}

        \column{.43\textwidth}
        \only<2-5>{\lstinputlisting[language=C++,name=while_1_n_continue]{ch07/while_1_n_continue.cc}}

        \column{.56\textwidth}
        \only<4-5>{\lstinputlisting[language=C++,name=for_1_n_continue]{ch07/for_1_n_continue.cc}}
    \end{columns}

    \begin{tikzpicture}[remember picture,overlay]
        \uncover<3->{\redbox{while_1_n_continue-4}{14}{9}}
        \uncover<3->{\draw[red, very thick, ->] (4.5,1.65) -- (4.7, 1.65) -- (4.7,2) -- (3.2,2);}
        \uncover<5>{\redbox{for_1_n_continue-4}{14}{9}}
        \uncover<5>{\draw[red, very thick, ->] (9.65,1.45) -- (10.2,1.45) -- (10.2,1.75);}
    \end{tikzpicture}

\end{frame}
%------------------------------------------------------------

%------------------------------------------------------------
\begin{frame}[fragile]
    \frametitle{break 和 continue 语句}

    \begin{itemize}
        \item \lstinline|break|
            \begin{itemize}
                \item 在 \lstinline|while| 循环中,直接跳出本层 \lstinline|while| 循环
                \item 在 \lstinline|for| 循环中,直接跳出本层 \lstinline|for| 循环
            \end{itemize}
        \item \lstinline|continue|
            \begin{itemize}
                \item 在 \lstinline|while| 循环中,直接跳转到\textbf{条件语句},需要额外注意
                \item 在 \lstinline|for| 循环中,直接跳转到每次循环后的动作
            \end{itemize}
    \end{itemize}

\end{frame}
%------------------------------------------------------------

%------------------------------------------------------------
\begin{frame}[fragile]
    \frametitle{随堂练习}

    \begin{exampleblock}{填空题}

        \begin{enumerate}
            \item 阅读程序写结果
                \lstinputlisting[language=C++,name=exercise]{ch07/exercise.cc}

                输入:\lstinline|7 12| \tabto{8em} 	
                输出:\uncover<2->{\textcolor{red}{\lstinline|7 8 9|}}
        \end{enumerate}

    \end{exampleblock}
\end{frame}
%------------------------------------------------------------


\section{总结}

%------------------------------------------------------------
\begin{frame}[fragile]
    \frametitle{总结}

    \begin{itemize}
        \item 循环与分支综合
            \begin{itemize}
                \item 求和问题
                \item 计数问题
                \item 最值问题
                \item 数位拆分
            \end{itemize}
        \item \lstinline|break| 和 \lstinline|continue| 语句
    \end{itemize}
\end{frame}
%------------------------------------------------------------

%---------------------------------------------------------
\begin{frame}
    \begin{center}
        {\Huge Thank you!}
    \end{center}
\end{frame}
%---------------------------------------------------------

\end{document}
