%------------------------------------------------------------
\title[03 - 表达式与运算]
{03 - 表达式与运算}

\subtitle{C++ 程序设计基础}

\author[Beiyu Li]
{Beiyu Li\\
\texttt{<sysulby@gmail.com>}}

% \institute[SOJ]
% {Sicily Online Judge}

\date[\today]
{\number\year 年 \number\month 月 \number\day 日}
%------------------------------------------------------------


\begin{document}

\author[sysulby]
{SOJ 信息学竞赛教练组}

\begin{frame}
    \titlepage
\end{frame}
\setcounter{framenumber}{0} % 标题页不编号


\section{复习回顾}

%------------------------------------------------------------
\begin{frame}[fragile]
    \frametitle{变量的类型}

    \begin{itemize}
        \item<1-> 整数类型

            \begin{itemize}
                \item \lstinline|int|:近似值 $2.1 \times 10^9$
                \item \lstinline|long long|:近似值 $9.2 \times 10^{18}$
            \end{itemize}

        \item<2-> 浮点数类型(实数/小数)

            \begin{itemize}
                \item \lstinline|double|:双精度浮点数
            \end{itemize}

        \item<3-> 字符类型

            \begin{itemize}
                \item \lstinline|char|:键盘上的字母、数字及符号
            \end{itemize}

        \item<4-> 布尔类型

            \begin{itemize}
                \item \lstinline|bool|:true / false(真 / 假)
            \end{itemize}

    \end{itemize}
\end{frame}
%------------------------------------------------------------

%------------------------------------------------------------
\begin{frame}[fragile]
    \frametitle{变量的声明与访问}

    \begin{itemize}
        \item<1-> 变量的声明

            \begin{itemize}
                \item \textbf{类型 \enspace 变量名}
                \item \lstinline|int a;|
            \end{itemize}

        \item<2-> 变量的赋值

            \begin{itemize}
                \item \lstinline|a = 100;|
            \end{itemize}

        \item<3-> 使用 \lstinline|cin| 语句输入变量,在输入流符号 \lstinline|>>| 后写上变量名

            \begin{itemize}
                \item \lstinline|cin >> a;|
            \end{itemize}

        \item<4-> 使用 \lstinline|cout| 语句输出变量,在输出流符号 \lstinline|<<| 后写上变量名

            \begin{itemize}
                \item \lstinline|cout << a << endl;|
            \end{itemize}

    \end{itemize}
\end{frame}
%------------------------------------------------------------


\section{算术运算}

%------------------------------------------------------------
\begin{frame}[fragile]
    \frametitle{算术运算}

    \begin{table}[]
        \begin{tabular}{cccc}
            \toprule
            操作          & 操作符        & 例子                & 结果            \\
            \midrule
            加法          & \lstinline|+| & \lstinline|20 + 10| & \lstinline|30|  \\
            减法          & \lstinline|–| & \lstinline|20 – 10| & \lstinline|10|  \\
            乘法          & \lstinline|*| & \lstinline|20 * 10| & \lstinline|200| \\
            除法          & \lstinline|/| & \lstinline|20 / 10| & \lstinline|2|   \\
            取模(求余数)& \lstinline|%| & \lstinline|20 % 10| & \lstinline|0|   \\
            \bottomrule
        \end{tabular}
    \end{table}
\end{frame}
%------------------------------------------------------------

%------------------------------------------------------------
\begin{frame}[fragile]
    \frametitle{运算规则}

    \begin{itemize}[<+->]
        \item \textbf{相同数据类型}的数据进行算术运算,得到的结果也是\textbf{相同类型}

            \begin{itemize}
                \item 例:整数除以整数,结果也是整数
                \item \lstinline|2 / 4| 的结果是什么?是 $0$ 还是 $0.5$?

                    \begin{itemize}
                        \item 是 $0$(整数)而不是 $0.5$(实数)
                    \end{itemize}

                \item 那如何让 \lstinline|2 / 4| 的结果为 $0.5$ 呢?
            \end{itemize}

    \end{itemize}
\end{frame}
%------------------------------------------------------------

%------------------------------------------------------------
\begin{frame}[fragile]
    \frametitle{隐式类型转换}

    \begin{itemize}[<+->]
        \item \textbf{不同数据类型}的数据进行算术运算,则存储空间小的类型自动转换为存储空间大的类型,再进行运算,运算的结果为\textbf{存储空间大}的类型

            \begin{itemize}
                \item 例:实数除以整数 ,结果是实数
                \item \lstinline|2.0 / 4| 的结果的结果为 $0.5$(实数)
            \end{itemize}

    \end{itemize}
\end{frame}
%------------------------------------------------------------

%------------------------------------------------------------
\begin{frame}[fragile]
    \frametitle{随堂练习}

    \begin{exampleblock}{选择题}

        \begin{enumerate}
            \item 下面语句中可以得到整数的有

                \alt<2>{
                    \begin{tasks}
                        \task[A.] \lstinline|10 / 4.0;|
                        \task[B.] \lstinline|(double)10 / 4;|
                        \task[C.] \lstinline|1.0 * 10;|
                        \task[D.] \textcolor{red}{\lstinline|10 * 4;|}
                    \end{tasks}
                }{
                    \begin{tasks}
                        \task[A.] \lstinline|10 / 4.0;|
                        \task[B.] \lstinline|(double)10 / 4;|
                        \task[C.] \lstinline|1.0 * 10;|
                        \task[D.] \lstinline|10 * 4;|
                    \end{tasks}
                }

        \end{enumerate}

    \end{exampleblock}
\end{frame}
%------------------------------------------------------------


\section{表达式}

%------------------------------------------------------------
\begin{frame}[fragile]
    \frametitle{表达式}

    \begin{itemize}[<+->]
        \item 表达式,是由变量、常量、运算符和分组符号等组成,能通过计算得出数值的式子,如 \lstinline|a * 10 - b|

            \begin{itemize}
                \item 分组符号只可以为\textbf{小括号},如 \lstinline|a * (10 – b)|
                \item 运算符\textbf{不可以}省略不写
                \item 括号使用时,需要成对出现
            \end{itemize}

    \end{itemize}
\end{frame}
%------------------------------------------------------------

%------------------------------------------------------------
\begin{frame}[fragile]
    \frametitle{随堂练习}

    \begin{exampleblock}{选择题}

        \begin{enumerate}
            \item 以下选项中出现的字母或单词均为变量或常量,那么以下选项中在 C++ 中属于合法的表达式的有

                \alt<2>{
                    \begin{tasks}
                        \task[A.] \lstinline|2(a + b)|
                        \task[B.] \lstinline|[(a + b) * h] / 2|
                        \task[C.] \lstinline|2pi * r|
                        \task[D.] \textcolor{red}{\lstinline|pi * r * r|}
                    \end{tasks}
                }{
                    \begin{tasks}
                        \task[A.] \lstinline|2(a + b)|
                        \task[B.] \lstinline|[(a + b) * h] / 2|
                        \task[C.] \lstinline|2pi * r|
                        \task[D.] \lstinline|pi * r * r|
                    \end{tasks}
                }

        \end{enumerate}

    \end{exampleblock}
\end{frame}
%------------------------------------------------------------

%------------------------------------------------------------
\begin{frame}[fragile]
    \frametitle{表达式}

    \begin{itemize}
        \item 在表达式中,参与运算不会改变变量的值

        \item 对于表达式的结果,我们可以直接输出,或存储后再使用

            \uncover<2-> {
                \alt<3> {
                    \lstinputlisting[basicstyle=\ttfamily\scriptsize,language=C++,name=expression2]{ch03/expression2.cc}
                }{
                    \lstinputlisting[basicstyle=\ttfamily\scriptsize,language=C++,name=expression1]{ch03/expression1.cc}
                }
            }
    \end{itemize}
\end{frame}
%------------------------------------------------------------

%------------------------------------------------------------
\begin{frame}[fragile]
    \frametitle{运算的优先级}

    \begin{itemize}
        \item 在表达式求值时,必须了解各种运算及其优先顺序

            \begin{itemize}
                \item 优先顺序从高到低排列:

                    \begin{multicols}{2}
                        \begin{enumerate}
                            \item 括号
                            \item 乘法 / 除法 / 取模
                            \item 加法 / 减法
                            \item 关系运算
                            \item 逻辑运算
                            \item 赋值运算
                        \end{enumerate}
                    \end{multicols}

                \item 同等优先顺序的两个运算则按照从左到右进行计算

                    \begin{itemize}
                        \item 除了赋值运算和逻辑非
                    \end{itemize}

            \end{itemize}

    \end{itemize}
\end{frame}
%------------------------------------------------------------


\section{赋值运算}

%------------------------------------------------------------
\begin{frame}[fragile]
    \frametitle{赋值运算}

    \begin{table}[]
        \begin{tabular}{ccc}
            \toprule
            操作     & 操作符         & 解释                                             \\
            \midrule
            赋值     & \lstinline|=|  & 将结果赋予变量                                   \\
            加法赋值 & \lstinline|+=| & \lstinline|a += b|,等价于 \lstinline|a = a + b| \\
            减法赋值 & \lstinline|–=| & \lstinline|a -= b|,等价于 \lstinline|a = a - b| \\
            乘法赋值 & \lstinline|*=| & \lstinline|a *= b|,等价于 \lstinline|a = a * b| \\
            除法赋值 & \lstinline|/=| & \lstinline|a /= b|,等价于 \lstinline|a = a / b| \\
            取模赋值 & \lstinline|%=| & \lstinline|a %= b|,等价于 \lstinline|a = a % b| \\
            \bottomrule
        \end{tabular}
    \end{table}
\end{frame}
%------------------------------------------------------------

%------------------------------------------------------------
\begin{frame}[fragile]
    \frametitle{随堂练习}

    \begin{exampleblock}{填空题}

        \begin{enumerate}
            \item 在各表达式互相独立、不相互影响的情况下,写出执行下列表达式后各变量的值。(a、b 的初始值分别为 21、5)

                \begin{itemize}
                    \item \lstinline|a + b;| \alt<2>{\textcolor{red}{\tabto{8em} $a = 21$ \tabto{12em} $b = 5$}}{}
                    \item \lstinline|a += b;| \alt<2>{\textcolor{red}{\tabto{8em} $a = 26$ \tabto{12em} $b = 5$}}{}
                    \item \lstinline|b = a / b;| \alt<2>{\textcolor{red}{\tabto{8em} $a = 21$ \tabto{12em} $b = 4$}}{}
                    \item \lstinline|a %= b;| \alt<2>{\textcolor{red}{\tabto{8em} $a = 1$ \tabto{12em} $b = 5$}}{}
                \end{itemize}

        \end{enumerate}

    \end{exampleblock}
\end{frame}
%------------------------------------------------------------

%------------------------------------------------------------
\begin{frame}[fragile]
    \frametitle{例 3.1:交换两个变量的值}

    \alt<2>{
        \lstinputlisting[basicstyle=\ttfamily\scriptsize,language=C++,name=swap]{ch03/swap.cc}
    }{
        \begin{exampleblock}{编程题}

            \begin{itemize}
                \item 输入两个整数 $a$, $b$ ($1 \le a, b \le 10^3$),输出交换后的整数。

                \item 样例输入

                    \lstinline|3 5|

                \item 样例输出

                    \lstinline|5 3|

            \end{itemize}

        \end{exampleblock}
    }
\end{frame}
%------------------------------------------------------------

%------------------------------------------------------------
\begin{frame}[fragile]
    \frametitle{自增自减}

    \begin{table}[]
        \begin{tabular}{ccc}
            \toprule
            操作 & 操作符         & 解释             \\
            \midrule
            自增 & \lstinline|++| & 变量的值增加 $1$ \\
            自减 & \lstinline|--| & 变量的值减少 $1$ \\
            \bottomrule
        \end{tabular}
    \end{table}

    \begin{itemize}
        \item<2-> 自增和自减运算符可放在变量前或变量后,单独使用时,效果都一样

            \begin{itemize}
                \item \lstinline|a++;| 与 \lstinline|++a;| 等价,都表示 \lstinline|a = a + 1;|
                \item \lstinline|a--;| 与 \lstinline|--a;| 等价,都表示 \lstinline|a = a - 1;|
            \end{itemize}

    \end{itemize}
\end{frame}
%------------------------------------------------------------

%------------------------------------------------------------
\begin{frame}[fragile]
    \frametitle{自增自减}

    \begin{itemize}
        \item 自增和自减运算符参与其他操作时,运算符的前后位置会影响结果

            \begin{itemize}
                \item<2-> \lstinline|b = a++;|

                    \begin{itemize}
                        \item 先赋值,再 \lstinline|a++|
                        \item 相当于 \lstinline|b = a; a = a + 1;|
                    \end{itemize}

                \item<3-> \lstinline|b = ++a;|

                    \begin{itemize}
                        \item 先 \lstinline|++a|,再赋值
                        \item 相当于 \lstinline|a = a + 1; b = a;|
                    \end{itemize}

            \end{itemize}

    \end{itemize}
\end{frame}
%------------------------------------------------------------

%------------------------------------------------------------
\begin{frame}[fragile]
    \frametitle{字符的简单运算}

    \begin{itemize}
        \item 字符类型的变量,也可以进行加减运算

            \begin{itemize}
                \item \lstinline|char ch = 'a';      // 字符类型变量 ch 赋值为字母 a|
                \item \lstinline|ch++;               // 变量 ch 的值加 1|
                \item \lstinline|cout << ch << endl; // 输出字母 b|
            \end{itemize}
    \end{itemize}
\end{frame}
%------------------------------------------------------------

%------------------------------------------------------------
\begin{frame}[fragile]
    \frametitle{例 3.2:输出下一个字母}

    \alt<2>{
        \lstinputlisting[basicstyle=\ttfamily\scriptsize,language=C++,name=next_alpha]{ch03/next_alpha.cc}
    }{
        \begin{exampleblock}{编程题}

            \begin{itemize}
                \item 输入一个 a 到 y 之间的小写字母,输出该字母的下一个字母。

                \item 样例输入

                    \lstinline|a|

                \item 样例输出

                    \lstinline|b|

            \end{itemize}

        \end{exampleblock}
    }
\end{frame}
%------------------------------------------------------------


\section{关系运算}

%------------------------------------------------------------
\begin{frame}[fragile]
    \frametitle{关系运算}

    \begin{table}[]
        \begin{tabular}{cccc}
            \toprule
            操作          & 操作符         & 例子                 & 结果              \\
            \midrule
            判断相等      & \lstinline|==| & \lstinline|20 == 10| & \lstinline|false| \\
            判断不等      & \lstinline|!=| & \lstinline|20 != 10| & \lstinline|true|  \\
            判断大于      & \lstinline|>|  & \lstinline|20 > 10|  & \lstinline|true|  \\
            判断小于      & \lstinline|<|  & \lstinline|20 < 10|  & \lstinline|false| \\
            判断大于等于  & \lstinline|>=| & \lstinline|20 >= 10| & \lstinline|true|  \\
            判断小于等于  & \lstinline|<=| & \lstinline|20 <= 20| & \lstinline|true|  \\
            \bottomrule
        \end{tabular}
    \end{table}
\end{frame}
%------------------------------------------------------------

%------------------------------------------------------------
\begin{frame}[fragile]
    \frametitle{关系运算}

    \begin{itemize}[<+->]
        \item 参与关系运算的元素是任意类型的表达式,结果是 \lstinline|bool| 类型

        \item 可以将关系运算表达式的结果赋值给 \lstinline|bool| 类型变量并输出

            \begin{itemize}
                \item
                    \lstinline|bool f = (20 > 10);|\\
                    \lstinline|cout << f << endl; // 输出结果为 1|

                \item
                    \lstinline|bool f = (10 + 10 != 20);|\\
                    \lstinline|cout << f << endl; // 输出结果为 0|
            \end{itemize}

    \end{itemize}
\end{frame}
%------------------------------------------------------------

%------------------------------------------------------------
\begin{frame}[fragile]
    \frametitle{判断倍数关系}

    \begin{itemize}[<+->]
        \item 如何判断整数 a 是否为 2 的倍数?

            \begin{itemize}
                \item \lstinline|bool f = (a % 2 == 0);|\\
                    \lstinline|cout << f << endl; // 2 的倍数则输出 1,否则输出 0|
            \end{itemize}

    \end{itemize}
\end{frame}
%------------------------------------------------------------

%------------------------------------------------------------
\begin{frame}[fragile]
    \frametitle{例 3.3:4 的倍数}

    \alt<2>{
        \lstinputlisting[basicstyle=\ttfamily\scriptsize,language=C++,name=mod4]{ch03/mod4.cc}
    }{
        \begin{exampleblock}{编程题}

            \begin{itemize}
                \item 输入一个整数 $n$ ($0 \le n \le 10^9$),如果该整数是 $4$ 的倍数,则输出 1,否则输出 0。

                \item 样例输入

                    \lstinline|12|

                \item 样例输出

                    \lstinline|1|

            \end{itemize}

        \end{exampleblock}
    }
\end{frame}
%------------------------------------------------------------


\section{逻辑运算}

%------------------------------------------------------------
\begin{frame}[fragile]
    \frametitle{逻辑运算}

    \begin{itemize}
        \item 参与逻辑运算的操作数是 bool 类型,结果也是 bool 类型

        \item 逻辑运算有以下三种:

            \begin{itemize}
                \item 逻辑与 \lstinline|&&|
                \item 逻辑或 \lstinline||||
                \item 逻辑非 \lstinline|!|
            \end{itemize}

    \end{itemize}
\end{frame}
%------------------------------------------------------------

%------------------------------------------------------------
\begin{frame}[fragile]
    \frametitle{逻辑与\enspace\&\&}

    \begin{itemize}
        \item 逻辑与运算符有“并且”的含义,当两侧的操作数都成立(\lstinline|true|)时,表达式才成立(\lstinline|true|)
    \end{itemize}

    \begin{table}[]
        \begin{tabular}{ccc}
            \toprule
            表达式                     & 操作数运算                 & 结果              \\
            \midrule
            \lstinline|2 > 1 && 2 < 3| & \lstinline|true && true|   & \lstinline|true|  \\
            \lstinline|2 > 1 && 2 > 3| & \lstinline|true && false|  & \lstinline|false| \\
            \lstinline|2 < 1 && 2 < 3| & \lstinline|false && true|  & \lstinline|false| \\
            \lstinline|2 < 1 && 2 > 3| & \lstinline|false && false| & \lstinline|false| \\
            \bottomrule
        \end{tabular}
    \end{table}
\end{frame}
%------------------------------------------------------------

%------------------------------------------------------------
\begin{frame}[fragile]
    \frametitle{逻辑或\enspace||}

    \begin{itemize}
        \item 逻辑或运算符有“或者”的含义,当两侧的操作数有一个成立(\lstinline|true|)时,表达式就成立(\lstinline|true|) 
    \end{itemize}

    \begin{table}[]
        \begin{tabular}{ccc}
            \toprule
            表达式                     & 操作数运算                 & 结果              \\
            \midrule
            \lstinline!2 > 1 || 2 < 3! & \lstinline!true || true!   & \lstinline!true!  \\
            \lstinline!2 > 1 || 2 > 3! & \lstinline!true || false!  & \lstinline!true!  \\
            \lstinline!2 < 1 || 2 < 3! & \lstinline!false || true!  & \lstinline!true!  \\
            \lstinline!2 < 1 || 2 > 3! & \lstinline!false || false! & \lstinline!false! \\
            \bottomrule
        \end{tabular}
    \end{table}
\end{frame}
%------------------------------------------------------------

%------------------------------------------------------------
\begin{frame}[fragile]
    \frametitle{逻辑非\enspace!}

    \begin{itemize}
        \item 逻辑非运算符会将表达式的结果取反
    \end{itemize}

    \begin{table}[]
        \begin{tabular}{ccc}
            \toprule
            表达式                  & 操作数运算             & 结果              \\
            \midrule
            \lstinline|!(2 > 1)|    & \lstinline|!(true)|    & \lstinline|false| \\
            \lstinline|!(2 < 1)|    & \lstinline|!(false)|   & \lstinline|true|  \\
            \lstinline|!(!(2 > 1))| & \lstinline|!(!(true))| & \lstinline|true|  \\
            \bottomrule
        \end{tabular}
    \end{table}
\end{frame}
%------------------------------------------------------------

%------------------------------------------------------------
\begin{frame}[fragile]
    \frametitle{随堂练习}

    \begin{exampleblock}{选择题}

        \begin{enumerate}
            \item 在 C++ 中,判断 a 等于 0 或 b 等于 0 的表达式是

                \alt<2>{
                    \begin{tasks}
                        \task[A.] \lstinline{a != 0 && b != 0}
                        \task[B.] \lstinline{a != 0 || b != 0}
                        \task[C.] \textcolor{red}{\lstinline{a == 0 || b == 0}}
                        \task[D.] \lstinline{a = 0 || b = 0}
                    \end{tasks}
                }{
                    \begin{tasks}
                        \task[A.] \lstinline{a != 0 && b != 0}
                        \task[B.] \lstinline{a != 0 || b != 0}
                        \task[C.] \lstinline{a == 0 || b == 0}
                        \task[D.] \lstinline{a = 0 || b = 0}
                    \end{tasks}
                }

        \end{enumerate}

    \end{exampleblock}
\end{frame}
%------------------------------------------------------------

%------------------------------------------------------------
\begin{frame}[fragile]
    \frametitle{例 3.4:两位数判断}

    \alt<2>{
        \lstinputlisting[basicstyle=\ttfamily\scriptsize,language=C++,name=double_digit]{ch03/double_digit.cc}
    }{
        \begin{exampleblock}{编程题}

            \begin{itemize}
                \item 输入一个整数 $n$ ($0 \le n \le 10^9$),如果该整数是两位数,则输出 1,否则输出 0。

                \item 样例输入

                    \lstinline|10|

                \item 样例输出

                    \lstinline|1|

            \end{itemize}

        \end{exampleblock}
    }
\end{frame}
%------------------------------------------------------------


\section{总结}

%------------------------------------------------------------
\begin{frame}[fragile]
    \frametitle{总结}

    \begin{itemize}
        \item 算术运算
        \item 赋值运算
        \item 关系运算
        \item 逻辑运算
        \item 运算优先级
    \end{itemize}
\end{frame}
%------------------------------------------------------------

%------------------------------------------------------------
\begin{frame}
    \begin{center}
        {\Huge Thank you!}
    \end{center}
\end{frame}
%------------------------------------------------------------

\end{document}
