%------------------------------------------------------------
\title[08 - 结构体 II]
{08 - 结构体 II}

\subtitle{C++ 程序设计进阶}

\author[Beiyu Li]
{Beiyu Li\\
\texttt{<sysulby@gmail.com>}}

% \institute[SOJ]
% {Sicily Online Judge}

\date[\today]
{\number\year 年 \number\month 月 \number\day 日}
%------------------------------------------------------------


\begin{document}

\author[sysulby]
{SOJ 信息学竞赛教练组}

\begin{frame}
    \titlepage
\end{frame}
\setcounter{framenumber}{0} % 标题页不编号


\section{复习回顾}

%------------------------------------------------------------
\begin{frame}[fragile]
    \frametitle{复习回顾}

    \begin{itemize}
        \item 结构体的定义
        
        \lstinputlisting[basicstyle=\ttfamily\footnotesize,language=C++,name=st_def]{ch20/st_def.cc}
        \begin{tikzpicture}[remember picture, overlay]
            \uncover<3->{\redbox{st_def}{3}{20}{3}{23} node[right,xshift=.2cm,yshift=.2cm]{一般不写 \lstinline|sum = s1 + s2;|};}
            \uncover<4->{\redbox{st_def}{5}{14}{5}{34};}
        \end{tikzpicture}

        \only<2-> {
            \begin{itemize}
                \item<2-> 结构体包含\textbf{成员变量}(属性)和\textbf{成员函数}(行为)
                \item<3-> 成员变量在声明时一般不会进行初始化
                \item<4-> 成员函数可以直接使用结构体内的成员变量
            \end{itemize}
        }
        
    \end{itemize}
    
\end{frame}
%------------------------------------------------------------

%------------------------------------------------------------
\begin{frame}[fragile]
    \frametitle{复习回顾}

    \begin{itemize}
        \item 结构体的使用
        
        \begin{itemize}
            \item <1-> 声明与初始化
            
            \lstinputlisting[basicstyle=\ttfamily\footnotesize,language=C++,name=st_init]{ch20/st_init.cc}
            \vspace{1em}

            \item <2-> 整体赋值
            
            \lstinputlisting[basicstyle=\ttfamily\footnotesize,language=C++,name=st_assign]{ch20/st_assign.cc}
            \vspace{1em}

            \item <3-> 访问成员列表
            
            \lstinputlisting[basicstyle=\ttfamily\footnotesize,language=C++,name=st_mem_list]{ch20/st_mem_list.cc}
            \vspace{1em}

            \lstinputlisting[basicstyle=\ttfamily\footnotesize,language=C++,name=st_mem_list_err]{ch20/st_mem_list_err.cc}
            \begin{tikzpicture}[remember picture, overlay]
                \uncover<4->{\draw[red, very thick] ([shift={(2pt, .25em)}] pic cs:line-st_mem_list_err-1-start) -- ++(8em, 0) |- ([shift={(2pt, .25em)}] pic cs:line-st_mem_list_err-1-start);}
                \uncover<4->{\draw[red, very thick] ([shift={(2pt, .25em)}] pic cs:line-st_mem_list_err-2-start) -- ++(5em, 0) |- ([shift={(2pt, .25em)}] pic cs:line-st_mem_list_err-2-end);}
            \end{tikzpicture}

        \end{itemize}
        
    \end{itemize}
    
\end{frame}
%------------------------------------------------------------

%------------------------------------------------------------
\begin{frame}[fragile]
    \begin{block}{}
        \vspace{.5cm}
        \begin{center}
            上节课使用 \{ \} 可以将结构体进行初始化,\\
            那除此之外还有其他快捷初始化的方法吗?
        \end{center}
        \vspace{.5cm}
    \end{block}
\end{frame}
%------------------------------------------------------------


\section{构造函数}

%------------------------------------------------------------
\begin{frame}[fragile]
    \frametitle{构造函数}

    \only<1-3>{
        \begin{columns}
            \column{.55\textwidth}
            \begin{itemize}
                \item 构造函数常用在声明结构体变量时\textbf{初始化成员变量}
                \item 构造函数的定义
                
                \begin{itemize}
                    \item \lstinline|函数名(参数列表)|
                    \item 构造函数不能写返回值类型
                    \item 构造函数的函数名只能是结构体名
                \end{itemize}

                \item <3-> 使用结构体声明变量时会自动调用构造函数进行初始化

            \end{itemize}

            \column{.45\textwidth}
            
            \only<1-2> {
                \lstinputlisting[basicstyle=\ttfamily\scriptsize,language=C++,name=st_constr]{ch20/st_constr.cc}
                \begin{tikzpicture}[remember picture, overlay]
                    \uncover<2->{\redbox{st_constr}{5}{3}{7}{19};}
                \end{tikzpicture}
            }

            \only<3-> {
                \lstinputlisting[basicstyle=\ttfamily\scriptsize,language=C++,name=st_constr_2]{ch20/st_constr_2.cc}
                \begin{tikzpicture}[remember picture, overlay]
                    \uncover<3->{\redbox{st_constr_2}{5}{3}{7}{19};}
                    \uncover<3->{\redbox{st_constr_2}{11}{3}{11}{12} node[above,xshift=0cm,yshift=.6cm]{默认调用该构造函数};}
                    \uncover<3->{\draw[red, very thick, ->] ([shift={(2pt, .25em)}] pic cs:line-st_constr_2-11-end) -- ++(5em, 0) |- ([shift={(2pt, .25em)}] pic cs:line-st_constr_2-6-end);}
                \end{tikzpicture}
            }

        \end{columns}
    }

    \only<4-> {
        \lstinputlisting[basicstyle=\ttfamily\scriptsize,language=C++,name=st_constr_3]{ch20/st_constr_3.cc}
        \begin{tikzpicture}[remember picture, overlay]
            \uncover<5->{\redbox{st_constr_3}{9}{5}{10}{14};}
        \end{tikzpicture}

        \begin{itemize}
            \item <5> 构造函数可以根据需要初始化成员变量的值
        \end{itemize}
    }
    
\end{frame}
%------------------------------------------------------------

%------------------------------------------------------------
\begin{frame}[fragile]
    \frametitle{构造函数}

        \begin{columns}
            \column{.43\textwidth}
            \begin{itemize}
                \item <1-> 构造函数可以传入参数,用参数初始化成员变量
                \item <3-> 声明结构体变量:
                
                \lstinline|结构体名 变量名(参数)|

                \item <4-> 此时还可以使用 \lstinline|Student b;| 
                
                声明结构体变量吗?

            \end{itemize}

            \column{.57\textwidth}
            
            \only<1-2> {
                \lstinputlisting[basicstyle=\ttfamily\scriptsize,language=C++,name=st_constr_par]{ch20/st_constr_par.cc}
                \begin{tikzpicture}[remember picture, overlay]
                    \uncover<2>{\redbox{st_constr_par}{5}{3}{9}{36};}
                \end{tikzpicture}
            }

            \only<3> {
                \lstinputlisting[basicstyle=\ttfamily\scriptsize,language=C++,name=st_constr_par_2]{ch20/st_constr_par_2.cc}
                \begin{tikzpicture}[remember picture, overlay]
                    \uncover<3>{\redbox{st_constr_par_2}{5}{3}{9}{36};}
                    \uncover<3>{\redbox{st_constr_par_2}{13}{3}{15}{26};}
                    \uncover<3>{\draw[red, very thick, ->] ([shift={(2pt, .25em)}] pic cs:line-st_constr_par_2-13-end) -- ++(3em, 0) -- (5.6, 3.3);}
                \end{tikzpicture}
            }

            \only<4-5> {
                \lstinputlisting[basicstyle=\ttfamily\scriptsize,language=C++,name=st_constr_par_3]{ch20/st_constr_par_3.cc}
                \begin{tikzpicture}[remember picture, overlay]
                    \uncover<4>{\redbox{st_constr_par_3}{14}{3}{14}{12};}
                    \uncover<5>{\redbox{st_constr_par_3}{14}{3}{14}{12} node[above,xshift=.5cm,yshift=1cm]{无法调用该构造函数};}
                    \uncover<5>{\redbox{st_constr_par_3}{5}{3}{9}{36};}
                    \uncover<5>{\draw[red, very thick, ->] ([shift={(2pt, .25em)}] pic cs:line-st_constr_par_3-14-end) -- ++(9em, 0) -- (5.55, 2.9);}
                \end{tikzpicture}
            }

        \end{columns}
    
\end{frame}
%------------------------------------------------------------

%------------------------------------------------------------
\begin{frame}[fragile]
    \frametitle{构造函数}

        \begin{columns}
            \column{.43\textwidth}
            \begin{itemize}
                \item <1-> 一个结构体可以有多个构造函数,但每个构造函数的\textbf{参数列表必须不同}
                \item <3-> 声明结构体变量的写法不同,会调用不同的构造函数
            \end{itemize}

            \column{.57\textwidth}
            
            \lstinputlisting[basicstyle=\ttfamily\scriptsize,language=C++,name=st_constr_par_4]{ch20/st_constr_par_4.cc}
            \begin{tikzpicture}[remember picture, overlay]
                \uncover<2>{\redbox{st_constr_par_4}{5}{3}{5}{11};}
                \uncover<2>{\redbox{st_constr_par_4}{8}{3}{8}{34};}

                \uncover<3>{\redbox{st_constr_par_4}{8}{3}{12}{36};}
                \uncover<3>{\redbox{st_constr_par_4}{16}{3}{16}{26} node[above,xshift=-.5cm,yshift=.5cm]{自动调用该构造函数};}
                \uncover<3>{\draw[red, very thick, ->] ([shift={(2pt, .25em)}] pic cs:line-st_constr_par_4-16-end) -- ++(4em, 0) -- (6, 2.9);}

                \uncover<4>{\redbox{st_constr_par_4}{5}{3}{7}{19};}
                \uncover<4>{\redbox{st_constr_par_4}{17}{3}{17}{12} node[above,xshift=2.3cm,yshift=1cm]{自动调用该构造函数};}
                \uncover<4>{\draw[red, very thick, ->] ([shift={(2pt, .25em)}] pic cs:line-st_constr_par_4-17-end) -- ++(11em, 0) |- ([shift={(2pt, .25em)}] pic cs:line-st_constr_par_4-6-end);}
            \end{tikzpicture}

        \end{columns}
    
\end{frame}
%------------------------------------------------------------

%------------------------------------------------------------
\begin{frame}[fragile]
    \frametitle{例 8.1:计算学生的成绩}

    \alt<2> {
        \lstinputlisting[basicstyle=\ttfamily\scriptsize,language=C++,name=calc]{ch20/calc.cc}
    } {
        \begin{exampleblock}{编程题}
            \begin{itemize}
                \item 编写一个 Student 结构体,成员变量包含姓名 $name$、学号 $id$、分数 $score$($name$ 是字符串,$id$、$score$ 是整数)
                        使用构造函数解决以下问题:

                        已知班级里有 $n$ ($1 \leq n \leq 100$) 个学生,并且学生的平均分为 $60$ 分。每个学生与平均分都有一定的差距 $d$,
                        $d$ 为正数表示比平均分高多少分,负数表示比平均分低多少分。
                        
                        最后输出 $n$ 个学生的 $name$、$id$、$score$
                
                \begin{columns}
                    \column{.04\textwidth}        

                    \column{.46\textwidth}        

                    \item 样例输入
        
                        \lstinline|3|\\
                        \lstinline|Tom 1 4|\\
                        \lstinline|Lucy 2 -3|\\
                        \lstinline|Ken 3 -1|
        
                    \column{.50\textwidth}        

                    \item 样例输出
                    
                        \lstinline|Tom 1 64|\\
                        \lstinline|Lucy 2 57|\\
                        \lstinline|Ken 3 59|

                \end{columns}
    
            \end{itemize}
        \end{exampleblock}
    }

\end{frame}
%------------------------------------------------------------


\section{运算符函数}

%------------------------------------------------------------
\begin{frame}[fragile]
    \frametitle{运算符函数}

    \begin{itemize}
        \item <1-> 已知两个 string 类型的变量可以通过 < 进行比较
        
        \lstinputlisting[basicstyle=\ttfamily\scriptsize,language=C++,name=calc_def_1]{ch20/calc_def_1.cc}

        \item <2-> 当比较两个结构体变量的大小关系时,可以直接使用 < 吗?
        
        \lstinputlisting[basicstyle=\ttfamily\scriptsize,language=C++,name=calc_def_2]{ch20/calc_def_2.cc}
        \begin{tikzpicture}[remember picture, overlay]
            \uncover<3>{\draw[red, very thick] ([shift={(2pt, .25em)}] pic cs:line-calc_def_2-3-start) -- ++(4em, 0) |- ([shift={(2pt, .25em)}] pic cs:line-calc_def_2-3-start);}
        \end{tikzpicture}

    \end{itemize}

\end{frame}
%------------------------------------------------------------

%------------------------------------------------------------
\begin{frame}[fragile]
    \frametitle{运算符函数}

    \begin{itemize}
        \item <1-> 运算符函数 使得结构体类型变量支持符号运算

        \item <2-> 常被定义(或重载)的运算符函数有:
        
        \begin{itemize}
            \item 算术运算符:\lstinline|+ - * / %|
            \item 关系运算符:\textcolor{red}{\lstinline|< >|} \lstinline|<= >= == !=|
        \end{itemize}

    \end{itemize}

\end{frame}
%------------------------------------------------------------

%------------------------------------------------------------
\begin{frame}[fragile]
    \frametitle{运算符函数}

    \begin{itemize}
        \item <1-> 运算符函数的函数头与普通函数相似,不同的是:
        
        \begin{itemize}
            \item 函数名由 \textcolor{red}{operator 运算符} 组成
            \item 参数列表往往需要 \textcolor{red}{const} 和 \textcolor{red}{\&}
        \end{itemize}

        \lstinputlisting[basicstyle=\ttfamily\scriptsize,language=C++,name=calc_def]{ch20/calc_def.cc}
        \begin{tikzpicture}[remember picture, overlay]
            \uncover<2->{\redbox{calc_def}{4}{20}{4}{24};}
            \uncover<2->{\redbox{calc_def}{4}{34}{4}{35} node[below,xshift=-1cm,yshift=-.3cm]{保证形参不被修改};}
            \uncover<2->{\draw[red, very thick, ->] (3.6,2) -- (3.6,2.3) -- (5.7,2.3) -- (5.7,2);}

            \uncover<3->{\redbox{calc_def}{2}{3}{3}{16};}
            \uncover<3->{\redbox{calc_def}{4}{38}{4}{42} node[above,xshift=-1cm,yshift=1cm]{保证自己的成员变量不被修改};}
            \uncover<3->{\draw[red, very thick, ->] (6.7,2) -- (6.7,2.5) -- (2.8,2.5);}
        \end{tikzpicture}

    \end{itemize}

\end{frame}
%------------------------------------------------------------

%------------------------------------------------------------
\begin{frame}[fragile]
    \frametitle{< 运算符函数}

    \begin{itemize}
        \item 运算符函数的功能是设计比较规则
        
        \only<1> {\lstinputlisting[basicstyle=\ttfamily\scriptsize,language=C++,name=calc_def_rule]{ch20/calc_def_rule.cc}}
        \only<2> {\lstinputlisting[basicstyle=\ttfamily\scriptsize,language=C++,name=calc_def_rule_2]{ch20/calc_def_rule_2.cc}}
        
    \end{itemize}

\end{frame}
%------------------------------------------------------------

%------------------------------------------------------------
\begin{frame}[fragile]
    \frametitle{例 8.2:比较两个学生的大小关系}

    \alt<2-5> {
        \only<2> {\lstinputlisting[basicstyle=\ttfamily\scriptsize,language=C++,name=compare]{ch20/compare.cc}}
        \only<3> {\lstinputlisting[basicstyle=\ttfamily\scriptsize,language=C++,name=compare_2]{ch20/compare_2.cc}}
        \only<4-5> {
            \lstinputlisting[basicstyle=\ttfamily\scriptsize,language=C++,name=compare_3]{ch20/compare_3.cc}

            \begin{itemize}
                \uncover<5> {\item 重载的运算符只有 <,所以不能使用 >、<= 等进行比较大小}
            \end{itemize}
        }
    } {
        \begin{exampleblock}{编程题}
            \begin{itemize}
                \item 编写一个 Student 结构体,成员变量包含姓名 $name$、学号 $id$、分数 $score$。

                        输入两个学生的信息,使用 < 比较两个结构体变量,输出其中分数较小的学生的名字   

                \item 样例输入
    
                    \lstinline|Tom 1 89|\\
                    \lstinline|Lucy 2 75|    

                \item 样例输出
                
                    \lstinline|Lucy|
    
            \end{itemize}
        \end{exampleblock}
    }

\end{frame}
%------------------------------------------------------------

%------------------------------------------------------------
\begin{frame}[fragile]
    \frametitle{> 运算符函数}

    \begin{itemize}
        \item 当想要使用 > 比较两个结构体变量时,必须重载 > 运算符函数
        
        \lstinputlisting[basicstyle=\ttfamily\scriptsize,language=C++,name=calc_def_rule_3]{ch20/calc_def_rule_3.cc}

        \item<2> 该运算符函数比较的是什么?
        
    \end{itemize}

\end{frame}
%------------------------------------------------------------

%------------------------------------------------------------
\begin{frame}[fragile]
    \frametitle{其他运算符函数}

    \begin{itemize}
        \item 同理,当需要判断两个结构体变量的 == 或 >= 等关系时,也可以重载对应的运算符函数
        
        \only<1> {\lstinputlisting[basicstyle=\ttfamily\scriptsize,language=C++,name=calc_def_rule_4]{ch20/calc_def_rule_4.cc}}
        \only<2> {\lstinputlisting[basicstyle=\ttfamily\scriptsize,language=C++,name=calc_def_rule_5]{ch20/calc_def_rule_5.cc}}
        \only<3> {\lstinputlisting[basicstyle=\ttfamily\scriptsize,language=C++,name=calc_def_rule_6]{ch20/calc_def_rule_6.cc}}

    \end{itemize}

\end{frame}
%------------------------------------------------------------

%------------------------------------------------------------
\begin{frame}[fragile]
    \frametitle{多个比较规则的运算符函数}

    \only<1>{
        \begin{block}{}
            \vspace{.5cm}
            \begin{center}
                当两个学生结构体变量需要先比较总分谁大,\\
                如果总分相同再比较姓名字典序谁大时,该怎么办?
            \end{center}
            \vspace{.5cm}
        \end{block}
    }

    \only<2-4>{
        \begin{itemize}
            \item 可以重载多个运算符函数来进行比较吗?
            
            \begin{itemize}
                \item<2-> 重载多个不同的运算符函数是可以的
                \item<3-> 但不可以多次重载相同的运算符函数
            \end{itemize}

            \item<4-> 理论上可行,但有没有更简单的写法?
        \end{itemize}

        \only<1-3> {
            \lstinputlisting[basicstyle=\ttfamily\scriptsize,language=C++,name=calc_defs_1]{ch20/calc_defs_1.cc}
            \begin{tikzpicture}[remember picture, overlay]
                \only<2> {\redbox{calc_defs_1}{4}{3}{9}{45};}
                \only<3> {
                    \redbox{calc_defs_1}{4}{3}{4}{17};
                    \redbox{calc_defs_1}{10}{3}{10}{17};
                }
            \end{tikzpicture}
        }
        
        \only<4> {
            \lstinputlisting[basicstyle=\ttfamily\scriptsize,language=C++,name=calc_defs_2]{ch20/calc_defs_2.cc}
            \begin{tikzpicture}[remember picture, overlay]
                \only<4> {\redbox{calc_defs_2}{10}{3}{12}{37};}
            \end{tikzpicture}
        } 
    }

    \only<5-6> {
        \begin{itemize}
            \item 可以将多个比较规则合并到一个运算符函数里:
            
            \only<5> {\lstinputlisting[basicstyle=\ttfamily\scriptsize,language=C++,name=calc_defs_3]{ch20/calc_defs_3.cc}}
            \only<6> {\lstinputlisting[basicstyle=\ttfamily\scriptsize,language=C++,name=calc_defs_4]{ch20/calc_defs_4.cc}}

        \end{itemize}
    }

\end{frame}
%------------------------------------------------------------

%------------------------------------------------------------
\begin{frame}[fragile]
    \frametitle{例 8.3:成绩最优异的学生}

    \alt<2> {
        \lstinputlisting[basicstyle=\ttfamily\scriptsize,language=C++,name=mx_stu]{ch20/mx_stu.cc}
    } {
        \begin{exampleblock}{编程题}
            \begin{itemize}
                \item 编写一个 Student 结构体,成员变量包含姓名 $name$、学号 $id$、分数 $score$。
                定义一个 > 运算符函数,用于比较学生分数的关系,找到 $n$ ($1 \leq n \leq 100$) 个学生里分数最高的学生。\\
                如果有多个同学的分数并列最高,则输出学号最大的学生的名字

                \item 样例输入
    
                    \lstinline|3|\\
                    \lstinline|Tom 3 90|\\
                    \lstinline|Lucy 1 90|\\
                    \lstinline|Ken 2 85|

                \item 样例输出
                
                    \lstinline|Tom|
    
            \end{itemize}
        \end{exampleblock}
    }

\end{frame}
%------------------------------------------------------------

%------------------------------------------------------------
\begin{frame}[fragile]
    \frametitle{例 8.4:时间运算}

    \alt<2-3> {
        \only<2> {\lstinputlisting[basicstyle=\ttfamily\scriptsize,language=C++,name=time_1]{ch20/time_1.cc}}
        \only<3> {
            \lstinputlisting[basicstyle=\ttfamily\scriptsize,language=C++,name=time_2]{ch20/time_2.cc}
            \begin{tikzpicture}[remember picture, overlay]
                \redbox{time_2}{8}{5}{8}{20};
            \end{tikzpicture}
        }
    } {
        \begin{exampleblock}{编程题}
            \begin{itemize}
                \item 给出 $n$ 个时间,每个时间用小时和分钟表示。\\
                    希望你计算所有时间的和并输出。

                \item 样例输入
    
                    \lstinline|4|\\
                    \lstinline|1 15|\\
                    \lstinline|0 56|\\
                    \lstinline|5 12|\\
                    \lstinline|3 8|

                \item 样例输出
                
                    \lstinline|10 31|
    
            \end{itemize}
        \end{exampleblock}
    }

\end{frame}
%------------------------------------------------------------


\section{总结}

%------------------------------------------------------------
\begin{frame}[fragile]
    \frametitle{总结}

    \begin{itemize}
        \item 构造函数
        
        \begin{itemize}
            \item 有、无参构造函数
        \end{itemize}

        \item 运算符函数
        
        \begin{itemize}
            \item \lstinline|<、>、+| 运算符函数
        \end{itemize}

        \item 运算符函数应用
        
        \begin{itemize}
            \item 多个比较规则
            \item 结构体数组找最值
        \end{itemize}

    \end{itemize}
            
\end{frame}
%------------------------------------------------------------

\end{document}