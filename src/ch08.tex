%------------------------------------------------------------
\title[08 - 多重循环]
{08 - 多重循环}

\subtitle{C++ 程序设计基础}

\author[Beiyu Li]
{Beiyu Li\\
\texttt{<sysulby@gmail.com>}}

% \institute[SOJ]
% {Sicily Online Judge}

\date[\today]
{\number\year 年 \number\month 月 \number\day 日}
%------------------------------------------------------------


\begin{document}

\author[sysulby]
{SOJ 信息学竞赛教练组}

\begin{frame}
    \titlepage
\end{frame}
\setcounter{framenumber}{0} % 标题页不编号


\section{复习回顾}

%------------------------------------------------------------
\begin{frame}[fragile]
    \frametitle{例题回顾 - 加减数列求和}

    \alt<2>{
        \lstinputlisting[language=C++,name=cal_seq1]{ch08/cal_seq1.cc}
    }{
        \begin{exampleblock}{编程题}

            \begin{itemize}
                \item 编写程序,输入一个整数 $n$ ($1 \leq n \leq 10^6$),求 $1 - 2 + 3 - 4 + 5 ...$,一直到 $n$ 的和。

                \item 样例输入

                    \lstinline|7|

                \item 样例输出

                    \lstinline|4|

                \item 解释说明

                    $1 - 2 + 3 - 4 + 5 - 6 + 7 = 4$

            \end{itemize}

        \end{exampleblock}
    }
\end{frame}
%------------------------------------------------------------

%------------------------------------------------------------
\begin{frame}[fragile]
    \frametitle{讨论}

    \begin{block}{}
        \vspace{.5cm}
        \begin{center}
            {\large 既然循环体中可以写顺序结构和分支结构,\\那么是否可以写循环结构呢?}
        \end{center}
        \vspace{.5cm}
    \end{block}

\end{frame}
%------------------------------------------------------------


\section{多重循环}

%------------------------------------------------------------
\begin{frame}[fragile]
    \frametitle{多重循环}

    \begin{itemize}[<+->]
        \item 一重循环
        \item 二重循环

            \begin{itemize}
                \item 在一重循环的循环体中再写一个循环
            \end{itemize}

        \item 三重循环
        \item ...
    \end{itemize}
\end{frame}
%------------------------------------------------------------

%------------------------------------------------------------
\begin{frame}[fragile]
    \frametitle{例8.1.1:输出一行 m 个星号}

    \alt<2>{
        \lstinputlisting[language=C++,name=star_m]{ch08/star_m.cc}
    }{
        \begin{exampleblock}{编程题}

            \begin{itemize}
                \item 编写程序,输入一个整数 $m$ ($1 \le n \le 1000$),输出一行 $m$ 个星号。

                \item 样例输入

                    \lstinline|5|

                \item 样例输出\\*****

            \end{itemize}

        \end{exampleblock}
    }

\end{frame}
%------------------------------------------------------------

%------------------------------------------------------------
\begin{frame}[fragile]
    \frametitle{例8.1.2:输出 n 行 m 个星号}


    \begin{exampleblock}{编程题}

        \begin{itemize}
            \item 编写程序,输入两个整数 $n,m$ ($1 \le n,m \le 1000$),输出 $n$ 行 $m$ 个星号。

            \item 样例输入

                \lstinline|3 5|

            \item 样例输出\\*****\\*****\\*****

        \end{itemize}

    \end{exampleblock}


\end{frame}
%------------------------------------------------------------

%------------------------------------------------------------
\begin{frame}[fragile]
    \frametitle{示例:双重循环}

    \begin{columns}
        \column{.01\textwidth}

        \column{.70\textwidth}
        \lstinputlisting[language=C++,name=star_n_m]{ch08/star_n_m.cc}
        \begin{tikzpicture}[remember picture, overlay]

            \uncover<2>{\redbox{star_n_m-8}{9}{10}}
            \uncover<3,17,31>{\redbox{star_n_m-8}{20}{7}}
            \uncover<16,30>{\redbox{star_n_m-8}{28}{3}}
            \uncover<4,18>{\redbox{star_n_m-9}{13}{10}}
            \uncover<5,8,11,14,19,22,25,28>{\redbox{star_n_m-9}{24}{7}}
            \uncover<7,10,13,21,24,27>{\redbox{star_n_m-9}{32}{3}}
            \uncover<6,9,12,20,23,26>{\redbox{star_n_m-10}{12}{12}}
            \uncover<15,29>{\redbox{star_n_m-12}{8}{13}}
            \uncover<32>{\redbox{star_n_m-14}{4}{9}}

        \end{tikzpicture}

        \column{.30\textwidth}
        \begin{itemize}
                \uncover<2-> {
                \item 输入

                    \lstinline|2 3|

                \item 变量

                    \only<2>{\textcolor{red}{\lstinline|i = 1|}}
                    \only<3-15>{\lstinline|i = 1|}
                    \only<16>{\textcolor{red}{\lstinline|i = 2|}}
                    \only<17-29>{\lstinline|i = 2|}
                    \only<30>{\textcolor{red}{\lstinline|i = 3|}}
                    \only<31>{\textcolor{red}{\lstinline|i <= 2| 不成立}}

                    \only<4,18>{\textcolor{red}{\lstinline|j = 1|}}
                    \only<5-6,19-20>{\lstinline|j = 1|}
                    \only<7,21>{\textcolor{red}{\lstinline|j = 2|}}
                    \only<8-9,22-23>{\lstinline|j = 2|}
                    \only<10,24>{\textcolor{red}{\lstinline|j = 3|}}
                    \only<11-12,25-26>{\lstinline|j = 3|}
                    \only<13,27>{\textcolor{red}{\lstinline|j = 4|}}
                    \only<14,28>{\textcolor{red}{\lstinline|j <= 3| 不成立}}

                }

                \uncover<6-> {
                \item 输出

                    \uncover<6->{\lstinline|*|}
                    \uncover<9->{\lstinline|*|}
                    \uncover<12->{\lstinline|*|\\}
                    \uncover<20->{\lstinline|*|}
                    \uncover<23->{\lstinline|*|}
                    \uncover<26->{\lstinline|*|}

                }
        \end{itemize}
    \end{columns}
\end{frame}
%------------------------------------------------------------


\section{多重循环的应用}

%------------------------------------------------------------
\begin{frame}[fragile]
    \frametitle{例8.2:输出 n 行从 1 到 m 的整数}

    \alt<2>{
        \lstinputlisting[language=C++,name=num_n_m]{ch08/num_n_m.cc}
    }{
        \begin{exampleblock}{编程题}

            \begin{itemize}
                \item 编写程序,输入两个整数 $n,m$ ($1 \le n,m \le 1000$),输出 $n$ 行从 $1$ 到 $m$ 的整数。

                \item 样例输入

                    \lstinline|2 4|

                \item 样例输出

                    \lstinline|1 2 3 4|\\
                    \lstinline|1 2 3 4|

            \end{itemize}

        \end{exampleblock}
    }

\end{frame}
%------------------------------------------------------------

%------------------------------------------------------------
\begin{frame}[fragile]
    \frametitle{例8.3:输出二重循环的循环变量}

    \alt<2>{
        \lstinputlisting[language=C++,name=num_ij]{ch08/num_ij.cc}
    }{
        \begin{exampleblock}{编程题}

            \begin{itemize}
                \item 在【例 8.2】的基础上,输出二重循环的循环变量 \lstinline|(i,j)| 的值,每次输出以空格间隔。

                \item 样例输入

                    \lstinline|2 4|

                \item 样例输出

                    \lstinline|(1,1) (1,2) (1,3) (1,4)|\\
                    \lstinline|(2,1) (2,2) (2,3) (2,4)|

            \end{itemize}

        \end{exampleblock}
    }

\end{frame}
%------------------------------------------------------------

%------------------------------------------------------------
\begin{frame}[fragile]
    \frametitle{例 8.4:输出二重循环的循环变量}
    \begin{itemize}
        \item 当输入 n 为 3,m 为 4 时,输出如图所示\\
            \lstinline|3 4|\\
            \lstinline|(1,1) (1,2) (1,3) (1,4)|\ \ \ \ \  
            % i
            \only<2>{\textcolor{red}{\lstinline|i = 1|}}
            \only<3>{\textcolor{red}{\lstinline|i = 2|}}
            \only<4>{\textcolor{red}{\lstinline|i = 3|}}
            % j
            \only<5>{\textcolor{red}{\lstinline|j = 1|}}
            \only<6>{\textcolor{red}{\lstinline|j = 2|}}
            \only<7>{\textcolor{red}{\lstinline|j = 3|}}
            \only<8>{\textcolor{red}{\lstinline|j = 4|}}
            \\
            \lstinline|(2,1) (2,2) (2,3) (2,4)|\\
            \lstinline|(3,1) (1,2) (1,3) (1,4)|
        \item i 表示所在的行号
        \item j 表示所在的列号
    \end{itemize}

    \begin{tikzpicture}[overlay]

        \begin{columns}
            \column{.50\textwidth}
            % i
            \only<2>{\draw[red, very thick] (0.3,3.00) rectangle (5.1,3.43);}
            \only<3>{\draw[red, very thick] (0.3,2.50) rectangle (5.1,2.93);}
            \only<4>{\draw[red, very thick] (0.3,2.03) rectangle (5.1,2.46);}
            % j
            \only<5>{\draw[red, very thick] (0.3,3.43) rectangle (1.5,2);}
            \only<6>{\draw[red, very thick] (1.5,3.43) rectangle (2.6,2);}
            \only<7>{\draw[red, very thick] (2.6,3.43) rectangle (3.7,2);}
            \only<8>{\draw[red, very thick] (3.7,3.43) rectangle (4.9,2);}
        \end{columns}

    \end{tikzpicture}

\end{frame}
%------------------------------------------------------------

%------------------------------------------------------------
\begin{frame}[fragile]
    \frametitle{例8.5:输出数字直角三角形}

    \alt<2>{
        \lstinputlisting[language=C++,name=num_triangle]{ch08/num_triangle.cc}
    }{
        \begin{exampleblock}{编程题}

            \begin{itemize}
                \item 编写程序,输入一个整数 $n$ ($1 \leq n \leq 10^3$),输出一个等腰直角三角形,腰长为 $n$,每行由整数 $1, 2, 3 ...$ 组成,整数之间用空格间隔。

                \item 样例输入

                    \lstinline|3|

                \item 样例输出

                    \lstinline|1|\\
                    \lstinline|1 2|\\
                    \lstinline|1 2 3|

            \end{itemize}

        \end{exampleblock}
    }

\end{frame}
%------------------------------------------------------------

%------------------------------------------------------------
\begin{frame}[fragile]
    \frametitle{例8.6:累加数列求和}

    \alt<2-3>{
        \lstinputlisting[language=C++,name=cal_seq2]{ch08/cal_seq2.cc}
        \begin{tikzpicture}[remember picture, overlay]
            \only<3>{\redbox{cal_seq2-11}{8}{7}}
        \end{tikzpicture}
    }{
        \begin{exampleblock}{编程题}

            \begin{itemize}
                \item 已知 $A_i = 1 + 2 + 3 + ... + i$,$S_n = A_1 + A_2 + A_3 + ... + A_n$。\\
                    编写程序,输入一个整数 $n$ ($1 \leq n \leq 10^4$),计算 $S_n$ 的结果并输出。

                \item 样例输入

                    \lstinline|3|

                \item 样例输出

                    \lstinline|10|

                \item 样例解释\\

                    $A_1 = 1$\\
                    $A_2 = 1 + 2$\\
                    $A_3 = 1 + 2 + 3$\\
                    $S_3 = A_1 + A_2 + A_3$

            \end{itemize}

        \end{exampleblock}
    }

\end{frame}
%------------------------------------------------------------

%------------------------------------------------------------
\begin{frame}[fragile]
    \frametitle{例8.6:累加数列求和 - 代码优化}
    \lstinputlisting[language=C++,name=cal_seq3]{ch08/cal_seq3.cc}
\end{frame}
%------------------------------------------------------------

%------------------------------------------------------------
\begin{frame}[fragile]
    \frametitle{例8.7:空心矩形}

    \alt<2-3>{
        \lstinputlisting[language=C++,name=rectangle_hollow]{ch08/rectangle_hollow.cc}
        \begin{tikzpicture}[remember picture, overlay]
            \only<3>{\redbox{rectangle_hollow-10}{16}{36}}
        \end{tikzpicture}
    }{
        \begin{exampleblock}{编程题}

            \begin{itemize}
                \item 编写程序,输入两个整数 $n,m$ ($1 \leq n,m \leq 10^3$),\\输出一个 $n$ 行 $m$ 列的空心矩形,矩形由星号形成。

                \item 样例输入

                    \lstinline|4 7|

                \item 样例输出\\

                    \lstinline|*******|\\
                    \lstinline|* \ \ \ \ \*|\\
                    \lstinline|* \ \ \ \ \*|\\
                    \lstinline|*******|

            \end{itemize}

        \end{exampleblock}
    }

\end{frame}
%------------------------------------------------------------

%------------------------------------------------------------
\begin{frame}[fragile]
    \frametitle{例8.8:等腰三角形}

    \alt<2>{
        \lstinputlisting[language=C++,name=triangle_isosceles]{ch08/triangle_isosceles.cc}
    }{
        \begin{exampleblock}{编程题}

            \begin{itemize}
                \item 编写程序,输入一个整数 $n$ ($1 \leq n \leq 20$),\\输出高为 $n$ 的由 * 号组成的等腰三角形。

                \item 样例输入

                    \lstinline|3|

                \item 样例输出\\

                    \lstinline| \ \ \*|\\
                    \lstinline| \ \***|\\
                    \lstinline| \*****|

            \end{itemize}

        \end{exampleblock}
    }

\end{frame}
%------------------------------------------------------------

%------------------------------------------------------------
\begin{frame}[fragile]
    \frametitle{例8.9:求一个整数中含有多少个 1}

    \alt<2-6>{
        \alt<2-4> {
            \begin{columns}

                \column{.65\textwidth}
                \lstinputlisting[language=C++,name=cnt_1_1]{ch08/cnt_1_1.cc}

                \column{.35\textwidth}
                \uncover<3->{
                    \begin{itemize}[<+->]
                        \item<3-> 运行结果正确吗?
                        \item<4> 不正确,\\数位拆分后 $n$ 的值会变为 $0$。
                    \end{itemize}
                }
            \end{columns}
        }{
            \begin{columns}

                \column{.65\textwidth}
                \lstinputlisting[language=C++,name=cnt_1_2]{ch08/cnt_1_2.cc}

                \begin{tikzpicture}[remember picture, overlay]
                    \only<6->{\redbox{cnt_1_2-6}{4}{12}}
                \end{tikzpicture}

                \column{.35\textwidth}
                \uncover<5->{
                    \begin{itemize}[<+->]
                        \item<6-> 由于后续需要输出 n,因此不能直接用 n 进行数位拆分
                    \end{itemize}
                }
            \end{columns}
        }
    }{
        \begin{exampleblock}{编程题}

            \begin{itemize}
                \item 编写程序,输入一个整数 $n$ ($1 \leq n \leq 10^9$),\\统计 $n$ 的数位上有多少个 $1$,输出 $n$ 和其含有 $1$ 的个数,用空格间隔。

                \item 样例输入

                    \lstinline|1351|

                \item 样例输出\\

                    \lstinline|1351 2|

            \end{itemize}

        \end{exampleblock}
    }

\end{frame}
%------------------------------------------------------------

%------------------------------------------------------------
\begin{frame}[fragile]
    \frametitle{例8.10:数 1 的个数}

    \alt<2-6>{
        \alt<2-4> {
            \begin{columns}

                \column{.65\textwidth}
                \lstinputlisting[language=C++,name=count_1_1]{ch08/count_1_1.cc}

                \column{.35\textwidth}
                \uncover<3->{
                    \begin{itemize}[<+->]
                        \item<3-> 运行结果正确吗?
                        \item<4> 不正确,会死循环。\\因为数位拆分后 $i$ 会变化!
                    \end{itemize}
                }

            \end{columns}
        }{
            \begin{columns}

                \column{.65\textwidth}
                \lstinputlisting[language=C++,name=count_1_2]{ch08/count_1_2.cc}

                \begin{tikzpicture}[remember picture, overlay]
                    \only<6->{\redbox{count_1_2-7}{8}{12}}
                \end{tikzpicture}

                \column{.35\textwidth}
                \uncover<5->{
                    \begin{itemize}[<+->]
                        \item<6-> 为了避免程序进入死循环,使用临时变量 \lstinline|tmp| 进行数位拆分
                    \end{itemize}
                }

            \end{columns}
        }
    }{
        \begin{exampleblock}{编程题}

            \begin{itemize}
                \item 编写程序,输入一个整数 $n$ ($1 \leq n \leq 10^6$),\\求从 $1$ 到 $n$ 的所有整数中,数位上出现数字 “$1$” 的个数。                

                \item 样例输入

                    \lstinline|12|

                \item 样例输出

                    \lstinline|5|

                \item 样例解释\\

                    当 \lstinline|n = 12| 时,$1, 2, 3, 4, 5, 6, 7, 8, 9, 10, 11, 12$ 中\\数位上出现了 $5$ 个数字 “$1$”。

            \end{itemize}

        \end{exampleblock}
    }

\end{frame}
%------------------------------------------------------------


\section{break 和 continue 语句}

%------------------------------------------------------------
\begin{frame}[fragile]
    \frametitle{break 在多重循环中的应用}

    \uncover<1->{\lstinputlisting[language=C++,name=for_break_1]{ch08/for_break_1.cc}}
    \uncover<3->{\lstinputlisting[language=C++,name=for_break_2]{ch08/for_break_2.cc}}

    \begin{tikzpicture}[remember picture, overlay]
        \uncover<2->{\redbox{for_break_1-4}{12}{6}}
        \uncover<2->{\draw[red, very thick, ->] (3.5,6.4) -- (6.5,6.4) -- (6.5,4.8) -- (0.3,4.8);}
        \uncover<4>{\redbox{for_break_2-3}{8}{6}}
        \uncover<4->{\draw[red, very thick, ->] (2.7,3.05) -- (6.5,3.05) -- (6.5,0.7) -- (0.3,0.7);}
    \end{tikzpicture}

\end{frame}
%------------------------------------------------------------

%------------------------------------------------------------
\begin{frame}[fragile]
    \frametitle{continue 在多重循环中的应用}

    \uncover<1->{\lstinputlisting[language=C++,name=for_continue_1]{ch08/for_continue_1.cc}}
    \uncover<3->{\lstinputlisting[language=C++,name=for_continue_2]{ch08/for_continue_2.cc}}

    \begin{tikzpicture}[remember picture, overlay]
        \uncover<2->{\redbox{for_continue_1-4}{12}{9}}
        \uncover<2->{\draw[red, very thick, ->] (4,6.4) -- (5.6,6.4) -- (5.6,6.8);}
        \uncover<4>{\redbox{for_continue_2-3}{8}{9}}
        \uncover<4->{\draw[red, very thick, ->] (3.3,3.05) -- (4.9,3.05) -- (4.9,3.7);}
    \end{tikzpicture}

\end{frame}
%------------------------------------------------------------


\section{总结}

%------------------------------------------------------------
\begin{frame}[fragile]
    \frametitle{总结}

    \begin{itemize}[<+->]
        \item 多重循环的概念
        \item 多重循环的应用
            \begin{itemize}[<+->]
                \item 输出图形问题
                \item 数列计算问题
                \item 数位拆分问题
            \end{itemize}
    \end{itemize}

\end{frame}
%------------------------------------------------------------

%---------------------------------------------------------
\begin{frame}
    \begin{center}
        {\Huge Thank you!}
    \end{center}
\end{frame}
%---------------------------------------------------------

\end{document}
