%------------------------------------------------------------
\title[08 - 多重循环]
{08 - 多重循环}

\subtitle{C++ 程序设计基础}

\author[Beiyu Li]
{Beiyu Li\\
\texttt{<sysulby@gmail.com>}}

% \institute[SOJ]
% {Sicily Online Judge}

\date[\today]
{\number\year 年 \number\month 月 \number\day 日}
%------------------------------------------------------------


\begin{document}

\author[sysulby]
{SOJ 信息学竞赛教练组}

\begin{frame}
    \titlepage
\end{frame}
\setcounter{framenumber}{0} % 标题页不编号


\section{复习回顾}

%------------------------------------------------------------
\begin{frame}[fragile]
    \frametitle{问题回顾 - 加减数列求和}

    \alt<2>{
        \lstinputlisting[basicstyle=\ttfamily\scriptsize,language=C++,name=alt_sum]{ch07/alt_sum.cc}
    }{
        \begin{exampleblock}{编程题}

            \begin{itemize}
                \item 编写程序,输入一个整数 $n$ ($1 \le n \le 1000$),输出 $1 - 2 + 3 - 4 + 5 - \dots$ 一直到 $n$ 的和。

                \item 样例输入

                    \lstinline|7|

                \item 样例输出

                    \lstinline|4|

                \item 样例说明

                    $1 - 2 + 3 - 4 + 5 - 6 + 7 = 4$

            \end{itemize}

        \end{exampleblock}
    }
\end{frame}
%------------------------------------------------------------

%------------------------------------------------------------
\begin{frame}[fragile]
    \frametitle{讨论}

    \begin{block}{}
        \vspace{.5cm}
        \begin{center}
            {\large 既然循环体中可以写顺序结构和分支结构,\\那么是否可以写循环结构呢?}
        \end{center}
        \vspace{.5cm}
    \end{block}

\end{frame}
%------------------------------------------------------------


\section{多重循环}

%------------------------------------------------------------
\begin{frame}[fragile]
    \frametitle{多重循环}

    \begin{itemize}[<+->]
        \item 一重循环
        \item 二重循环

            \begin{itemize}
                \item 在一重循环的循环体中再写一个循环
            \end{itemize}

        \item 三重循环
        \item ...
    \end{itemize}
\end{frame}
%------------------------------------------------------------

%------------------------------------------------------------
\begin{frame}[fragile]
    \frametitle{例 8.1.1:输出一行 m 个星号}

    \alt<2>{
        \lstinputlisting[basicstyle=\ttfamily\scriptsize,language=C++,name=star_m]{ch08/star_m.cc}
    }{
        \begin{exampleblock}{编程题}

            \begin{itemize}
                \item 编写程序,输入一个整数 $m$ ($1 \le m \le 1000$),输出一行 $m$ 个星号。

                \item 样例输入

                    \lstinline|5|

                \item 样例输出

                    \lstinline|*****|

            \end{itemize}

        \end{exampleblock}
    }
\end{frame}
%------------------------------------------------------------

%------------------------------------------------------------
\begin{frame}[fragile]
    \frametitle{例 8.1.2:输出 n 行 m 个星号}


    \begin{exampleblock}{编程题}

        \begin{itemize}
            \item 编写程序,输入两个整数 $n,m$ ($1 \le n,m \le 1000$),输出 $n$ 行 $m$ 个星号。

            \item 样例输入

                \lstinline|3 5|

            \item 样例输出

                \lstinline|*****|\\
                \lstinline|*****|\\
                \lstinline|*****|

        \end{itemize}

    \end{exampleblock}

\end{frame}
%------------------------------------------------------------

%------------------------------------------------------------
\begin{frame}[fragile]
    \frametitle{示例:双重循环}

    \begin{columns}
        \column{.01\textwidth}

        \column{.63\textwidth}
        \lstinputlisting[basicstyle=\ttfamily\scriptsize,language=C++,name=star_n_m]{ch08/star_n_m.cc}
        \begin{tikzpicture}[remember picture, overlay]

            \uncover<2>{\redbox{star_n_m}{8}{8}{8}{17};}
            \uncover<3,17,31>{\redbox{star_n_m}{8}{19}{8}{25};}
            \uncover<16,30>{\redbox{star_n_m}{8}{27}{8}{29};}
            \uncover<4,18>{\redbox{star_n_m}{9}{10}{9}{19};}
            \uncover<5,8,11,14,19,22,25,28>{\redbox{star_n_m}{9}{21}{9}{27};}
            \uncover<7,10,13,21,24,27>{\redbox{star_n_m}{9}{29}{9}{31};}
            \uncover<6,9,12,20,23,26>{\redbox{star_n_m}{10}{7}{10}{18};}
            \uncover<15,29>{\redbox{star_n_m}{12}{5}{12}{17};}
            \uncover<32>{\redbox{star_n_m}{15}{3}{15}{11};}

        \end{tikzpicture}

        \column{.36\textwidth}
        \begin{itemize}
                \uncover<2-> {
                \item 输入

                    \lstinline|2 3|

                \item 变量

                    \only<2>{\textcolor{red}{\lstinline|i = 1|}}
                    \only<3-15>{\lstinline|i = 1|}
                    \only<16>{\textcolor{red}{\lstinline|i = 2|}}
                    \only<17-29>{\lstinline|i = 2|}
                    \only<30>{\textcolor{red}{\lstinline|i = 3|}}
                    \only<31>{\textcolor{red}{\lstinline|i <= 2| 不成立}}
                    \only<32>{\lstinline||}

                    \only<1-3,15-17,29->{\lstinline||}
                    \only<4,18>{\textcolor{red}{\lstinline|j = 1|}}
                    \only<5-6,19-20>{\lstinline|j = 1|}
                    \only<7,21>{\textcolor{red}{\lstinline|j = 2|}}
                    \only<8-9,22-23>{\lstinline|j = 2|}
                    \only<10,24>{\textcolor{red}{\lstinline|j = 3|}}
                    \only<11-12,25-26>{\lstinline|j = 3|}
                    \only<13,27>{\textcolor{red}{\lstinline|j = 4|}}
                    \only<14,28>{\textcolor{red}{\lstinline|j <= 3| 不成立}}

                }

                \uncover<6-> {
                \item 输出

                    \uncover<6->{\lstinline|*|}
                    \uncover<9->{\lstinline|*|}
                    \uncover<12->{\lstinline|*|\\}
                    \uncover<20->{\lstinline|*|}
                    \uncover<23->{\lstinline|*|}
                    \uncover<26->{\lstinline|*|}

                }
        \end{itemize}
    \end{columns}
\end{frame}
%------------------------------------------------------------


\section{多重循环的应用}

%------------------------------------------------------------
\begin{frame}[fragile]
    \frametitle{例 8.2:输出 n 行从 1 到 m 的整数}

    \alt<2>{
        \lstinputlisting[basicstyle=\ttfamily\scriptsize,language=C++,name=num_n_m]{ch08/num_n_m.cc}
    }{
        \begin{exampleblock}{编程题}

            \begin{itemize}
                \item 编写程序,输入两个整数 $n,m$ ($1 \le n, m \le 1000$),输出 $n$ 行从 $1$ 到 $m$ 的整数。

                \item 样例输入

                    \lstinline|2 4|

                \item 样例输出

                    \lstinline|1 2 3 4|\\
                    \lstinline|1 2 3 4|

            \end{itemize}

        \end{exampleblock}
    }
\end{frame}
%------------------------------------------------------------

%------------------------------------------------------------
\begin{frame}[fragile]
    \frametitle{例 8.3:输出二重循环的循环变量}

    \alt<2-10>{
        \alt<2>{
            \lstinputlisting[basicstyle=\ttfamily\scriptsize,language=C++,name=num_ij]{ch08/num_ij.cc}
        }{
            \begin{itemize}
                \item 当输入 n 为 3,m 为 4 时,输出如图所示

                    \lstinputlisting[basicstyle=\ttfamily\small,numbers=none,name=num_ij_out]{ch08/num_ij.out}

                    \begin{tikzpicture}[remember picture, overlay, scale=1.25]
                        \uncover<4>{\redbox{num_ij_out}{2}{1}{2}{23} node[right,yshift=.22cm]{\lstinline|i = 1|};}
                        \uncover<5>{\redbox{num_ij_out}{3}{1}{3}{23} node[right,yshift=.22cm]{\lstinline|i = 2|};}
                        \uncover<6>{\redbox{num_ij_out}{4}{1}{4}{23} node[right,yshift=.22cm]{\lstinline|i = 3|};}
                        \uncover<7>{\redbox{num_ij_out}{2}{1}{4}{5} node[below,xshift=-.55cm]{\lstinline|j = 1|};}
                        \uncover<8>{\redbox{num_ij_out}{2}{7}{4}{11} node[below,xshift=-.55cm]{\lstinline|j = 2|};}
                        \uncover<9>{\redbox{num_ij_out}{2}{13}{4}{17} node[below,xshift=-.55cm]{\lstinline|j = 3|};}
                        \uncover<10>{\redbox{num_ij_out}{2}{19}{4}{23} node[below,xshift=-.55cm]{\lstinline|j = 4|};}
                    \end{tikzpicture}

                \item<4-> i 表示所在的行号

                \item<7-> j 表示所在的列号

            \end{itemize}
        }
    }{
        \begin{exampleblock}{编程题}

            \begin{itemize}
                \item 在【例 8.2】的基础上,输出二重循环的循环变量 \lstinline|(i,j)| 的值,以空格间隔。

                \item 样例输入

                    \lstinline|2 4|

                \item 样例输出

                    \lstinline|(1,1) (1,2) (1,3) (1,4)|\\
                    \lstinline|(2,1) (2,2) (2,3) (2,4)|

            \end{itemize}

        \end{exampleblock}
    }
\end{frame}
%------------------------------------------------------------

%------------------------------------------------------------
\begin{frame}[fragile]
    \frametitle{例 8.4:输出数字直角三角形}

    \alt<2-3>{
        \lstinputlisting[basicstyle=\ttfamily\scriptsize,language=C++,name=num_triangle]{ch08/num_triangle.cc}

        \begin{tikzpicture}[remember picture, overlay]
            \only<3>{\redbox{num_triangle}{9}{21}{9}{27};}
        \end{tikzpicture}
    }{
        \begin{exampleblock}{编程题}

            \begin{itemize}
                \item 编写程序,输入一个整数 $n$ ($1 \leq n \leq 1000$),输出一个等腰直角三角形,腰长为 $n$,每行由整数 $1, 2, 3 ...$ 组成,以空格间隔。

                \item 样例输入

                    \lstinline|3|

                \item 样例输出

                    \lstinline|1|\\
                    \lstinline|1 2|\\
                    \lstinline|1 2 3|

            \end{itemize}

        \end{exampleblock}
    }
\end{frame}
%------------------------------------------------------------

%------------------------------------------------------------
\begin{frame}[fragile]
    \frametitle{例 8.5:累加数列求和}

    \alt<2-3>{
        \lstinputlisting[basicstyle=\ttfamily\scriptsize,language=C++,name=cal_seq1]{ch08/cal_seq1.cc}

        \begin{tikzpicture}[remember picture, overlay]
            \only<3>{\redbox{cal_seq1}{10}{5}{10}{15};}
        \end{tikzpicture}
    }{
        \begin{exampleblock}{编程题}

            \begin{itemize}
                \item 已知 $A_i = 1 + 2 + 3 + ... + i$,$S_n = A_1 + A_2 + A_3 + ... + A_n$。\\
                    编写程序,输入一个整数 $n$ ($1 \leq n \leq 1000$),计算 $S_n$ 的结果并输出。

                \item 样例输入

                    \lstinline|3|

                \item 样例输出

                    \lstinline|10|

                \item 样例说明

                    $A_1 = 1$\\
                    $A_2 = 1 + 2 = 3$\\
                    $A_3 = 1 + 2 + 3 = 6$\\
                    $S_3 = A_1 + A_2 + A_3 = 1 + 3 + 6 = 10$

            \end{itemize}

        \end{exampleblock}
    }
\end{frame}
%------------------------------------------------------------

%------------------------------------------------------------
\begin{frame}[fragile]
    \frametitle{例 8.5:累加数列求和 - 代码优化}

    \lstinputlisting[basicstyle=\ttfamily\scriptsize,language=C++,name=cal_seq2]{ch08/cal_seq2.cc}
\end{frame}
%------------------------------------------------------------

%------------------------------------------------------------
\begin{frame}[fragile]
    \frametitle{例 8.6:空心矩形}

    \alt<2-3>{
        \lstinputlisting[basicstyle=\ttfamily\scriptsize,language=C++,name=rectangle_hollow]{ch08/rectangle_hollow.cc}

        \begin{tikzpicture}[remember picture, overlay]
            \only<3>{\redbox{rectangle_hollow}{10}{11}{10}{46};}
        \end{tikzpicture}
    }{
        \begin{exampleblock}{编程题}

            \begin{itemize}
                \item 编写程序,输入两个整数 $n$, $m$ ($1 \leq n ,m \leq 1000$),输出 $n$ 行 $m$ 列的由空格与星号组成的空心矩形。

                \item 样例输入

                    \lstinline|4 7|

                \item 样例输出

                    \lstinline|*******|\\
                    \lstinline|*\ \ \ \ \ *|\\
                    \lstinline|*\ \ \ \ \ *|\\
                    \lstinline|*******|

            \end{itemize}

        \end{exampleblock}
    }
\end{frame}
%------------------------------------------------------------

%------------------------------------------------------------
\begin{frame}[fragile]
    \frametitle{例 8.7:等腰三角形}

    \alt<2>{
        \lstinputlisting[basicstyle=\ttfamily\scriptsize,language=C++,name=triangle_isosceles]{ch08/triangle_isosceles.cc}
    }{
        \begin{exampleblock}{编程题}

            \begin{itemize}
                \item 编写程序,输入一个整数 $n$ ($1 \leq n \leq 20$),输出高为 $n$ 的由空格与星号组成的等腰三角形。

                \item 样例输入

                    \lstinline|3|

                \item 样例输出

                    \lstinline|\ \ *|\\
                    \lstinline|\ ***|\\
                    \lstinline|*****|

            \end{itemize}

        \end{exampleblock}
    }

\end{frame}
%------------------------------------------------------------

%------------------------------------------------------------
\begin{frame}[fragile]
    \frametitle{例 8.8:求一个整数中含有多少个 1}

    \alt<2-6>{
        \begin{overlayarea}{\textwidth}{.9\textheight}
            \alt<2-4> {
                \begin{columns}
                    \column{.01\textwidth}

                    \column{.55\textwidth}
                    \lstinputlisting[basicstyle=\ttfamily\scriptsize,language=C++,name=cnt_1_1]{ch08/cnt_1_1.cc}

                    \column{.44\textwidth}
                    \uncover<3->{
                        \begin{itemize}[<+->]
                            \item<3-> 运行结果正确吗?
                            \item<4-> 不正确,数位拆分后 $n$ 的值会变为 $0$
                        \end{itemize}
                    }
                \end{columns}
            }{
                \begin{columns}
                    \column{.01\textwidth}

                    \column{.55\textwidth}
                    \lstinputlisting[basicstyle=\ttfamily\scriptsize,language=C++,name=cnt_1_2]{ch08/cnt_1_2.cc}

                    \begin{tikzpicture}[remember picture, overlay]
                        \only<6->{\redbox{cnt_1_2}{8}{16}{8}{23};}
                    \end{tikzpicture}

                    \column{.44\textwidth}
                    \uncover<5->{
                        \begin{itemize}[<+->]
                            \item<6-> 由于后续需要输出 $n$,因此不能直接用 $n$ 进行数位拆分
                        \end{itemize}
                    }
                \end{columns}
            }
        \end{overlayarea}
    }{
        \begin{exampleblock}{编程题}

            \begin{itemize}
                \item 编写程序,输入一个整数 $n$ ($1 \leq n \leq 10^6$),输出 $n$ 及其数位中 $1$ 的个数,用空格间隔。

                \item 样例输入

                    \lstinline|1351|

                \item 样例输出

                    \lstinline|1351 2|

            \end{itemize}

        \end{exampleblock}
    }

\end{frame}
%------------------------------------------------------------

%------------------------------------------------------------
\begin{frame}[fragile]
    \frametitle{例 8.9:数 1 的个数}

    \alt<2-6>{
        \alt<2-4> {
            \begin{columns}
                \column{.01\textwidth}

                \column{.55\textwidth}
                \lstinputlisting[basicstyle=\ttfamily\scriptsize,language=C++,name=count_1_1]{ch08/count_1_1.cc}

                \column{.44\textwidth}
                \uncover<3->{
                    \begin{itemize}
                        \item<3-> 运行结果正确吗?
                        \item<4-> 不正确,会死循环。\\因为数位拆分后 $i$ 会变化!
                    \end{itemize}
                }
            \end{columns}
        }{
            \begin{columns}
                \column{.01\textwidth}

                \column{.55\textwidth}
                \lstinputlisting[basicstyle=\ttfamily\scriptsize,language=C++,name=count_1_2]{ch08/count_1_2.cc}

                \begin{tikzpicture}[remember picture, overlay]
                    \only<6->{\redbox{count_1_2}{10}{5}{10}{16};}
                \end{tikzpicture}

                \column{.44\textwidth}
                \uncover<5->{
                    \begin{itemize}
                        \item<6-> 为了避免程序进入死循环,使用临时变量 \lstinline|tmp| 进行数位拆分
                    \end{itemize}
                }
            \end{columns}
        }
    }{
        \begin{exampleblock}{编程题}

            \begin{itemize}
                \item 编写程序,输入一个整数 $n$ ($1 \leq n \leq 10^9$),统计从 $1$ 到 $n$ 的所有整数中,数位上 $1$ 的个数。

                \item 样例输入

                    \lstinline|12|

                \item 样例输出

                    \lstinline|5|

                \item 样例说明

                    当 \lstinline|n = 12| 时,$1, 2, 3, 4, 5, 6, 7, 8, 9, 10, 11, 12$ 中,数位上 $1$ 的个数为 $5$。

            \end{itemize}

        \end{exampleblock}
    }

\end{frame}
%------------------------------------------------------------


\section{break 和 continue 语句}

%------------------------------------------------------------
\begin{frame}[fragile]
    \frametitle{break 在多重循环中的应用}

    \uncover<1->{\lstinputlisting[basicstyle=\ttfamily\scriptsize,language=C++,name=for_break_1]{ch08/for_break_1.cc}}
    \uncover<3->{\lstinputlisting[basicstyle=\ttfamily\scriptsize,language=C++,name=for_break_2]{ch08/for_break_2.cc}}

    \begin{tikzpicture}[remember picture, overlay]
        \uncover<2->{\redbox{for_break_1}{4}{7}{4}{12};}
        \uncover<2->{\draw[red, very thick, ->] ([shift={(2pt, .25em)}] pic cs:line-for_break_1-4-end) -- ++(4em, 0) |- ([shift={(2pt, .25em)}] pic cs:line-for_break_1-6-end);}
        \uncover<4>{\redbox{for_break_2}{3}{5}{3}{10};}
        \uncover<4>{\draw[red, very thick, ->] ([shift={(2pt, .25em)}] pic cs:line-for_break_2-3-end) -- ++(10em, 0) |- ([shift={(2pt, .25em)}] pic cs:line-for_break_2-9-end);}
    \end{tikzpicture}

\end{frame}
%------------------------------------------------------------

%------------------------------------------------------------
\begin{frame}[fragile]
    \frametitle{continue 在多重循环中的应用}

    \uncover<1->{\lstinputlisting[basicstyle=\ttfamily\scriptsize,language=C++,name=for_continue_1]{ch08/for_continue_1.cc}}
    \uncover<3->{\lstinputlisting[basicstyle=\ttfamily\scriptsize,language=C++,name=for_continue_2]{ch08/for_continue_2.cc}}

    \begin{tikzpicture}[remember picture, overlay]
        \uncover<2->{\redbox{for_continue_1}{4}{7}{4}{15};}
        \uncover<2->{\draw[red, very thick, ->] ([shift={(2pt, .25em)}] pic cs:line-for_continue_1-4-end) -| ([shift={(-.8cm, -.2em)}] pic cs:line-for_continue_1-2-end);}
        \uncover<4>{\redbox{for_continue_2}{3}{5}{3}{13};}
        \uncover<4>{\draw[red, very thick, ->] ([shift={(2pt, .25em)}] pic cs:line-for_continue_2-3-end) -| ([shift={(-.8cm, -.2em)}] pic cs:line-for_continue_2-1-end);}
    \end{tikzpicture}

\end{frame}
%------------------------------------------------------------


\section{总结}

%------------------------------------------------------------
\begin{frame}[fragile]
    \frametitle{总结}

    \begin{itemize}[<+->]
        \item 多重循环的概念
        \item 多重循环的应用
            \begin{itemize}[<+->]
                \item 输出图形问题
                \item 数列计算问题
                \item 数位拆分问题
            \end{itemize}
    \end{itemize}

\end{frame}
%------------------------------------------------------------

%---------------------------------------------------------
\begin{frame}
    \begin{center}
        {\Huge Thank you!}
    \end{center}
\end{frame}
%---------------------------------------------------------

\end{document}
