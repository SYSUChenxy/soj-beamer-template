%------------------------------------------------------------
\title[10 - 递归 I]
{10 - 递归 I}

\subtitle{C++ 程序设计进阶}

\author[Beiyu Li]
{Beiyu Li\\
\texttt{<sysulby@gmail.com>}}

% \institute[SOJ]
% {Sicily Online Judge}

\date[\today]
{\number\year 年 \number\month 月 \number\day 日}
%------------------------------------------------------------


\begin{document}

\author[sysulby]
{SOJ 信息学竞赛教练组}

\begin{frame}
    \titlepage
\end{frame}
\setcounter{framenumber}{0} % 标题页不编号


\section{复习回顾}

%------------------------------------------------------------
\begin{frame}[fragile]
    \frametitle{内存空间与指针}

    \begin{itemize}
        \item C++ 内存划分:栈区、堆区、全局区、代码段
        \item<2-> 指针
            \begin{itemize}
                \item<2-> 存储什么内容?
                \item<3-> 以下代码的含义是什么?
                
                \lstinputlisting[basicstyle=\ttfamily\footnotesize,language=C++,name=pointer]{ch22/pointer.cc}
                \vspace{1em}

                \item<4-> 指针的加减运算
            \end{itemize}
    \end{itemize}
    
\end{frame}
%------------------------------------------------------------


\section{函数调用与递归}

%------------------------------------------------------------
\begin{frame}[fragile]
    \frametitle{问题引入}
    \begin{exampleblock}{编程题}
        \begin{itemize}
            \item 小明班里要去春游,所有人按照学号从小到大排成一队,前面一部分人已经上车出发,
                小明的学号是 $40$ 号,他想知道现在自己在队伍中排第几个   

            \begin{itemize}
                \item<2-> 小明问前面的 $39$ 号:这位同学你排第几个?
                \item<2-> $39$ 号问前面的 $38$ 号:同学你排第几个?
                \item<2-> ...
                \item<2-> $17$ 号问前面的 $16$ 号:同学你排第几个
                \item<3-> $16$ 号回答 $17$ 号:我是第 $1$ 个哦
                \item<4-> $17$ 号回答 $18$ 号:我是第 $2$ 个
                \item<5-> $18$ 号回答 $19$ 号:我是第 $3$ 个
                \item<5-> ...
            \end{itemize}

        \end{itemize}
    \end{exampleblock}

\end{frame}
%------------------------------------------------------------

%------------------------------------------------------------
\begin{frame}[fragile]
    \frametitle{解决问题的思维模式}

    \begin{itemize}
        \item 原问题:学号为 $40$ 号的同学在队伍中的排位是多少?
        \begin{itemize}
            \item<2-> 知道 $39$ 号的排位后 $+ 1$
        \end{itemize}

        \item<3-> $39$ 号在队伍中的排位是多少?
        \begin{itemize}
            \item<4-> 知道 $38$ 号的排位后 $+ 1$
        \end{itemize}

        \item<4-> $...$
        
        \item<4-> $x$ 号在队伍中的排位是多少?
        \begin{itemize}
            \item<5-> 知道 $x-1$ 号的排位后 $+ 1$
        \end{itemize}

        \vspace{1em}

        \item<6> 原问题的求解需要先解决一个同类的子问题,子问题的求解需要先解决一个更小的同类子问题,不断把问题规模分解得更小

    \end{itemize}

\end{frame}
%------------------------------------------------------------

%------------------------------------------------------------
\begin{frame}[fragile]
    \frametitle{设计函数}

    \begin{itemize}
        \item 函数 \lstinline|query()| 的功能是传入参数 $x$,返回学号为 $x$ 的人的排位
        
        \begin{columns}
            \column{.01\textwidth}

            \column{.45\textwidth}
            \begin{itemize}
                \item<2-> 返回前一个人的排位 $+ 1$
                \item<2-> 前一个人的排位就是\\ \lstinline|query(x - 1)| 的返回值
            \end{itemize}
    
            \column{.50\textwidth}
            \only<3->{\lstinputlisting[basicstyle=\ttfamily\footnotesize,language=C++,name=query]{ch22/query.cc}}
        \end{columns}    
    \end{itemize}

    \vspace{1em}

    \begin{columns}
        \column{.01\textwidth}

        \column{.45\textwidth}
        \begin{itemize}
            \item<4-> 函数的调用无穷无尽\\怎么办?
        \end{itemize}

        \column{.50\textwidth}
        \uncover<4->{
            \begin{tikzpicture}[node distance=0.4cm and 0.1cm, % 控制节点之间的距离
                every node/.style={rectangle, draw, rounded corners, align=center, minimum width=1cm, minimum height=0.5cm}  % 设置节点样式
            ]
            
            % 定义节点
            \node (n1) {\lstinline|query(40): query(39) + 1|};
            \node (n2) [below=of n1] {\lstinline|query(39): query(38) + 1|};
            \node (n3) [below=of n2] {\lstinline|query(38): query(37) + 1|};
            \node (n4) [below=of n3] {\lstinline|...|};

            % 绘制连线
            \draw[->] (n1.south) -- (n2);
            \draw[->] (n2.south) -- (n3);
            \draw[->] (n3.south) -- (n4);

            \end{tikzpicture}
        }
    \end{columns}    

\end{frame}
%------------------------------------------------------------

%------------------------------------------------------------
\begin{frame}[fragile]
    \frametitle{函数调用的终止}

    \begin{columns}
        \column{.5\textwidth}
        \begin{itemize}
            \item 需要在某个情况下不再调用函数
            \begin{itemize}
                \item $16$ 号排在第 $1$ 位,不需要再询问前面的人了
            \end{itemize}
            \item 不需要分解为更小的子问题,能直接解答的问题被称为\\\textbf{最简子问题}
        \end{itemize}

        \column{.5\textwidth}
        \begin{tikzpicture}[node distance=0.4cm and 0.1cm, % 控制节点之间的距离
            every node/.style={rectangle, draw, rounded corners, align=center, minimum width=1cm, minimum height=0.5cm}  % 设置节点样式
        ]

        % 定义节点
        \node (n1) {\lstinline|query(40): query(39) + 1|};
        \node (n2) [below=of n1] {\lstinline|query(39): query(38) + 1|};
        \node (n3) [below=of n2] {\lstinline|query(38): query(37) + 1|};
        \node (n4) [below=of n3] {\lstinline|...|};
        \node (n5) [below=of n4] {\lstinline|query(17): query(16) + 1|};
        \node (n6) [below=of n5,color=red] {\lstinline|query(16)|};

        % % 绘制连线
        \draw[->] (n1.south) -- (n2);
        \draw[->] (n2.south) -- (n3);
        \draw[->] (n3.south) -- (n4);
        \draw[->] (n4.south) -- (n5);
        \draw[->] (n5.south) -- (n6);

        \end{tikzpicture}
    \end{columns}

\end{frame}
%------------------------------------------------------------

%------------------------------------------------------------
\begin{frame}[fragile]
    \frametitle{函数调用的递进与回归}

    \begin{columns}
        \column{.5\textwidth}
        \lstinputlisting[basicstyle=\ttfamily\footnotesize,language=C++,name=query_4]{ch22/query_4.cc}
        \vspace{1em}
        \begin{itemize}
            \item<13> 重复将问题分解为更小的同类子问题来解决的方法,称为\\\textbf{递归}
            \item<13> 调用自身的函数,称为\\\textbf{递归函数}
        \end{itemize}

        \column{.5\textwidth}
        \begin{tikzpicture}[node distance=0.4cm and 0.1cm, % 控制节点之间的距离
            every node/.style={rectangle, draw, rounded corners, align=center, minimum width=1cm, minimum height=0.5cm}  % 设置节点样式
        ]

        % 定义节点
        \node (n0) [draw=none] { };
        \uncover<1-> {\node (n1) [below=of n0] {\lstinline|query(40): query(39) + 1|};}
        \uncover<2-11> {\node (n2) [below=of n1] {\lstinline|query(39): query(38) + 1|};}
        \uncover<3-10> {\node (n3) [below=of n2] {\lstinline|query(38): query(37) + 1|};}
        \uncover<4-9> {\node (n4) [below=of n3] {\lstinline|...|};}
        \uncover<5-8> {\node (n5) [below=of n4] {\lstinline|query(17): query(16) + 1|};}
        \uncover<6-7> {\node (n6) [below=of n5] {\lstinline|query(16)|};}

        % % 绘制连线
        \uncover<2-10> {\draw[->] (n1.south) -- (n2.north);}
        \uncover<3-9> {\draw[->] (n2.south) -- (n3.north);}
        \uncover<4-8> {\draw[->] (n3.south) -- (n4.north);}
        \uncover<5-7> {\draw[->] (n4.south) -- (n5.north);}
        \uncover<6> {\draw[->] (n5.south) -- (n6.north);}
        \uncover<7> {\draw[red,->] (n6.north) -- (n5.south) node[midway, right, draw=none] {\lstinline|1|};}
        \uncover<8> {\draw[red,->] (n5.north) -- (n4.south) node[midway, right, draw=none] {\lstinline|2|};}
        \uncover<9> {\draw[red,->] (n4.north) -- (n3.south) node[midway, right, draw=none] {\lstinline|23|};}
        \uncover<10> {\draw[red,->] (n3.north) -- (n2.south) node[midway, right, draw=none] {\lstinline|24|};}
        \uncover<11> {\draw[red,->] (n2.north) -- (n1.south) node[midway, right, draw=none] {\lstinline|25|};}
        \uncover<12-> {\draw[red,->] (n1.north) -- (n0.south) node[midway, right, draw=none] {\lstinline|26|};}

        \end{tikzpicture}
    \end{columns}

\end{frame}
%------------------------------------------------------------

%------------------------------------------------------------
\begin{frame}[fragile]
    \frametitle{递归分析问题的重点}

    \begin{itemize}
        \item 原问题的求解
        \begin{itemize}
            \item 分解为一个或多个性质相同、规模更小的子问题
            \item 解决子问题后,经过组合/操作形成原问题的答案
        \end{itemize}
        \item 最简子问题直接解答,无需分解
    \end{itemize}

\end{frame}
%------------------------------------------------------------

%------------------------------------------------------------
\begin{frame}[fragile]
    \frametitle{递归的代码框架}

    \begin{itemize}
        \item 主要有结束条件和自我调用两部分
        
        \begin{columns}
            \column{.55\textwidth}
            \lstinputlisting[basicstyle=\ttfamily\footnotesize,language=C++,name=re_frame]{ch22/re_frame.cc}
            \column{.45\textwidth}
            \lstinputlisting[basicstyle=\ttfamily\footnotesize,language=C++,name=query_5]{ch22/query_5.cc}
        \end{columns}
    \end{itemize}

\end{frame}
%------------------------------------------------------------


\section{递归解决数列问题}

%------------------------------------------------------------
\begin{frame}[fragile]
    \frametitle{例 10.1:斐波那契数列}
    \only<1> {
        \begin{exampleblock}{编程题}
            \begin{itemize}
                \item 斐波那契数列的定义如下: \\
                    \begin{itemize}
                        \item $f_1=f_2=1$
                        \item $f_n=f_{n-1}+f_{n-2} (n>2)$
                    \end{itemize}
                    该数列前几项的值为: $1, 1, 2, 3, 5, 8, 13, ...$\\
                    输入 $k$ ($1 \leq k \leq 20$),问第 $k$ 项的值是多少?
                
                \item 样例输入
    
                    \lstinline|4|     

                \item 样例输出
                
                    \lstinline|3|
    
            \end{itemize}
        \end{exampleblock}
    }
    \only<2>{
        \begin{itemize}
            \item 由 $f_1=f_2=1$ 可知:\\
                第 $1$ 项和第 $2$ 项是能被直接解答的最简子问题
            \item 由 $f_n=f_{n-1}+f_{n-2}$ 可知:\\
                 $n>2$ 时,求解第 $n$ 项可以分解为求解第 $n-1$ 项和第 $n-2$ 项两个子问题,再相加得到第 $n$ 项的答案
            \begin{itemize}
                \item 子问题和原问题的相同性质:都是求解斐波那契某一项的
                \item 根据相同性质设计函数: \lstinline|fibo(n)| ,返回斐波那契数列第 $n$ 项的值
                \item 求解子问题可以通过调用 \lstinline|fibo()| 函数自身来解决
            \end{itemize}
        \end{itemize}
    }
    \only<3>{
        \lstinputlisting[basicstyle=\ttfamily\footnotesize,language=C++,name=fibo]{ch22/fib.cc}
    }

\end{frame}
%------------------------------------------------------------

%------------------------------------------------------------
\begin{frame}[fragile]
    \frametitle{递归调用的过程}
    \begin{itemize}
        \item 当 $n = 5$ 时,\lstinline|fibo(5)| 的调用过程如下
        
        \vspace{1em}

        \begin{tikzpicture} [
            node distance=0.5cm and 0.1cm, % 控制节点之间的距离
            every node/.style={rectangle, draw, rounded corners, align=center, minimum width=1cm, minimum height=0.5cm}  % 设置节点样式
        ]

        % 定义节点
        \uncover<1-31> {\node (n1) {\lstinline|fibo(5)|};}
        \uncover<2-18,30-31> {\node (n2) [below left=of n1] {\lstinline|fibo(4)|};}
        \uncover<20-28,30-31> {\node (n3) [below right=of n1] {\lstinline|fibo(3)|};}
        \uncover<3-12,30-31> {\node (n4) [below left=of n2] {\lstinline|fibo(3)|};}
        \uncover<14-16,30-31> {\node (n5) [below right=of n2] {\lstinline|fibo(2)|};}
        \uncover<21-23,30-31> {\node (n6) [below left=of n3] {\lstinline|fibo(2)|};}
        \uncover<24-26,30-31> {\node (n7) [below right=of n3] {\lstinline|fibo(1)|};}
        \uncover<4-6,30-31> {\node (n8) [below left=of n4] {\lstinline|fibo(2)|};}
        \uncover<8-10,30-31> {\node (n9) [below right=of n4] {\lstinline|fibo(1)|};}

        % 绘制连线
        \uncover<2-17,30-31> {\draw[red,->] (n1.south) -- (n2);} \uncover<18> {\draw[blue,->] (n2) -- (n1.south);} % back
        \uncover<20-27,30-31> {\draw[red,->] (n1.south) -- (n3);} \uncover<28> {\draw[blue,->] (n3) -- (n1.south);} % back
        \uncover<3-11,30-31> {\draw[red,->] (n2.south) -- (n4);} \uncover<12> {\draw[blue,->] (n4) -- (n2.south);} % back
        \uncover<14-15,30-31> {\draw[red,->] (n2.south) -- (n5);} \uncover<16> {\draw[blue,->] (n5) -- (n2.south);} % back
        \uncover<21-22,30-31> {\draw[red,->] (n3.south) -- (n6);} \uncover<23> {\draw[blue,->] (n6) -- (n3.south);} % back
        \uncover<24-25,30-31> {\draw[red,->] (n3.south) -- (n7);} \uncover<26> {\draw[blue,->] (n7) -- (n3.south);} % back
        \uncover<4-5,30-31> {\draw[red,->] (n4.south) -- (n8);} \uncover<6> {\draw[blue,->] (n8) -- (n4.south);} % back
        \uncover<8-9,30-31> {\draw[red,->] (n4.south) -- (n9);} \uncover<10> {\draw[blue,->] (n9) -- (n4.south);} % back

        % 节点描述
        \uncover<5-6>{\node[below=0.02cm of n8, draw=none, font=\small] {返回值: 1};}
        \uncover<7-10>{\node[below=0.02cm of n4, draw=none, font=\small] {\lstinline|1 + fibo(1)|};} \uncover<11-12>{\node[below=0.02cm of n4, draw=none, font=\small] {返回值: 1+1=2};}
        \uncover<9-10>{\node[below=0.02cm of n9, draw=none, font=\small] {返回值: 1};}
        \uncover<13-16>{\node[below=0.02cm of n2, draw=none, font=\small] {\lstinline|2 + fibo(2)|};} \uncover<17-18>{\node[below=0.02cm of n2, draw=none, font=\small] {返回值: 2+1=3};}
        \uncover<15-16>{\node[below=0.02cm of n5, draw=none, font=\small] {返回值: 1};}
        \uncover<19-28>{\node[below=0.02cm of n1, draw=none, font=\small] {\lstinline|3 + fibo(3)|};} \uncover<29>{\node[below=0.02cm of n1, draw=none, font=\small] {返回值: 3+2=5};} 
        \uncover<22-23>{\node[below=0.02cm of n6, draw=none, font=\small] {返回值: 1};}
        \uncover<24-26>{\node[below=0.02cm of n3, draw=none, font=\small] {\lstinline|1 + fibo(1)|};} \uncover<27-28>{\node[below=0.02cm of n3, draw=none, font=\small] {返回值: 1+1=2};} 
        \uncover<25-26>{\node[below=0.02cm of n7, draw=none, font=\small] {返回值: 1};}

        \end{tikzpicture}

        \item<30-31> 计算效率高不高? 
        \item<31> 不高,递归过程中会有多次重复计算的情况。如 \lstinline|fibo(3)| 就进行了两次计算操作

    \end{itemize}

\end{frame}
%------------------------------------------------------------

%------------------------------------------------------------
\begin{frame}[fragile]
    \frametitle{递归的优缺点}
    \begin{itemize}
        \item 缺点
        \begin{itemize}
            \item 可能出现重复计算,导致计算效率低
            \item 函数调用时要占用一定的栈空间,如果递归层数太多会占用较多内存
        \end{itemize}
        \item 优点
        \begin{itemize}
            \item 结构清晰,可读性强
            \item 增加不同的思维方式,锻炼分析问题的能力
        \end{itemize}
    \end{itemize}

\end{frame}
%------------------------------------------------------------

%------------------------------------------------------------
\begin{frame}[fragile]
    \frametitle{思考}
    \begin{itemize}
        \item 如何不用递归计算斐波那契数列的第 $k$ 项?
    \end{itemize}

    \uncover<2> {\lstinputlisting[basicstyle=\ttfamily\footnotesize,language=C++,name=fib_2]{ch22/fib_2.cc}}

\end{frame}
%------------------------------------------------------------

%------------------------------------------------------------
\begin{frame}[fragile]
    \frametitle{例 10.2:X 数列}
    \only<1> {
        \begin{exampleblock}{编程题}
            \begin{itemize}
                \item X 数列的定义如下: \\
                    \begin{itemize}
                        \item $x_1=1,x_2=2,x_3=3$
                        \item $x_n=x_{n-1}+x_{n-2}+2*x_{n-3}$ ($n>3$)
                    \end{itemize}
                    该数列前几项的值为: $1,2,3,7,14,27,55,...$\\
                    输入 $k$ ($1 \leq k \leq 20$),问第 $k$ 项的值是多少?
                
                \item 样例输入
    
                    \lstinline|4|     

                \item 样例输出
                
                    \lstinline|7|
    
            \end{itemize}
        \end{exampleblock}
    }
    \only<2>{
        \lstinputlisting[basicstyle=\ttfamily\footnotesize,language=C++,name=x]{ch22/x.cc}
    }

\end{frame}
%------------------------------------------------------------


\section{递归解决计算问题}

%------------------------------------------------------------
\begin{frame}[fragile]
    \frametitle{例 10.3:计算阶乘}
    \only<1> {
        \begin{exampleblock}{编程题}
            \begin{itemize}
                \item 阶乘的定义如下: \\
                    \begin{itemize}
                        \item $n! = 1 \times 2 \times 3 \times ... \times (n-2) \times (n-1) \times n$
                    \end{itemize}
                    该数列前几项的值为: $1,2,6,24,120,...$\\
                    输入 $k(1 \leq k \leq 12)$,问 $k!$ 的值是多少?
                
                \item 样例输入
    
                    \lstinline|4|     

                \item 样例输出
                
                    \lstinline|24|
    
            \end{itemize}
        \end{exampleblock}
    }
    \only<2-6> {
        \begin{itemize}
            \only<2> {\item $n! = 1 \times 2 \times 3 \times ... \times (n-2) \times (n-1) \times n$}
            \only<3-> {\item $n! = $ \uline{$1 \times 2 \times 3 \times ... \times (n-2) \times (n-1)$} $ \times n$}
            \item<3-> $n! = $ \uline{$(n-1)!$} $ \times n$
            \item<3-> 计算 $n!$ 可以分解为计算 $(n-1)!$ 这个子问题,再 $\times n$
            \item<4-> 原问题和子问题的相同性质是什么?
            \begin{itemize}
                \item<5-> 都是计算某个数的阶乘
            \end{itemize}
            \item<6> 最简子问题为 $1!=1$
        \end{itemize}
    }
    \only<7> {
        \lstinputlisting[basicstyle=\ttfamily\footnotesize,language=C++,name=k]{ch22/k.cc}
    }

\end{frame}
%------------------------------------------------------------

%------------------------------------------------------------
\begin{frame}[fragile]
    \frametitle{最大公约数}
    \begin{itemize}
        \item 最大公约数,也称为最大公因数,指两个或多个整数共有的约数中最大的一个
        \item 例:
        \begin{itemize}
            \item $24$ 的约数有 $1, 2, 3, 4, 6, 8, 12, 24$
            \item $16$ 的约数有 $1, 2, 4, 8, 16$
            \item $24$ 和 $16$ 的公约数有 $1, 2, 4, 8$
            \item $24$ 和 $16$ 的最大公约数是 $8$
        \end{itemize}
    \end{itemize}

\end{frame}
%------------------------------------------------------------

%------------------------------------------------------------
\begin{frame}[fragile]
    \frametitle{例 10.4:求最大公约数}
    \only<1> {
        \begin{exampleblock}{编程题}
            \begin{itemize}
                \item 输入两个整数 $x$ 和 $y (0<x,y<10^9)$,输出 $x$ 和 $y$ 的最大公约数
                
                \item 样例输入
    
                    \lstinline|24 16|     

                \item 样例输出
                
                    \lstinline|8|
    
            \end{itemize}
        \end{exampleblock}
    }
    \only<2> {
        \begin{itemize}
            \item 辗转相除法是一种效率很高的计算最大公约数的方法
            \item 记 $gcd(a, b)$ 为 $a$ 和 $b$ 的最大公约数,有如下结论:
            
            \[
            gcd(a,b) = 
            \begin{cases}
            a, & \text{$b = 0$} \\
            gcd(b, a \% b), & \text{$b \neq 0$}
            \end{cases}
            \]

            \item 这个式子实际上给出了求解最大公约数的递归解法
            \item 例: $gcd(32, 26) = gcd(26, 6) = gcd(6, 2) = gcd(2, 0) = 2$

        \end{itemize}
        
    }
    \only<3> {
        \lstinputlisting[basicstyle=\ttfamily\footnotesize,language=C++,name=gcd]{ch22/gcd.cc}
    }

\end{frame}
%------------------------------------------------------------


\section{总结}

%------------------------------------------------------------
\begin{frame}[fragile]
    \frametitle{总结}
    \begin{itemize}
        \item 递归的特点
        \begin{itemize}
            \item 函数调用自身
        \end{itemize}
        \item 思维模式
        \begin{itemize}
            \item 将问题分解为性质相同规模更小的子问题,通过调用函数自身得到子问题的答案
        \end{itemize}
        \item 代码框架重点
        \begin{itemize}
            \item 结束条件、自我调用
        \end{itemize}
        \item 应用:数列、阶乘、最大公约数的计算
    \end{itemize}

\end{frame}
%------------------------------------------------------------

\end{document}