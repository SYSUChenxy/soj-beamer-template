%------------------------------------------------------------
\title[04 - 分支结构:if..else 语句]
{04 - 分支结构:if..else 语句}

\subtitle{C++ 程序设计基础}

\author[Beiyu Li]
{Beiyu Li\\
\texttt{<sysulby@gmail.com>}}

% \institute[SOJ]
% {Sicily Online Judge}

\date[\today]
{\number\year 年 \number\month 月 \number\day 日}
%------------------------------------------------------------


\begin{document}

\author[sysulby]
{SOJ 信息学竞赛教练组}

\begin{frame}
    \titlepage
\end{frame}
\setcounter{framenumber}{0} % 标题页不编号


\section{复习回顾}

%------------------------------------------------------------
\begin{frame}[fragile]
    \frametitle{问题回顾:两位数判断}

    \alt<2-3>{
        \begin{columns}
            \column{.01\textwidth}

            \column{.74\textwidth}
            \lstinputlisting[language=C++,name=double_digit]{ch04/double_digit.cc}

            \column{.25\textwidth}
            \uncover<3>{
                \begin{tikzpicture}[node distance=1cm]
                    \node (start) [process] {...};
                    \node (process1) [process, below of=start] {步骤 1};
                    \node (process2) [process, below of=process1] {步骤 2};
                    \node (process3) [process, below of=process2] {...};
                    \node (process4) [process, below of=process3] {步骤 n};
                    \node (stop) [process, below of=process4] {...};

                    \draw [arrow] (start) -- (process1);
                    \draw [arrow] (process1) -- (process2);
                    \draw [arrow] (process2) -- (process3);
                    \draw [arrow] (process3) -- (process4);
                    \draw [arrow] (process4) -- (stop);
                \end{tikzpicture}
            }
        \end{columns}
    }{
        \begin{exampleblock}{编程题}

            \begin{itemize}
                \item 编写程序,由用户输入一个整数 $n$ ($0 \le n \le 10^9$),如果该整数是两位数,则输出 1,否则输出 0。

                \item 样例输入

                    \lstinline|10|

                \item 样例输出

                    \lstinline|1|

            \end{itemize}

        \end{exampleblock}
    }
\end{frame}
%------------------------------------------------------------

%------------------------------------------------------------
\begin{frame}[fragile]
    \frametitle{顺序结构}

    \begin{columns}
        \column{.75\textwidth}
        \begin{itemize}
            \item<1-> 程序按照从上到下的顺序逐条执行各个语句

            \item<2-> 程序单一,无法处理复杂多样的问题

                \begin{itemize}
                    \item  成绩分级
                    \item  电梯运行(接送逻辑)
                \end{itemize}

        \end{itemize}

        \column{.01\textwidth}

        \column{.24\textwidth}
        \begin{tikzpicture}[node distance=1cm]
            \node (start) [process] {...};
            \node (process1) [process, below of=start] {步骤 1};
            \node (process2) [process, below of=process1] {步骤 2};
            \node (process3) [process, below of=process2] {...};
            \node (process4) [process, below of=process3] {步骤 n};
            \node (stop) [process, below of=process4] {...};

            \draw [arrow] (start) -- (process1);
            \draw [arrow] (process1) -- (process2);
            \draw [arrow] (process2) -- (process3);
            \draw [arrow] (process3) -- (process4);
            \draw [arrow] (process4) -- (stop);
        \end{tikzpicture}
    \end{columns}
\end{frame}
%------------------------------------------------------------

%------------------------------------------------------------
\begin{frame}[fragile]
    \frametitle{分支结构}

    \begin{itemize}
        \item<1-> 分支结构:可以根据不同情况,执行不同的语句

        \item<2-> 主要的分支结构有

            \begin{itemize}
                \item \lstinline|if..else| 语句
                \item \lstinline|switch| 语句
                \item 条件运算符
            \end{itemize}

    \end{itemize}
\end{frame}
%------------------------------------------------------------


\section{双分支结构}

%------------------------------------------------------------
\begin{frame}[fragile]
    \frametitle{双分支结构}

    \begin{columns}
        \column{.7\textwidth}
        \begin{overlayarea}{\textwidth}{.6\textheight}
            \begin{itemize}
                \item \lstinline|if..else| 语句:以条件语句作为判断依据,选择执行不同的分支
            \end{itemize}

            \uncover<2->{
                \vspace*{1em}
                \centering
                \begin{tikzpicture}[node distance=1.25cm]
                    \node (start) [process] {...};
                    \node (decision1) [decision, below of=start] {条件语句};
                    \node (process1) [process, below of=decision1, xshift=-2.2cm] {语句块 1};
                    \node (process2) [process, below of=decision1, xshift=2.2cm] {语句块 2};
                    \node (stop) [process, below of=decision1, yshift=-1cm] {...};

                    \draw [arrow] (start) -- (decision1);
                    \draw [arrow] (decision1) -| node[above, xshift=.2cm] {yes} (process1);
                    \draw [arrow] (decision1) -| node[above, xshift=-.2cm] {no} (process2);
                    \draw [arrow] (process1) -| (stop);
                    \draw [arrow] (process2) -| (stop);
                \end{tikzpicture}
            }
        \end{overlayarea}

        \column{.3\textwidth}
        \begin{overlayarea}{\textwidth}{.3\textheight}
            \uncover<3->{
                \lstinputlisting[language=C++,name=if_else]{ch04/if_else.cc}
            }
        \end{overlayarea}
    \end{columns}
\end{frame}
%------------------------------------------------------------

%------------------------------------------------------------
\begin{frame}[fragile]
    \frametitle{if..else 语句}

    \begin{columns}
        \column{.7\textwidth}
        \begin{overlayarea}{\textwidth}{.6\textheight}
            \begin{itemize}
                \item \lstinline|if..else| 语句:以条件语句作为判断依据,选择执行不同的分支

                \item 条件语句通常是布尔类型的表达式

                    \begin{itemize}
                        \item<2-> 条件成立(\lstinline|true|)则执行语句块 1
                        \item<3-> 条件不成立(\lstinline|false|)则执行语句块 2
                    \end{itemize}

                \item<4-> 当语句块中只有一个语句时,可以省略花括号
            \end{itemize}
        \end{overlayarea}

        \column{.3\textwidth}
        \begin{overlayarea}{\textwidth}{.3\textheight}
            \lstinputlisting[language=C++,name=if_else]{ch04/if_else.cc}
        \end{overlayarea}
    \end{columns}
\end{frame}
%------------------------------------------------------------

%------------------------------------------------------------
\begin{frame}[fragile]
    \frametitle{if..else 语句}

    \begin{columns}
        \column{.7\textwidth}
        \begin{overlayarea}{\textwidth}{.6\textheight}
            \begin{itemize}
                \item \lstinline|if..else| 语句:以条件语句作为判断依据,选择执行不同的分支

                \item 条件语句也可以是数值,但会被转化为布尔值

                    \begin{itemize}
                        \item<2-> 数值非 $0$ 值时,会转化成 \lstinline|true|
                        \item<3-> 数值为 $0$ 时,会转化为 \lstinline|false|
                    \end{itemize}

            \end{itemize}
        \end{overlayarea}

        \column{.3\textwidth}
        \begin{overlayarea}{\textwidth}{.3\textheight}
            \lstinputlisting[language=C++,name=if_else]{ch04/if_else.cc}
        \end{overlayarea}
    \end{columns}
\end{frame}
%------------------------------------------------------------

%------------------------------------------------------------
\begin{frame}[fragile]
    \frametitle{例 4.1:4 的倍数}

    \alt<2>{
        \lstinputlisting[language=C++,name=mod4]{ch04/mod4.cc}
    }{
        \begin{exampleblock}{编程题}

            \begin{itemize}
                \item 编写程序,由用户输入一个整数 $n$ ($1 \le n \le 10^9$),如果该整数是 $4$ 的倍数,则输出 Yes,否则输出 No。

                \item 样例输入

                    \lstinline|12|

                \item 样例输出

                    \lstinline|Yes|

            \end{itemize}

        \end{exampleblock}
    }
\end{frame}
%------------------------------------------------------------

%------------------------------------------------------------
\begin{frame}[fragile]
    \frametitle{例 4.2:闰年判断}

    \alt<2-3>{
        \alt<2>{
            \lstinputlisting[language=C++,name=leap1]{ch04/leap1.cc}
        }{
            \lstinputlisting[language=C++,name=leap2]{ch04/leap2.cc}
        }
    }{
        \begin{exampleblock}{编程题}

            \begin{itemize}
                \item 闰年分为世纪闰年和普通闰年。\\
                    普通闰年的年份是 $4$ 的倍数,且不是 $100$ 的倍数;世纪闰年的年份是 $400$ 的倍数。\\
                    编写程序,输入一个整数 $year$ ($1000 \le year \le 3000$),表示一个年份,判断该年份是否为闰年,输出对应的判断结果。

                \item 样例输入

                    \lstinline|2010|

                \item 样例输出

                    \lstinline|2010 is not a leap year|

            \end{itemize}

        \end{exampleblock}
    }
\end{frame}
%------------------------------------------------------------

%------------------------------------------------------------
\begin{frame}[fragile]
    \frametitle{条件运算符}

    \begin{itemize}[<+->]
        \item 条件运算符也是双分支结构的一种形式

            \begin{itemize}
                \item \textbf{条件语句? \enspace 表达式1: \enspace 表达式2}
            \end{itemize}

        \item 依照条件语句是否成立,取不同的结果 

            \begin{itemize}
                \item 条件成立(\lstinline|true|)则取表达式 1 的结果
                \item 条件不成立(\lstinline|false|)则取表达式 2 的结果
                \item \lstinline|int a = 3, b = 5;|\\
                    \lstinline{cout << (a < b? a: b) << endl; // 输出 3}
            \end{itemize}

        \item 注意,应尽量保证两个表达式运算结果的数据类型是一致的
    \end{itemize}
\end{frame}
%------------------------------------------------------------

%------------------------------------------------------------
\begin{frame}[fragile]
    \frametitle{随堂练习}

    \begin{exampleblock}{填空题}

        \begin{enumerate}
                \only<1-2>{
                \item[1.] 阅读程序写结果
                    \lstinputlisting[language=C++,name=exercise1]{ch04/exercise1.cc}

                    输出:\uncover<2>{\textcolor{red}{\lstinline|false|}}
                }

                \only<3-4>{
                \item[2.] 阅读程序写结果
                    \lstinputlisting[language=C++,name=exercise2]{ch04/exercise2.cc}

                    输出:\uncover<4>{\textcolor{red}{\lstinline|true|}}
                }

                \only<5-6>{
                \item[3.] 阅读程序写结果
                    \lstinputlisting[language=C++,name=exercise3]{ch04/exercise3.cc}

                    输出:\uncover<6>{\textcolor{red}{\lstinline|true|}}
                }
        \end{enumerate}

    \end{exampleblock}
\end{frame}
%------------------------------------------------------------


\section{单分支与多分支}

%------------------------------------------------------------
\begin{frame}[fragile]
    \frametitle{单分支结构}

    \begin{columns}
        \column{.64\textwidth}
        \begin{itemize}[<+->]
            \item 单分支结构没有 \lstinline|else| 部分

                \begin{itemize}
                    \item 条件语句成立时,执行语句块
                    \item 条件语句不成立时,跳过语句块
                \end{itemize}

        \end{itemize}

        \column{.36\textwidth}
        \lstinputlisting[language=C++,name=if]{ch04/if.cc}
    \end{columns}
\end{frame}
%------------------------------------------------------------

%------------------------------------------------------------
\begin{frame}[fragile]
    \frametitle{随堂练习}

    \begin{exampleblock}{填空题}

        \begin{enumerate}
            \item 阅读程序写结果
                \lstinputlisting[language=C++,name=days]{ch04/days.cc}

                输入:2020 \tabto{8em} 输出:\uncover<2->{\textcolor{red}{\lstinline|366|}}

                输入:1900 \tabto{8em} 输出:\uncover<3->{\textcolor{red}{\lstinline|365|}}
        \end{enumerate}

    \end{exampleblock}
\end{frame}
%------------------------------------------------------------

%------------------------------------------------------------
\begin{frame}[fragile]
    \frametitle{多分支结构}

    \begin{columns}
        \column{.54\textwidth}
        \begin{itemize}[<+->]
            \item 多分支结构会依次判断条件语句

                \begin{itemize}
                    \item 当前条件成立时,执行对应的语句块,并结束判断
                \end{itemize}

            \item 因此至多只会执行一个语句块

            \item 注意,最后的 \lstinline|else| 不用写条件
        \end{itemize}

        \column{.46\textwidth}
        \lstinputlisting[language=C++,name=if_else_if]{ch04/if_else_if.cc}
    \end{columns}
\end{frame}
%------------------------------------------------------------

%------------------------------------------------------------
\begin{frame}[fragile]
    \frametitle{例 4.3:成绩分级}

    \alt<2-3>{
        \alt<2>{
            \lstinputlisting[language=C++,name=grade1]{ch04/grade1.cc}
        }{
            \lstinputlisting[language=C++,name=grade2]{ch04/grade2.cc}
        }
    }{
        \begin{exampleblock}{编程题}
            \begin{columns}[onlytextwidth]
                \column{.66\textwidth}
                \begin{itemize}
                    \item 编写程序,输入一个整数 $score$ ($0 \le score \le 100$),表示成绩,输出对应的等级。

                    \item 样例输入

                        \lstinline|90|

                    \item 样例输出

                        \lstinline|A|

                \end{itemize}

                \column{.02\textwidth}

                \column{.32\textwidth}
                \begin{tabular}{cc}
                    \toprule
                    分数          & 等级 \\
                    \midrule
                    $90 \sim 100$ & A    \\
                    $80 \sim 89$  & B    \\
                    $60 \sim 79$  & C    \\
                    $0 \sim 59$   & F    \\
                    \bottomrule
                \end{tabular}
            \end{columns}
        \end{exampleblock}
    }
\end{frame}
%------------------------------------------------------------


\section{分支嵌套}

%------------------------------------------------------------
\begin{frame}[fragile]
    \frametitle{多分支与多个分支}

    \begin{itemize}
        \item<1-> 多分支结构条件不能同时成立,最多只会执行一个分支
        \item<2-> 多个分支结构条件可能同时成立,符合条件就执行对应的分支,可执行多个分支
    \end{itemize}

    \begin{columns}
        \column{.08\textwidth}

        \column{.48\textwidth}
        \uncover<1->{
            \lstinputlisting[language=C++,name=if_else_if]{ch04/if_else_if.cc}
        }

        \column{.02\textwidth}

        \column{.42\textwidth}
        \uncover<2->{
            \lstinputlisting[language=C++,name=if_x_3]{ch04/if_x_3.cc}
        }
    \end{columns}
\end{frame}
%------------------------------------------------------------

%------------------------------------------------------------
\begin{frame}[fragile]
    \frametitle{多分支与多个分支}

    \begin{itemize}
        \item 当 $score$ 为 $95$ 时,分别会输出什么?
    \end{itemize}

    \begin{columns}
        \column{.08\textwidth}

        \column{.45\textwidth}
        \lstinputlisting[language=C++,name=grade3]{ch04/grade3.cc}

        \column{.02\textwidth}

        \column{.45\textwidth}
        \lstinputlisting[language=C++,name=grade4]{ch04/grade4.cc}
    \end{columns}
\end{frame}
%------------------------------------------------------------

%------------------------------------------------------------
\begin{frame}[fragile]
    \frametitle{分支嵌套}

    \begin{columns}
        \column{.55\textwidth}
        \begin{itemize}[<+->]
            \item 在分支结构的语句块中,使用分支结构,这就形成了分支嵌套
            \item 注意,应尽量不省略大括号,以明确各层次的嵌套关系
        \end{itemize}

        \column{.45\textwidth}
        \lstinputlisting[language=C++,name=if_nest]{ch04/if_nest.cc}
    \end{columns}
\end{frame}
%------------------------------------------------------------

%------------------------------------------------------------
\begin{frame}[fragile]
    \frametitle{例 4.4:数字判断}

    \alt<2>{
        \lstinputlisting[language=C++,name=judge]{ch04/judge.cc}
    }{
        \begin{exampleblock}{编程题}

            \begin{itemize}
                \item 输入一个整数 $n$,按以下规则输出对应描述 $n$ 的语句。
                    \begin{itemize}
                        \item 当 $n$ 是正数时,输出 $n$ 是奇数 odd 还是偶数 even
                        \item 当 $n$ 是 $0$ 时,输出 zero
                        \item 当 $n$ 是负数时,输出 negative
                    \end{itemize}

                \item 样例输入

                    \lstinline|5|

                \item 样例输出

                    \lstinline|odd|

            \end{itemize}

        \end{exampleblock}
    }
\end{frame}
%------------------------------------------------------------

%------------------------------------------------------------
\begin{frame}[fragile]
    \frametitle{例 4.5:求三个数的最大值}

    \alt<2>{
        \lstinputlisting[language=C++,name=max_of_three]{ch04/max_of_three.cc}
    }{
        \begin{exampleblock}{编程题}

            \begin{itemize}
                \item 编写程序,由用户输入三个整数 $a, b, c$ ($0 \le a, b, c \le 1000$),输出这三个数中的最大值。

                \item 样例输入

                    \lstinline|4 29 16|

                \item 样例输出

                    \lstinline|29|

            \end{itemize}

        \end{exampleblock}
    }
\end{frame}
%------------------------------------------------------------


\section{总结}

%------------------------------------------------------------
\begin{frame}[fragile]
    \frametitle{总结}

    \begin{itemize}
        \item 双分支结构
        \item 单分支结构
        \item 多分支结构
        \item 分支嵌套
    \end{itemize}
\end{frame}
%------------------------------------------------------------

%------------------------------------------------------------
\begin{frame}
    \begin{center}
        {\Huge Thank you!}
    \end{center}
\end{frame}
%------------------------------------------------------------

\end{document}
