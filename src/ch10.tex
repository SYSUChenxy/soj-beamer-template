%------------------------------------------------------------
\title[10 - 一维数组的应用]
{10 - 一维数组的应用}

\subtitle{C++ 程序设计基础}

\author[Beiyu Li]
{Beiyu Li\\
\texttt{<sysulby@gmail.com>}}

% \institute[SOJ]
% {Sicily Online Judge}

\date[\today]
{\number\year 年 \number\month 月 \number\day 日}
%------------------------------------------------------------


\begin{document}

\author[sysulby]
{SOJ 信息学竞赛教练组}

\begin{frame}
    \titlepage
\end{frame}
\setcounter{framenumber}{0} % 标题页不编号


\section{复习回顾}

%------------------------------------------------------------
\begin{frame}[fragile]
    \frametitle{数组的声明}

    \begin{overlayarea}{\textwidth}{.55\textheight}
        \begin{itemize}
            \item 数组的声明

                \begin{itemize}
                    \item \textbf{元素类型 \enspace 数组名[数组大小];}
                    \item \lstinline|int a[8]; // 定义了可以存放 8 个整数的数组 a|
                \end{itemize}

                \begin{itemize}
                    \item 数组中的基本单元叫作“元素”
                    \item 数组大小定义为可能存放元素数量的最大值(可适当再大一些)
                \end{itemize}

                \begin{itemize}
                    \item 建议声明数组为全局变量
                \end{itemize}
        \end{itemize}
    \end{overlayarea}

    \begin{columns}
        \column{.01\textwidth}

        \column{.16\textwidth}
        \lstinline|int a[8];|

        \column{.83\textwidth}
        \begin{tikzpicture}
            [nodes in empty cells, nodes={minimum width=1cm, minimum height=.75cm}, row sep=-\pgflinewidth, column sep=-\pgflinewidth]
            \matrix(a) [matrix of nodes, ampersand replacement=\&, row 1/.style={nodes={draw=none}}, nodes={draw, anchor=center}]{
                \lstinline|a[0]| \& \lstinline|a[1]| \& \lstinline|a[2]| \& \lstinline|a[3]| \& \lstinline|a[4]| \& \lstinline|a[5]| \& \lstinline|a[6]| \& \lstinline|a[7]| \\
                \lstinline|71|   \& \lstinline|80|   \& \lstinline|62|   \& \lstinline|91|   \& \lstinline|99|   \& \lstinline|82|   \& \lstinline|43|   \& \lstinline|53|   \\
            };
        \end{tikzpicture}
    \end{columns}
\end{frame}
%------------------------------------------------------------

%------------------------------------------------------------
\begin{frame}[fragile]
    \frametitle{数组元素的访问}

    \begin{overlayarea}{\textwidth}{.55\textheight}
        \begin{itemize}
            \item 数组元素的访问

                \begin{itemize}
                    \item 通过 \textbf{数组名 [下标]} 访问存储在数组中 某一位置 的值
                        \begin{itemize}
                            \item 下标是指在元素在数组中的位置,从 $0$ 开始
                            \item 下标的范围:$0 \sim size - 1$ ($size$ 为数组大小)
                        \end{itemize}
                \end{itemize}

                \begin{itemize}
                    \item 例如:声明数组 \lstinline|double b[5];|
                        \begin{itemize}
                            \item 可以使用的数组下标:$0$, $1$, $2$, $3$, $4$
                            \item 对应的可以使用的数组元素:\lstinline|b[0], b[1], b[2], b[3], b[4]|
                            \item 可以使用变量作为数组下标:\lstinline|b[k]| ($0 \le k \le 4$)
                        \end{itemize}
                \end{itemize}
        \end{itemize}
    \end{overlayarea}

    \begin{columns}
        \column{.01\textwidth}

        \column{.16\textwidth}
        \lstinline|int a[8];|

        \column{.83\textwidth}
        \begin{tikzpicture}
            [nodes in empty cells, nodes={minimum width=1cm, minimum height=.75cm}, row sep=-\pgflinewidth, column sep=-\pgflinewidth]
            \matrix(a) [matrix of nodes, ampersand replacement=\&, row 1/.style={nodes={draw=none}}, nodes={draw, anchor=center}]{
                \lstinline|a[0]| \& \lstinline|a[1]| \& \lstinline|a[2]| \& \lstinline|a[3]| \& \lstinline|a[4]| \& \lstinline|a[5]| \& \lstinline|a[6]| \& \lstinline|a[7]| \\
                \lstinline|71|   \& \lstinline|80|   \& \lstinline|62|   \& \lstinline|91|   \& \lstinline|99|   \& \lstinline|82|   \& \lstinline|43|   \& \lstinline|53|   \\
            };
        \end{tikzpicture}
    \end{columns}
\end{frame}
%------------------------------------------------------------

%------------------------------------------------------------
\begin{frame}[fragile]
    \frametitle{数组的遍历}

    \begin{overlayarea}{\textwidth}{.66\textheight}
        \begin{itemize}
            \item 遍历:对数组中所有元素逐个访问一遍的过程称为遍历

            \item 遍历输入数组

                \begin{itemize}
                    \item<1-> 存储在下标 $0 \sim n - 1$ 中
                    \item<1-> \lstinline|for (int i = 0; i < n; i++) {|\\
                        \lstinline|  cin >> a[i];|\\
                        \lstinline|}|
                    \item<2-> 存储在下标 $1 \sim n$ 中
                    \item<2-> \lstinline|for (int i = 1; i <= n; i++) {|\\
                        \lstinline|  cin >> a[i];|\\
                        \lstinline|}|
                \end{itemize}

        \end{itemize}
    \end{overlayarea}
\end{frame}
%------------------------------------------------------------

%------------------------------------------------------------
\begin{frame}[fragile]
    \frametitle{数组的遍历}

    \begin{overlayarea}{\textwidth}{.66\textheight}
        \begin{itemize}
            \item 遍历:对数组中所有元素逐个访问一遍的过程称为遍历

            \item 遍历输出数组

                \begin{itemize}
                    \item<1-> 存储在下标 $0 \sim n - 1$ 中
                    \item<1-> \lstinline|for (int i = 0; i < n; i++) {|\\
                        \lstinline|  cout << a[i] << " ";|\\
                        \lstinline|}|\\
                        \lstinline|cout << endl;|\\
                    \item<2-> 存储在下标 $1 \sim n$ 中
                    \item<2-> \lstinline|for (int i = 1; i <= n; i++) {|\\
                        \lstinline|  cout << a[i] << " ";|\\
                        \lstinline|}|\\
                        \lstinline|cout << endl;|\\
                \end{itemize}

        \end{itemize}
    \end{overlayarea}
\end{frame}
%------------------------------------------------------------

%------------------------------------------------------------
\begin{frame}[fragile]
    \frametitle{例 10.1:求数组元素的最大值及所有位置}

    \alt<2-3>{
        \lstinputlisting[basicstyle=\ttfamily\scriptsize,language=C++,name=array_max_all_pos]{ch10/array_max_all_pos.cc}

        \begin{tikzpicture}[remember picture, overlay]
            \uncover<3>{\redbox{array_max_all_pos}{17}{5}{17}{41};}
        \end{tikzpicture}
    }{
        \begin{exampleblock}{编程题}

            \begin{itemize}
                \item 编写程序,输入一个整数 $n$ ($1 \le n \le 100$),表示有 $n$ 个整数,接下来输入 $n$ 个整数存储在数组中,求这 $n$ 个数组元素中的最大值及所有位置。\\
                    第一行输出最大值,第二行以空格间隔输出所有位置。

                \item 样例输入

                    \lstinline|9|\\
                    \lstinline|2 9 9 7 9 2 4 5 8|

                \item 样例输出

                    \lstinline|9|\\
                    \lstinline|2 3 5|

            \end{itemize}

        \end{exampleblock}
    }
\end{frame}
%------------------------------------------------------------


\section{数组计数}

%------------------------------------------------------------
\begin{frame}[fragile]
    \frametitle{例 10.2:投票选班长 I}

    \alt<10>{
        \lstinputlisting[basicstyle=\ttfamily\scriptsize,language=C++,name=vote_1]{ch10/vote_1.cc}
    }{
        \begin{exampleblock}{编程题}

            \alt<1>{
                \begin{itemize}
                    \item 6 年 A 班在选班长,有 $5$ 位候选人,编号 $1 \sim 5$,共 $n$ 名同学将进行投票。\\
                        编写程序,第一行输入整数 $n$ ($1 \le n \le 100$),如上所述,第二行输入 $n$ 个整数 $x$ ($1 \le x \le 5$),表示每位同学的投票结果。\\
                        按候选人的编号顺序输出 $5$ 行,每行输出候选人编号及票数,用空格间隔,表示得票结果。
                \end{itemize}

                \begin{columns}[onlytextwidth,T]
                    \column{.5\textwidth}
                    \begin{itemize}
                        \item 样例输入

                            \lstinline|6|\\
                            \lstinline|1 1 4 5 1 4|

                    \end{itemize}

                    \column{.5\textwidth}
                    \begin{itemize}
                        \item 样例输出

                            \lstinline|1 3|\\
                            \lstinline|2 0|\\
                            \lstinline|3 0|\\
                            \lstinline|4 2|\\
                            \lstinline|5 1|

                    \end{itemize}
                \end{columns}
            }{
                \begin{columns}[onlytextwidth,T]
                    \column{.5\textwidth}
                    \begin{itemize}
                        \item 样例输入

                            \lstinline|6|\\
                            \lstinline|1 1 4 5 1 4|

                    \end{itemize}

                    \column{.5\textwidth}
                    \uncover<9->{
                        \begin{itemize}
                            \item 样例输出

                                \lstinline|1 3|\\
                                \lstinline|2 0|\\
                                \lstinline|3 0|\\
                                \lstinline|4 2|\\
                                \lstinline|5 1|

                        \end{itemize}
                    }
                \end{columns}

                \begin{itemize}
                    \item 样例说明

                    \begin{tikzpicture}
                        [nodes in empty cells, nodes={minimum width=.9cm, minimum height=.7cm}, row sep=-\pgflinewidth, column sep=-\pgflinewidth]
                        \only<2>{
                            \matrix(cnt) [matrix of nodes, ampersand replacement=\&, row 1/.style={nodes={draw=none}}, nodes={draw, anchor=center}]{
                                \lstinline|1| \& \lstinline|2| \& \lstinline|3| \& \lstinline|4| \& \lstinline|5| \\
                                \lstinline|0| \& \lstinline|0| \& \lstinline|0| \& \lstinline|0| \& \lstinline|0| \\
                            };
                        }
                        \only<3>{
                            \matrix(cnt) [matrix of nodes, ampersand replacement=\&, row 1/.style={nodes={draw=none}}, nodes={draw, anchor=center}]{
                                \lstinline|1| \& \lstinline|2| \& \lstinline|3| \& \lstinline|4| \& \lstinline|5| \\
                                \lstinline|1| \& \lstinline|0| \& \lstinline|0| \& \lstinline|0| \& \lstinline|0| \\
                            };
                        }
                        \only<4>{
                            \matrix(cnt) [matrix of nodes, ampersand replacement=\&, row 1/.style={nodes={draw=none}}, nodes={draw, anchor=center}]{
                                \lstinline|1| \& \lstinline|2| \& \lstinline|3| \& \lstinline|4| \& \lstinline|5| \\
                                \lstinline|2| \& \lstinline|0| \& \lstinline|0| \& \lstinline|0| \& \lstinline|0| \\
                            };
                        }
                        \only<5>{
                            \matrix(cnt) [matrix of nodes, ampersand replacement=\&, row 1/.style={nodes={draw=none}}, nodes={draw, anchor=center}]{
                                \lstinline|1| \& \lstinline|2| \& \lstinline|3| \& \lstinline|4| \& \lstinline|5| \\
                                \lstinline|2| \& \lstinline|0| \& \lstinline|0| \& \lstinline|1| \& \lstinline|0| \\
                            };
                        }
                        \only<6>{
                            \matrix(cnt) [matrix of nodes, ampersand replacement=\&, row 1/.style={nodes={draw=none}}, nodes={draw, anchor=center}]{
                                \lstinline|1| \& \lstinline|2| \& \lstinline|3| \& \lstinline|4| \& \lstinline|5| \\
                                \lstinline|2| \& \lstinline|0| \& \lstinline|0| \& \lstinline|1| \& \lstinline|1| \\
                            };
                        }
                        \only<7>{
                            \matrix(cnt) [matrix of nodes, ampersand replacement=\&, row 1/.style={nodes={draw=none}}, nodes={draw, anchor=center}]{
                                \lstinline|1| \& \lstinline|2| \& \lstinline|3| \& \lstinline|4| \& \lstinline|5| \\
                                \lstinline|3| \& \lstinline|0| \& \lstinline|0| \& \lstinline|1| \& \lstinline|1| \\
                            };
                        }
                        \only<8->{
                            \matrix(cnt) [matrix of nodes, ampersand replacement=\&, row 1/.style={nodes={draw=none}}, nodes={draw, anchor=center}]{
                                \lstinline|1| \& \lstinline|2| \& \lstinline|3| \& \lstinline|4| \& \lstinline|5| \\
                                \lstinline|3| \& \lstinline|0| \& \lstinline|0| \& \lstinline|2| \& \lstinline|1| \\
                            };
                        }
                    \end{tikzpicture}
                \end{itemize}
            }

        \end{exampleblock}
    }
\end{frame}
%------------------------------------------------------------

%------------------------------------------------------------
\begin{frame}[fragile]
    \frametitle{例 10.3:投票选班长 II}

    \alt<2>{
        \lstinputlisting[basicstyle=\ttfamily\scriptsize,language=C++,name=vote_2]{ch10/vote_2.cc}
    }{
        \begin{exampleblock}{编程题}

            \begin{itemize}
                \item 6 年 A 班在选班长,有 $m$ 位候选人,编号 $1 \sim m$,共 $n$ 名同学将进行投票。\\
                    编写程序,第一行输入整数 $n$, $m$ ($1 \le n \le 100$,$1 \le m \le 10$),如上所述,第二行输入 $n$ 个整数 $x$ ($1 \le x \le 5$),表示每位同学的投票结果。\\
                    按候选人的编号顺序输出 $m$ 行,每行输出候选人编号及票数,用空格间隔,表示得票结果。
            \end{itemize}

            \begin{columns}[onlytextwidth,T]
                \column{.5\textwidth}
                \begin{itemize}
                    \item 样例输入

                        \lstinline|6 6|\\
                        \lstinline|1 1 4 5 1 4|

                \end{itemize}

                \column{.5\textwidth}
                \begin{itemize}
                    \item 样例输出

                        \lstinline|1 3|\\
                        \lstinline|2 0|\\
                        \lstinline|3 0|\\
                        \lstinline|4 2|\\
                        \lstinline|5 1|\\
                        \lstinline|6 0|

                \end{itemize}
            \end{columns}

        \end{exampleblock}
    }
\end{frame}
%------------------------------------------------------------

%------------------------------------------------------------
\begin{frame}[fragile]
    \frametitle{例 10.4:梦中的统计}

    \alt<2>{
        \lstinputlisting[basicstyle=\ttfamily\scriptsize,language=C++,name=count_digit_in_range]{ch10/count_digit_in_range.cc}
    }{
        \begin{exampleblock}{编程题}

            \begin{itemize}
                \item 输入两个整数 $a$ 和 $b$ ($1 \le a \le b \le 10^{6}$),统计 $a$ 到 $b$ 之间每个数中每个数码 ($0 \sim 9$) 的出现次数。

                \item 样例输入

                    \lstinline|129 137|

                \item 样例输出

                    \lstinline|1 10 2 9 1 1 1 1 0 1|

                \item 样例说明

                    在 $129$ 到 $137$ 之间,数码 $0$ 出现了 $1$ 次,数码 $1$ 出现了 $10$ 次,以此类推

            \end{itemize}

        \end{exampleblock}
    }
\end{frame}
%------------------------------------------------------------


\section{一维数组的部分遍历}

%------------------------------------------------------------
\begin{frame}[fragile]
    \frametitle{例 10.5:求前 k 个数组元素的和}

    \alt<2-3>{
        \lstinputlisting[language=C++,name=sum_pre_k]{ch10/sum_pre_k.cc}
        
        \begin{tikzpicture}[remember picture, overlay]
            \uncover<3>{\redbox{sum_pre_k}{14}{13}{14}{28};}
        \end{tikzpicture}
    }{
        \begin{exampleblock}{编程题}

            \begin{itemize}
                \item 编写程序,输入一个整数 $n$ ($1 \le n \le 100$) ,表示有 $n$ 个数组元素,接下来输入 $n$ 个整数。再输入一个整数 $k$ ($1 \le k \le n$),求前 $k$ 个数组元素的和。

                \item 样例输入

                    \lstinline|4|\\
                    \lstinline|3 1 4 5|\\
                    \lstinline|3|

                \item 样例输出

                    \lstinline|8|

            \end{itemize}

        \end{exampleblock}
    }
\end{frame}
%------------------------------------------------------------

%------------------------------------------------------------
\begin{frame}[fragile]
    \frametitle{例 10.6:求部分数组元素的和}

    \alt<2-3>{
        \lstinputlisting[language=C++,name=sum_lr]{ch10/sum_lr.cc}
        
        \begin{tikzpicture}[remember picture, overlay]
            \uncover<3>{\redbox{sum_lr}{14}{13}{14}{28};}
        \end{tikzpicture}
    }{
        \begin{exampleblock}{编程题}

            \begin{itemize}
                \item 编写程序,输入一个整数 $n$ ($1 \le n \le 100$) ,表示有 $n$ 个数组元素,接下来输入 $n$ 个整数。
                再输入两个整数 $l,r$ ($1 \le l \le r \le n$),求数组中第 $l$ 至第 $r$ 个元素之和。

                \item 样例输入

                    \lstinline|7|\\
                    \lstinline|1 0 7 5 4 4 3|\\
                    \lstinline|3 5|

                \item 样例输出

                    \lstinline|16|

            \end{itemize}

        \end{exampleblock}
    }
\end{frame}
%------------------------------------------------------------


\section{数组表示状态}

\tikzset{
    realtree/.pic = {
        \fill[green!20!gray]  (0, -6) ellipse (4 and 1);
        \fill[brown!30!black] (0, -6) ellipse (1 and 0.25);
        \fill[brown!30!black] (-1, -6) rectangle (1, 0);
        \fill[green!40!black] (-6, 0) -- (0, 12) -- (6, 0) -- cycle;
        \fill[green!50!black] (-4, 4) -- (4, 4) -- (2, 8) -- (-2, 8) -- cycle;
        \fill[green!60!black] (-2, 8) -- (2, 8) -- (0, 12) -- cycle;
    }
}
\newcommand{\drawtree}{\pic[scale=.05]{realtree};}

%------------------------------------------------------------
\begin{frame}[fragile]
    \frametitle{例 10.7:校门外的树}

    \alt<10>{
    \begin{block}{}
        \vspace{.5cm}
        \begin{center}
            如何表示以上过程?
        \end{center}
        \vspace{.5cm}
    \end{block}
    }{
            \begin{exampleblock}{编程题}

                \temporal<2>{
                    \begin{itemize}
                        \item 某校大门外长度为 $L$ ($1 \le L \le 10000$) 的马路上有一排树,每两棵相邻的树之间的间隔都是 $1$ 米。\\
                            在马路中,第一棵树在 $0$ 的位置,最后一棵树在 $L$ 的位置。例如 $L = 5$ 时,马路的情况如下图所示。

                            \vspace{.1cm}
                            \begin{tikzpicture}
                                [nodes in empty cells, nodes={minimum width=.9cm, minimum height=.7cm}, row sep=-\pgflinewidth, column sep=-\pgflinewidth]
                                \matrix(idx) [matrix of nodes, ampersand replacement=\&, row 1/.style={nodes={draw=none}}, row 2/.style={nodes={draw=none}}, nodes={draw, anchor=center}]{
                                    \drawtree     \& \drawtree     \& \drawtree     \& \drawtree     \& \drawtree     \& \drawtree     \\
                                    \lstinline|0| \& \lstinline|1| \& \lstinline|2| \& \lstinline|3| \& \lstinline|4| \& \lstinline|5| \\
                                };
                            \end{tikzpicture}

                    \end{itemize}
                }{
                    \begin{itemize}
                        \item 由于马路上有 $m$ 个区域要建地铁,每个区域的起点和终点分别为 $a$, $b$ ($0 \le a \le b \le L$),不同区域之间可能有重合的部分。\\
                            现在要把这些区域中的树移走,对于每个区域,需要移走 $a \sim b$ 之间的树(包含端点 $a$ 和 $b$)。\\
                            你的任务是计算将这些树都移走后,马路上还有多少棵树。
                    \end{itemize}

                    \begin{columns}[onlytextwidth,T]
                        \column{.5\textwidth}
                        \begin{itemize}
                            \item 样例输入

                                \lstinline|10 3|\\
                                \lstinline|2 3|\\
                                \lstinline|5 7|\\
                                \lstinline|6 8|

                        \end{itemize}

                        \column{.5\textwidth}
                        \begin{itemize}
                            \item 样例输出

                                \lstinline|5|

                        \end{itemize}
                    \end{columns}
                }{
                    \begin{columns}[onlytextwidth,T]
                        \column{.5\textwidth}
                        \begin{itemize}
                            \item 样例输入

                                \lstinline|10 3|\\
                                \lstinline|2 3|\\
                                \lstinline|5 7|\\
                                \lstinline|6 8|\\
                                \lstinline||

                        \end{itemize}

                        \column{.5\textwidth}
                        \begin{itemize}
                            \item 样例输出

                                \lstinline|5|

                        \end{itemize}
                    \end{columns}

                    \begin{itemize}
                        \item 样例说明

                            \vspace{.1cm}
                            \begin{tikzpicture}
                                [nodes in empty cells, nodes={minimum width=.9cm, minimum height=.7cm}, row sep=-\pgflinewidth, column sep=-\pgflinewidth]
                                \only<3-4>{
                                    \matrix(idx) [matrix of nodes, ampersand replacement=\&, row 1/.style={nodes={draw=none}}, row 2/.style={nodes={draw=none}}, nodes={draw, anchor=center}]{
                                        \drawtree     \& \drawtree     \& \drawtree     \& \drawtree     \& \drawtree     \& \drawtree     \& \drawtree     \& \drawtree     \& \drawtree     \& \drawtree     \& \drawtree      \\
                                        \lstinline|0| \& \lstinline|1| \& \lstinline|2| \& \lstinline|3| \& \lstinline|4| \& \lstinline|5| \& \lstinline|6| \& \lstinline|7| \& \lstinline|8| \& \lstinline|9| \& \lstinline|10| \\
                                    };
                                }
                                \only<5-6>{
                                    \matrix(idx) [matrix of nodes, ampersand replacement=\&, row 1/.style={nodes={draw=none}}, row 2/.style={nodes={draw=none}}, nodes={draw, anchor=center}]{
                                        \drawtree     \& \drawtree     \&               \&               \& \drawtree     \& \drawtree     \& \drawtree     \& \drawtree     \& \drawtree     \& \drawtree     \& \drawtree      \\
                                        \lstinline|0| \& \lstinline|1| \& \lstinline|2| \& \lstinline|3| \& \lstinline|4| \& \lstinline|5| \& \lstinline|6| \& \lstinline|7| \& \lstinline|8| \& \lstinline|9| \& \lstinline|10| \\
                                    };
                                }
                                \only<7-8>{
                                    \matrix(idx) [matrix of nodes, ampersand replacement=\&, row 1/.style={nodes={draw=none}}, row 2/.style={nodes={draw=none}}, nodes={draw, anchor=center}]{
                                        \drawtree     \& \drawtree     \&               \&               \& \drawtree     \&               \&               \&               \& \drawtree     \& \drawtree     \& \drawtree      \\
                                        \lstinline|0| \& \lstinline|1| \& \lstinline|2| \& \lstinline|3| \& \lstinline|4| \& \lstinline|5| \& \lstinline|6| \& \lstinline|7| \& \lstinline|8| \& \lstinline|9| \& \lstinline|10| \\
                                    };
                                }
                                \only<9->{
                                    \matrix(idx) [matrix of nodes, ampersand replacement=\&, row 1/.style={nodes={draw=none}}, row 2/.style={nodes={draw=none}}, nodes={draw, anchor=center}]{
                                        \drawtree     \& \drawtree     \&               \&               \& \drawtree     \&               \&               \&               \&               \& \drawtree     \& \drawtree      \\
                                        \lstinline|0| \& \lstinline|1| \& \lstinline|2| \& \lstinline|3| \& \lstinline|4| \& \lstinline|5| \& \lstinline|6| \& \lstinline|7| \& \lstinline|8| \& \lstinline|9| \& \lstinline|10| \\
                                    };
                                }
                            \end{tikzpicture}

                    \end{itemize}

                    \begin{tikzpicture}[remember picture, overlay]
                        \uncover<4>{\draw[red, very thick] (0.35, 4.5) rectangle (1, 4.8);}
                        \uncover<4>{\draw[red, very thick] (2.35, 1.25) rectangle (4.11, 2.5);}
                        \uncover<6>{\draw[red, very thick] (0.35, 4.025) rectangle (1, 4.325);}
                        \uncover<6>{\draw[red, very thick] (5.02, 1.25) rectangle (7.65, 2.5);}
                        \uncover<8>{\draw[red, very thick] (0.35, 3.55) rectangle (1, 3.85);}
                        \uncover<8>{\draw[red, very thick] (5.91, 1.25) rectangle (8.54, 2.5);}
                    \end{tikzpicture}
                }

            \end{exampleblock}
    }
\end{frame}
%------------------------------------------------------------

%------------------------------------------------------------
\begin{frame}[fragile]
    \frametitle{例 10.7:校门外的树}

    \alt<8>{
        \lstinputlisting[basicstyle=\ttfamily\scriptsize,language=C++,name=cut_tree]{ch10/cut_tree.cc}
    }{
            \begin{exampleblock}{编程题}

                    \begin{columns}[onlytextwidth,T]
                        \column{.5\textwidth}
                        \begin{itemize}
                            \item 样例输入

                                \lstinline|10 3|\\
                                \lstinline|2 3|\\
                                \lstinline|5 7|\\
                                \lstinline|6 8|\\
                                \lstinline||

                        \end{itemize}

                        \column{.5\textwidth}
                        \begin{itemize}
                            \item 样例输出

                                \lstinline|5|

                        \end{itemize}
                    \end{columns}

                    \begin{itemize}
                        \item 使用数组记录状态

                            \vspace{.1cm}
                            \begin{tikzpicture}
                                [nodes in empty cells, nodes={minimum width=.9cm, minimum height=.7cm}, row sep=-\pgflinewidth, column sep=-\pgflinewidth]
                                \only<1-2>{
                                    \matrix(idx) [matrix of nodes, ampersand replacement=\&, row 1/.style={nodes={draw=none}}, row 2/.style={nodes={draw=none}}, nodes={draw, anchor=center}]{
                                        \drawtree     \& \drawtree     \& \drawtree     \& \drawtree     \& \drawtree     \& \drawtree     \& \drawtree     \& \drawtree     \& \drawtree     \& \drawtree     \& \drawtree      \\
                                        \lstinline|0| \& \lstinline|1| \& \lstinline|2| \& \lstinline|3| \& \lstinline|4| \& \lstinline|5| \& \lstinline|6| \& \lstinline|7| \& \lstinline|8| \& \lstinline|9| \& \lstinline|10| \\
                                        \lstinline|0| \& \lstinline|0| \& \lstinline|0| \& \lstinline|0| \& \lstinline|0| \& \lstinline|0| \& \lstinline|0| \& \lstinline|0| \& \lstinline|0| \& \lstinline|0| \& \lstinline|0| \\
                                    };
                                }
                                \only<3-4>{
                                    \matrix(idx) [matrix of nodes, ampersand replacement=\&, row 1/.style={nodes={draw=none}}, row 2/.style={nodes={draw=none}}, nodes={draw, anchor=center}]{
                                        \drawtree     \& \drawtree     \&               \&               \& \drawtree     \& \drawtree     \& \drawtree     \& \drawtree     \& \drawtree     \& \drawtree     \& \drawtree      \\
                                        \lstinline|0| \& \lstinline|1| \& \lstinline|2| \& \lstinline|3| \& \lstinline|4| \& \lstinline|5| \& \lstinline|6| \& \lstinline|7| \& \lstinline|8| \& \lstinline|9| \& \lstinline|10| \\
                                        \lstinline|0| \& \lstinline|0| \& \lstinline|1| \& \lstinline|1| \& \lstinline|0| \& \lstinline|0| \& \lstinline|0| \& \lstinline|0| \& \lstinline|0| \& \lstinline|0| \& \lstinline|0| \\
                                    };
                                }
                                \only<5-6>{
                                    \matrix(idx) [matrix of nodes, ampersand replacement=\&, row 1/.style={nodes={draw=none}}, row 2/.style={nodes={draw=none}}, nodes={draw, anchor=center}]{
                                        \drawtree     \& \drawtree     \&               \&               \& \drawtree     \&               \&               \&               \& \drawtree     \& \drawtree     \& \drawtree      \\
                                        \lstinline|0| \& \lstinline|1| \& \lstinline|2| \& \lstinline|3| \& \lstinline|4| \& \lstinline|5| \& \lstinline|6| \& \lstinline|7| \& \lstinline|8| \& \lstinline|9| \& \lstinline|10| \\
                                        \lstinline|0| \& \lstinline|0| \& \lstinline|1| \& \lstinline|1| \& \lstinline|0| \& \lstinline|1| \& \lstinline|1| \& \lstinline|1| \& \lstinline|0| \& \lstinline|0| \& \lstinline|0| \\
                                    };
                                }
                                \only<7->{
                                    \matrix(idx) [matrix of nodes, ampersand replacement=\&, row 1/.style={nodes={draw=none}}, row 2/.style={nodes={draw=none}}, nodes={draw, anchor=center}]{
                                        \drawtree     \& \drawtree     \&               \&               \& \drawtree     \&               \&               \&               \&               \& \drawtree     \& \drawtree      \\
                                        \lstinline|0| \& \lstinline|1| \& \lstinline|2| \& \lstinline|3| \& \lstinline|4| \& \lstinline|5| \& \lstinline|6| \& \lstinline|7| \& \lstinline|8| \& \lstinline|9| \& \lstinline|10| \\
                                        \lstinline|0| \& \lstinline|0| \& \lstinline|1| \& \lstinline|1| \& \lstinline|0| \& \lstinline|1| \& \lstinline|1| \& \lstinline|1| \& \lstinline|1| \& \lstinline|0| \& \lstinline|0| \\
                                    };
                                }
                            \end{tikzpicture}

                    \end{itemize}

                    \begin{tikzpicture}[remember picture, overlay]
                        \uncover<2>{\draw[red, very thick] (0.35, 5.2) rectangle (1, 5.5);}
                        \uncover<2>{\draw[red, very thick] (2.35, 0.72) rectangle (4.15, 1.42);}
                        \uncover<4>{\draw[red, very thick] (0.35, 4.725) rectangle (1, 5.025);}
                        \uncover<4>{\draw[red, very thick] (5.05, 0.72) rectangle (7.75, 1.42);}
                        \uncover<6>{\draw[red, very thick] (0.35, 4.25) rectangle (1, 4.55);}
                        \uncover<6>{\draw[red, very thick] (5.95, 0.72) rectangle (8.65, 1.42);}
                    \end{tikzpicture}

            \end{exampleblock}
    }
\end{frame}
%------------------------------------------------------------


\section{数组下标维护}

%------------------------------------------------------------
\begin{frame}[fragile]
    \frametitle{例 10.8:小鱼的数字游戏}

    \alt<2-3>{
        \lstinputlisting[basicstyle=\ttfamily\scriptsize,language=C++,name=vector]{ch10/vector.cc}

        \begin{tikzpicture}[remember picture,overlay]
            \uncover<3>{\redbox{vector}{8}{7}{8}{12};}
            \uncover<3>{\redbox{vector}{13}{8}{13}{29};}
        \end{tikzpicture}
    }{
        \begin{exampleblock}{编程题}

            \begin{itemize}
                \item 输入若干个数(不超过 $100$ 个数),以 $0$ 结尾,逆序输出所有输入的数。

                \item 样例输入

                    \lstinline|3 1 5 7 6 0|

                \item 样例输出

                    \lstinline|6 7 5 1 3|

            \end{itemize}

        \end{exampleblock}
    }
\end{frame}
%------------------------------------------------------------


\section{总结}

%------------------------------------------------------------
\begin{frame}[fragile]
    \frametitle{总结}

    \begin{itemize}
        \item 数组的应用
            \begin{itemize}
                \item 数组计数
                \item 数组表示状态
                \item 数组下标维护
            \end{itemize}
    \end{itemize}
\end{frame}
%------------------------------------------------------------

%------------------------------------------------------------
\begin{frame}
    \begin{center}
        {\Huge Thank you!}
    \end{center}
\end{frame}
%------------------------------------------------------------

\end{document}
