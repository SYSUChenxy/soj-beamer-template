%------------------------------------------------------------
\title[09 - 一维数组]
{09 - 一维数组}

\subtitle{C++ 程序设计基础}

\author[Beiyu Li]
{Beiyu Li\\
\texttt{<sysulby@gmail.com>}}

% \institute[SOJ]
% {Sicily Online Judge}

\date[\today]
{\number\year 年 \number\month 月 \number\day 日}
%------------------------------------------------------------


\begin{document}

\author[sysulby]
{SOJ 信息学竞赛教练组}

\begin{frame}
    \titlepage
\end{frame}
\setcounter{framenumber}{0} % 标题页不编号


\section{复习回顾}

%------------------------------------------------------------
\begin{frame}[fragile]
    \frametitle{多重循环}

    \begin{itemize}[<+->]
        \item 一重循环
        \item 二重循环
            \begin{itemize}
                \item 在一重循环的循环体中再写一个循环
            \end{itemize}
        \item 三重循环
        \item ...
    \end{itemize}
\end{frame}
%------------------------------------------------------------


\section{一维数组}

%------------------------------------------------------------
\begin{frame}[fragile]
    \frametitle{引入}

    \begin{itemize}[<+->]
        \item 使用循环求和的时候,都是一边输入数据一边维护当前的和,并没有将所有的数据都存储起来
        \item 如果需要将求和的数据重新输出一遍,则没有办法实现
    \end{itemize}
\end{frame}
%------------------------------------------------------------

%------------------------------------------------------------
\begin{frame}[fragile]
    \frametitle{思考}

    \begin{block}{}
        \vspace{.5cm}
        \begin{center}
            当变量很多且需要记录他们的值时,该怎么办?
        \end{center}
        \vspace{.5cm}
    \end{block}
\end{frame}
%------------------------------------------------------------

%------------------------------------------------------------
\begin{frame}[fragile]
    \frametitle{数组}

    \begin{overlayarea}{\textwidth}{.55\textheight}
        \begin{itemize}
            \item<1-> 数组的概念

                \begin{itemize}
                    \item 数组是用于存储多个同类型数据的结构
                    \item 多个数据存放在一片连续的内存空间中
                \end{itemize}

            \item<2-> 数组的优势

                \begin{itemize}
                    \item 代码简洁
                    \item 通用、易维护
                \end{itemize}
        \end{itemize}
    \end{overlayarea}

    \begin{columns}
        \column{.01\textwidth}

        \column{.16\textwidth}
        \lstinline|int A[8];|

        \column{.83\textwidth}
        \begin{tikzpicture}
            \matrix (A) [matrix of nodes, row 1/.style={nodes={draw=none}}, nodes={draw, minimum size=1cm}, column sep=-\pgflinewidth]{
                \small{A[0]} & \small{A[1]} & \small{A[2]} & \small{A[3]} & \small{A[4]} & \small{A[5]} & \small{A[6]} & \small{A[7]}\\
                $71$         & $80$         & $62$         & $91$         & $99$         & $82$         & $44$         & $35$        \\
            }; 
        \end{tikzpicture}
    \end{columns}
\end{frame}
%------------------------------------------------------------

%------------------------------------------------------------
\begin{frame}[fragile]
    \frametitle{数组}

    \begin{overlayarea}{\textwidth}{.55\textheight}
        \begin{itemize}
            \item<1-> 数组的声明

                \begin{itemize}
                    \item \textbf{元素类型 \enspace 数组名[数组大小];}
                    \item \lstinline|int A[8]; // 定义了可以存放 8 个整数的数组 A|
                \end{itemize}

                \begin{itemize}
                    \item<2-> 数组中的基本单元叫作“元素”
                    \item<2-> 数组大小定义为可能存放元素数量的最大值(可适当再大一些)
                \end{itemize}

                \begin{itemize}
                    \item<3-> 建议声明为全局变量
                \end{itemize}
        \end{itemize}
    \end{overlayarea}

    \begin{columns}
        \column{.01\textwidth}

        \column{.16\textwidth}
        \lstinline|int A[8];|

        \column{.83\textwidth}
        \begin{tikzpicture}
            \matrix (A) [matrix of nodes, row 1/.style={nodes={draw=none}}, nodes={draw, minimum size=1cm}, column sep=-\pgflinewidth]{
                \small{A[0]} & \small{A[1]} & \small{A[2]} & \small{A[3]} & \small{A[4]} & \small{A[5]} & \small{A[6]} & \small{A[7]}\\
                $71$         & $80$         & $62$         & $91$         & $99$         & $82$         & $43$         & $53$        \\
            }; 
        \end{tikzpicture}
    \end{columns}
\end{frame}
%------------------------------------------------------------

%------------------------------------------------------------
\begin{frame}[fragile]
    \frametitle{数组}

    \begin{overlayarea}{\textwidth}{.55\textheight}
        \begin{itemize}
            \item 数组的初始化

                \begin{itemize}
                    \item<1-> 可以通过一个赋值符号和花括号,将花括号中给定的数值序列,从第 $0$ 位开始,依次赋值给数组中的每一个元素
                        \begin{itemize}
                            \item \lstinline|int A[8] = {71, 80, 62, 91, 99, 82, 43, 53};|
                        \end{itemize}
                \end{itemize}

                \begin{itemize}
                    \item<2-> 数组大小可以指定或留空,会按照需要的最小空间申请内存
                        \begin{itemize}
                            \item \lstinline|double B[] = {1.0, 2.7, 3.14, 9.8};|
                        \end{itemize}
                \end{itemize}
        \end{itemize}
    \end{overlayarea}

    \begin{columns}
        \column{.01\textwidth}

        \column{.16\textwidth}
        \lstinline|int A[8];|

        \column{.83\textwidth}
        \begin{tikzpicture}
            \matrix (A) [matrix of nodes, row 1/.style={nodes={draw=none}}, nodes={draw, minimum size=1cm}, column sep=-\pgflinewidth]{
                \small{A[0]} & \small{A[1]} & \small{A[2]} & \small{A[3]} & \small{A[4]} & \small{A[5]} & \small{A[6]} & \small{A[7]}\\
                $71$         & $80$         & $62$         & $91$         & $99$         & $82$         & $43$         & $53$        \\
            }; 
        \end{tikzpicture}
    \end{columns}
\end{frame}
%------------------------------------------------------------

%------------------------------------------------------------
\begin{frame}[fragile]
    \frametitle{数组}

    \begin{overlayarea}{\textwidth}{.55\textheight}
        \begin{itemize}
            \item 数组元素的访问

                \begin{itemize}
                    \item 通过 \textbf{数组名 [下标]} 访问存储在数组中 某一位置 的值
                        \begin{itemize}
                            \item 下标是指在元素在数组中的位置,从 $0$ 开始
                            \item 下标的范围:$0 \sim size - 1$ ($size$ 为数组大小)
                        \end{itemize}
                \end{itemize}

                \begin{itemize}
                    \item<2-> 例如:声明数组 \lstinline[basicstyle=\ttfamily\scriptsize]|int A[5];|
                        \begin{itemize}
                            \item 可以使用的数组下标:$0$, $1$, $2$, $3$, $4$
                            \item 对应的可以使用的数组元素:\lstinline[basicstyle=\ttfamily\scriptsize]|A[0], A[1], A[2], A[3], A[4]|
                            \item 可以使用变量作为数组下标:\lstinline[basicstyle=\ttfamily\scriptsize]|A[k]| ($0 \le k \le 4$)
                        \end{itemize}
                \end{itemize}
        \end{itemize}
    \end{overlayarea}

    \begin{columns}
        \column{.01\textwidth}

        \column{.16\textwidth}
        \lstinline|int A[8];|

        \column{.83\textwidth}
        \begin{tikzpicture}
            \matrix (A) [matrix of nodes, row 1/.style={nodes={draw=none}}, nodes={draw, minimum size=1cm}, column sep=-\pgflinewidth]{
                \small{A[0]} & \small{A[1]} & \small{A[2]} & \small{A[3]} & \small{A[4]} & \small{A[5]} & \small{A[6]} & \small{A[7]}\\
                $71$         & $80$         & $62$         & $91$         & $99$         & $82$         & $43$         & $53$        \\
            }; 
        \end{tikzpicture}
    \end{columns}
\end{frame}
%------------------------------------------------------------

%------------------------------------------------------------
\begin{frame}[fragile]
    \frametitle{例 9.1:成绩录入}

    \alt<2>{
        \lstinputlisting[language=C++,name=grade_output]{ch09/grade_output.cc}
    }{
        \begin{exampleblock}{编程题}

            \begin{itemize}
                \item 6 年 A 班进行了一次小测,现要求编写程序,录入所有学生成绩,并输出所有学生的成绩,学生成绩如图所示。

                    \begin{tikzpicture}
                        \matrix (A) [matrix of nodes, nodes={draw, minimum size=1cm}, column sep=-\pgflinewidth, ampersand replacement=\&]{
                            $71$ \& $80$ \& $62$ \& $91$ \& $99$ \& $82$ \& $43$ \& $53$ \\
                        }; 
                    \end{tikzpicture}

                \item 样例输入

                    无

                \item 样例输出

                    \lstinline|71 80 62 91 99 82 42 53|

            \end{itemize}
        \end{exampleblock}
    }
\end{frame}
%------------------------------------------------------------

%------------------------------------------------------------
\begin{frame}[fragile]
    \frametitle{例 9.2:成绩查询}

    \alt<2>{
        \lstinputlisting[language=C++,name=grade_query]{ch09/grade_query.cc}
    }{
        \begin{exampleblock}{编程题}

            \begin{itemize}
                \item 已知 6 年 A 班的小测成绩如图所示,编写程序,输入一个整数 $k$ ($0 \le k < 8$),输出 $k$ 号学生的分数。

                    \begin{tikzpicture}
                        \matrix (A) [matrix of nodes, nodes={draw, minimum size=1cm}, column sep=-\pgflinewidth, ampersand replacement=\&]{
                            $71$ \& $80$ \& $62$ \& $91$ \& $99$ \& $82$ \& $43$ \& $53$ \\
                        }; 
                    \end{tikzpicture}

                \item 样例输入

                    \lstinline|5|

                \item 样例输出

                    \lstinline|82|

            \end{itemize}
        \end{exampleblock}
    }
\end{frame}
%------------------------------------------------------------

%------------------------------------------------------------
\begin{frame}[fragile]
    \frametitle{思考}

    \begin{block}{}
        \vspace{.5cm}
        \begin{center}
            当 $k = -1$ 时运行会怎样?
        \end{center}
        \vspace{.5cm}
    \end{block}
\end{frame}
%------------------------------------------------------------

%------------------------------------------------------------
\begin{frame}[fragile]
    \frametitle{数组}

    \begin{overlayarea}{\textwidth}{.55\textheight}
        \begin{itemize}
            \item 数组访问越界

                \begin{itemize}
                    \item 一个大小为 $size$ 的数组,那下标在 $0$ 到 $size-1$ 的之间才是有意义的
                    \item 访问不在数组有意义区间的元素称为 \textbf{数组越界}
                \end{itemize}

                \begin{itemize}
                    \item<2-> 这可能会导致程序出现重大错误,但编译不会报错,需要有意识地避免数组越界
                    \item<2-> 尤其是当数组的下标是以变量的形式出现时,更要加以小心
                \end{itemize}

        \end{itemize}
    \end{overlayarea}

    \begin{columns}
        \column{.01\textwidth}

        \column{.16\textwidth}
        \lstinline|int A[8];|

        \column{.83\textwidth}
        \begin{tikzpicture}
            \matrix (A) [matrix of nodes, row 1/.style={nodes={draw=none}}, nodes={draw, minimum size=1cm}, column sep=-\pgflinewidth]{
                \small{A[0]} & \small{A[1]} & \small{A[2]} & \small{A[3]} & \small{A[4]} & \small{A[5]} & \small{A[6]} & \small{A[7]}\\
                $71$         & $80$         & $62$         & $91$         & $99$         & $82$         & $43$         & $53$        \\
            }; 
        \end{tikzpicture}
    \end{columns}
\end{frame}
%------------------------------------------------------------


\section{总结}

%------------------------------------------------------------
\begin{frame}[fragile]
    \frametitle{总结}

    \begin{itemize}
        \item 数组的概念
        \item 数组的声明
        \item 数组的初始化
        \item 数组元素的访问
        \item 数组的应用
            \begin{itemize}
                \item 数组求和
                \item 求最值及其下标
            \end{itemize}
    \end{itemize}
\end{frame}
%------------------------------------------------------------

%------------------------------------------------------------
\begin{frame}
    \begin{center}
        {\Huge Thank you!}
    \end{center}
\end{frame}
%------------------------------------------------------------

\end{document}
