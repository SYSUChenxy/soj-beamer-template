%------------------------------------------------------------
\title[09 - 一维数组]
{09 - 一维数组}

\subtitle{C++ 程序设计基础}

\author[Beiyu Li]
{Beiyu Li\\
\texttt{<sysulby@gmail.com>}}

% \institute[SOJ]
% {Sicily Online Judge}

\date[\today]
{\number\year 年 \number\month 月 \number\day 日}
%------------------------------------------------------------


\begin{document}

\author[sysulby]
{SOJ 信息学竞赛教练组}

\begin{frame}
    \titlepage
\end{frame}
\setcounter{framenumber}{0} % 标题页不编号


\section{复习回顾}

%------------------------------------------------------------
\begin{frame}[fragile]
    \frametitle{多重循环}

    \begin{itemize}[<+->]
        \item 一重循环
        \item 二重循环
            \begin{itemize}
                \item 在一重循环的循环体中再写一个循环
            \end{itemize}
        \item 三重循环
        \item ...
    \end{itemize}
\end{frame}
%------------------------------------------------------------

%------------------------------------------------------------
\begin{frame}[fragile]
    \frametitle{引入}

    \begin{itemize}[<+->]
        \item 使用循环的时候,都是一边输入数据一边处理,并没有将所有的数据都存储起来
        \item 如果需要将循环处理的数据重新输出一遍,则没有办法实现
    \end{itemize}
\end{frame}
%------------------------------------------------------------

%------------------------------------------------------------
\begin{frame}[fragile]
    \frametitle{思考}

    \begin{block}{}
        \vspace{.5cm}
        \begin{center}
            当变量很多且需要记录他们的值时,该怎么办?
        \end{center}
        \vspace{.5cm}
    \end{block}
\end{frame}
%------------------------------------------------------------


\section{一维数组}

%------------------------------------------------------------
\begin{frame}[fragile]
    \frametitle{数组}

    \begin{overlayarea}{\textwidth}{.55\textheight}
        \begin{itemize}
            \item<1-> 数组的概念

                \begin{itemize}
                    \item 数组是用于存储多个同类型数据的结构
                    \item 多个数据存放在一片连续的内存空间中
                \end{itemize}

            \item<2-> 数组的优势

                \begin{itemize}
                    \item 代码简洁
                    \item 通用、易维护
                \end{itemize}
        \end{itemize}
    \end{overlayarea}

    \begin{columns}
        \column{.01\textwidth}

        \column{.16\textwidth}
        \lstinline|int a[8];|

        \column{.83\textwidth}
        \begin{tikzpicture}
            [nodes in empty cells, nodes={minimum width=1cm, minimum height=.7cm}, row sep=-\pgflinewidth, column sep=-\pgflinewidth]
            \matrix(A) [matrix of nodes, ampersand replacement=\&, row 1/.style={nodes={draw=none}}, nodes={draw, anchor=center}]{
                \lstinline|a[0]| \& \lstinline|a[1]| \& \lstinline|a[2]| \& \lstinline|a[3]| \& \lstinline|a[4]| \& \lstinline|a[5]| \& \lstinline|a[6]| \& \lstinline|a[7]| \\
                \lstinline|71|   \& \lstinline|80|   \& \lstinline|62|   \& \lstinline|91|   \& \lstinline|99|   \& \lstinline|82|   \& \lstinline|43|   \& \lstinline|53|   \\
            };
        \end{tikzpicture}
    \end{columns}
\end{frame}
%------------------------------------------------------------

%------------------------------------------------------------
\begin{frame}[fragile]
    \frametitle{数组}

    \begin{overlayarea}{\textwidth}{.55\textheight}
        \begin{itemize}
            \item<1-> 数组的声明

                \begin{itemize}
                    \item \textbf{元素类型 \enspace 数组名[数组大小];}
                    \item \lstinline|int a[8]; // 定义了可以存放 8 个整数的数组 a|
                \end{itemize}

                \begin{itemize}
                    \item<2-> 数组中的基本单元叫作“元素”
                    \item<2-> 数组大小定义为可能存放元素数量的最大值(可适当再大一些)
                \end{itemize}

                \begin{itemize}
                    \item<3-> 建议声明数组为全局变量
                \end{itemize}
        \end{itemize}
    \end{overlayarea}

    \begin{columns}
        \column{.01\textwidth}

        \column{.16\textwidth}
        \lstinline|int a[8];|

        \column{.83\textwidth}
        \begin{tikzpicture}
            [nodes in empty cells, nodes={minimum width=1cm, minimum height=.7cm}, row sep=-\pgflinewidth, column sep=-\pgflinewidth]
            \matrix(A) [matrix of nodes, ampersand replacement=\&, row 1/.style={nodes={draw=none}}, nodes={draw, anchor=center}]{
                \lstinline|a[0]| \& \lstinline|a[1]| \& \lstinline|a[2]| \& \lstinline|a[3]| \& \lstinline|a[4]| \& \lstinline|a[5]| \& \lstinline|a[6]| \& \lstinline|a[7]| \\
                \lstinline|71|   \& \lstinline|80|   \& \lstinline|62|   \& \lstinline|91|   \& \lstinline|99|   \& \lstinline|82|   \& \lstinline|43|   \& \lstinline|53|   \\
            };
        \end{tikzpicture}
    \end{columns}
\end{frame}
%------------------------------------------------------------

%------------------------------------------------------------
\begin{frame}[fragile]
    \frametitle{数组}

    \begin{overlayarea}{\textwidth}{.55\textheight}
        \begin{itemize}
            \item 数组的初始化

                \begin{itemize}
                    \item<1-> 可以通过一个赋值符号和花括号,将花括号中给定的数值序列,从第 $0$ 位开始,依次赋值给数组中的每一个元素
                        \begin{itemize}
                            \item \lstinline|int a[8] = {71, 80, 62, 91, 99, 82, 43, 53};|
                        \end{itemize}
                \end{itemize}

                \begin{itemize}
                    \item<8-> 数组大小可以指定或留空,会按照需要的最小空间申请内存
                        \begin{itemize}
                            \item \lstinline|double b[] = {1.0, 2.7, 3.14, 9.8};|
                        \end{itemize}
                \end{itemize}
        \end{itemize}
    \end{overlayarea}

    \begin{columns}
        \column{.01\textwidth}

        \column{.16\textwidth}
        \lstinline|int a[8];|

        \column{.83\textwidth}
        \begin{tikzpicture}
            [nodes in empty cells, nodes={minimum width=1cm, minimum height=.7cm}, row sep=-\pgflinewidth, column sep=-\pgflinewidth]
            \only<1>{
                \matrix(A) [matrix of nodes, ampersand replacement=\&, row 1/.style={nodes={draw=none}}, nodes={draw, anchor=center}]{
                    \lstinline|a[0]| \& \lstinline|a[1]| \& \lstinline|a[2]| \& \lstinline|a[3]| \& \lstinline|a[4]| \& \lstinline|a[5]| \& \lstinline|a[6]| \& \lstinline|a[7]| \\
                    \lstinline|  |   \& \lstinline|  |   \& \lstinline|  |   \& \lstinline|  |   \& \lstinline|  |   \& \lstinline|  |   \& \lstinline|  |   \& \lstinline|  |   \\
                };
            }
            \only<2>{
                \matrix(A) [matrix of nodes, ampersand replacement=\&, row 1/.style={nodes={draw=none}}, nodes={draw, anchor=center}]{
                    \lstinline|a[0]| \& \lstinline|a[1]| \& \lstinline|a[2]| \& \lstinline|a[3]| \& \lstinline|a[4]| \& \lstinline|a[5]| \& \lstinline|a[6]| \& \lstinline|a[7]| \\
                    \lstinline|71|   \& \lstinline|  |   \& \lstinline|  |   \& \lstinline|  |   \& \lstinline|  |   \& \lstinline|  |   \& \lstinline|  |   \& \lstinline|  |   \\
                };
            }
            \only<3>{
                \matrix(A) [matrix of nodes, ampersand replacement=\&, row 1/.style={nodes={draw=none}}, nodes={draw, anchor=center}]{
                    \lstinline|a[0]| \& \lstinline|a[1]| \& \lstinline|a[2]| \& \lstinline|a[3]| \& \lstinline|a[4]| \& \lstinline|a[5]| \& \lstinline|a[6]| \& \lstinline|a[7]| \\
                    \lstinline|71|   \& \lstinline|80|   \& \lstinline|  |   \& \lstinline|  |   \& \lstinline|  |   \& \lstinline|  |   \& \lstinline|  |   \& \lstinline|  |   \\
                };
            }
            \only<4>{
                \matrix(A) [matrix of nodes, ampersand replacement=\&, row 1/.style={nodes={draw=none}}, nodes={draw, anchor=center}]{
                    \lstinline|a[0]| \& \lstinline|a[1]| \& \lstinline|a[2]| \& \lstinline|a[3]| \& \lstinline|a[4]| \& \lstinline|a[5]| \& \lstinline|a[6]| \& \lstinline|a[7]| \\
                    \lstinline|71|   \& \lstinline|80|   \& \lstinline|62|   \& \lstinline|  |   \& \lstinline|  |   \& \lstinline|  |   \& \lstinline|  |   \& \lstinline|  |   \\
                };
            }
            \only<5>{
                \matrix(A) [matrix of nodes, ampersand replacement=\&, row 1/.style={nodes={draw=none}}, nodes={draw, anchor=center}]{
                    \lstinline|a[0]| \& \lstinline|a[1]| \& \lstinline|a[2]| \& \lstinline|a[3]| \& \lstinline|a[4]| \& \lstinline|a[5]| \& \lstinline|a[6]| \& \lstinline|a[7]| \\
                    \lstinline|71|   \& \lstinline|80|   \& \lstinline|62|   \& \lstinline|91|   \& \lstinline|...|  \& \lstinline|...|  \& \lstinline|...|  \& \lstinline|...|  \\
                };
            }
            \only<6->{
                \matrix(A) [matrix of nodes, ampersand replacement=\&, row 1/.style={nodes={draw=none}}, nodes={draw, anchor=center}]{
                    \lstinline|a[0]| \& \lstinline|a[1]| \& \lstinline|a[2]| \& \lstinline|a[3]| \& \lstinline|a[4]| \& \lstinline|a[5]| \& \lstinline|a[6]| \& \lstinline|a[7]| \\
                    \lstinline|71|   \& \lstinline|80|   \& \lstinline|62|   \& \lstinline|91|   \& \lstinline|99|   \& \lstinline|82|   \& \lstinline|43|   \& \lstinline|53|   \\
                };
            }
        \end{tikzpicture}
    \end{columns}
\end{frame}
%------------------------------------------------------------

%------------------------------------------------------------
\begin{frame}[fragile]
    \frametitle{数组}

    \begin{overlayarea}{\textwidth}{.55\textheight}
        \begin{itemize}
            \item 数组元素的访问

                \begin{itemize}
                    \item 通过 \textbf{数组名 [下标]} 访问存储在数组中 某一位置 的值
                        \begin{itemize}
                            \item 下标是指在元素在数组中的位置,从 $0$ 开始
                            \item 下标的范围:$0 \sim L - 1$ ($L$ 为数组大小)
                        \end{itemize}
                \end{itemize}

                \begin{itemize}
                    \item<2-> 例如:声明数组 \lstinline|double b[5];|
                        \begin{itemize}
                            \item 可以使用的数组下标:$0$, $1$, $2$, $3$, $4$
                            \item 对应的可以使用的数组元素:\lstinline|b[0], b[1], b[2], b[3], b[4]|
                            \item 可以使用变量作为数组下标:\lstinline|b[i]| ($0 \le i \le 4$)
                        \end{itemize}
                \end{itemize}
        \end{itemize}
    \end{overlayarea}

    \begin{columns}
        \column{.01\textwidth}

        \column{.16\textwidth}
        \lstinline|int a[8];|

        \column{.83\textwidth}
        \begin{tikzpicture}
            [nodes in empty cells, nodes={minimum width=1cm, minimum height=.7cm}, row sep=-\pgflinewidth, column sep=-\pgflinewidth]
            \matrix(A) [matrix of nodes, ampersand replacement=\&, row 1/.style={nodes={draw=none}}, nodes={draw, anchor=center}]{
                \lstinline|a[0]| \& \lstinline|a[1]| \& \lstinline|a[2]| \& \lstinline|a[3]| \& \lstinline|a[4]| \& \lstinline|a[5]| \& \lstinline|a[6]| \& \lstinline|a[7]| \\
                \lstinline|71|   \& \lstinline|80|   \& \lstinline|62|   \& \lstinline|91|   \& \lstinline|99|   \& \lstinline|82|   \& \lstinline|43|   \& \lstinline|53|   \\
            };
        \end{tikzpicture}
    \end{columns}
\end{frame}
%------------------------------------------------------------

%------------------------------------------------------------
\begin{frame}[fragile]
    \frametitle{例 9.1:成绩录入}

    \alt<2>{
        \lstinputlisting[basicstyle=\ttfamily\scriptsize,language=C++,name=grade_output]{ch09/grade_output.cc}
    }{
        \begin{exampleblock}{编程题}

            \begin{itemize}
                \item 6 年 A 班进行了一次小测,现要求编写程序,录入所有学生成绩,并输出所有学生的成绩,学生的成绩如下。

                    \begin{tikzpicture}
                        [nodes in empty cells, nodes={minimum width=1.1cm, minimum height=.7cm}, row sep=-\pgflinewidth, column sep=-\pgflinewidth]
                        \matrix(A) [matrix of nodes, ampersand replacement=\&, row 1/.style={nodes={draw=none}}, nodes={draw, anchor=center}]{
                            \lstinline|a[0]| \& \lstinline|a[1]| \& \lstinline|a[2]| \& \lstinline|a[3]| \& \lstinline|a[4]| \& \lstinline|a[5]| \& \lstinline|a[6]| \& \lstinline|a[7]| \\
                            \lstinline|71|   \& \lstinline|80|   \& \lstinline|62|   \& \lstinline|91|   \& \lstinline|99|   \& \lstinline|82|   \& \lstinline|43|   \& \lstinline|53|   \\
                        };
                    \end{tikzpicture}

                \item 样例输入

                    无

                \item 样例输出

                    \lstinline|71 80 62 91 99 82 43 53|

            \end{itemize}

        \end{exampleblock}
    }
\end{frame}
%------------------------------------------------------------

%------------------------------------------------------------
\begin{frame}[fragile]
    \frametitle{例 9.2:成绩查询}

    \alt<2>{
        \lstinputlisting[basicstyle=\ttfamily\scriptsize,language=C++,name=grade_query]{ch09/grade_query.cc}
    }{
        \begin{exampleblock}{编程题}

            \begin{itemize}
                \item 已知 6 年 A 班的小测成绩如下,编写程序,输入一个整数 $k$ ($0 \le k < 8$),输出 $k$ 号学生的分数。

                    \begin{tikzpicture}
                        [nodes in empty cells, nodes={minimum width=1.1cm, minimum height=.7cm}, row sep=-\pgflinewidth, column sep=-\pgflinewidth]
                        \matrix(A) [matrix of nodes, ampersand replacement=\&, row 1/.style={nodes={draw=none}}, nodes={draw, anchor=center}]{
                            \lstinline|a[0]| \& \lstinline|a[1]| \& \lstinline|a[2]| \& \lstinline|a[3]| \& \lstinline|a[4]| \& \lstinline|a[5]| \& \lstinline|a[6]| \& \lstinline|a[7]| \\
                            \lstinline|71|   \& \lstinline|80|   \& \lstinline|62|   \& \lstinline|91|   \& \lstinline|99|   \& \lstinline|82|   \& \lstinline|43|   \& \lstinline|53|   \\
                        };
                    \end{tikzpicture}

                \item 样例输入

                    \lstinline|5|

                \item 样例输出

                    \lstinline|82|

            \end{itemize}

        \end{exampleblock}
    }
\end{frame}
%------------------------------------------------------------

%------------------------------------------------------------
\begin{frame}[fragile]
    \frametitle{思考}

    \begin{block}{}
        \vspace{.5cm}
        \begin{center}
            当 $k = -1$ 时会有怎样的运行结果?
        \end{center}
        \vspace{.5cm}
    \end{block}
\end{frame}
%------------------------------------------------------------

%------------------------------------------------------------
\begin{frame}[fragile]
    \frametitle{数组}

    \begin{overlayarea}{\textwidth}{.55\textheight}
        \begin{itemize}
            \item 数组访问越界

                \begin{itemize}
                    \item 一个大小为 $L$ 的数组,那下标在 $0$ 到 $L - 1$ 的之间才是有意义的
                    \item 访问不在数组有意义区间的元素称为 \textbf{数组越界}
                \end{itemize}

                \begin{itemize}
                    \item<2-> 这可能会导致程序出现重大错误,但编译不会报错,需要有意识地避免数组越界
                    \item<2-> 尤其是当数组的下标是以变量的形式出现时,更要加以小心
                \end{itemize}

        \end{itemize}
    \end{overlayarea}

    \begin{columns}
        \column{.01\textwidth}

        \column{.16\textwidth}
        \lstinline|int a[8];|

        \column{.83\textwidth}
        \begin{tikzpicture}
            [nodes in empty cells, nodes={minimum width=1cm, minimum height=.7cm}, row sep=-\pgflinewidth, column sep=-\pgflinewidth]
            \matrix(A) [matrix of nodes, ampersand replacement=\&, row 1/.style={nodes={draw=none}}, nodes={draw, anchor=center}]{
                \lstinline|a[0]| \& \lstinline|a[1]| \& \lstinline|a[2]| \& \lstinline|a[3]| \& \lstinline|a[4]| \& \lstinline|a[5]| \& \lstinline|a[6]| \& \lstinline|a[7]| \\
                \lstinline|71|   \& \lstinline|80|   \& \lstinline|62|   \& \lstinline|91|   \& \lstinline|99|   \& \lstinline|82|   \& \lstinline|43|   \& \lstinline|53|   \\
            };
        \end{tikzpicture}
    \end{columns}
\end{frame}
%------------------------------------------------------------


\section{数组元素的使用}

%------------------------------------------------------------
\begin{frame}[fragile]
    \frametitle{数组元素的使用}

    \begin{itemize}
        \item 数组元素的使用与一般变量的用法相同

        \item 例如:输入输出

            \begin{itemize}
                \item \lstinline|cin >> a[0] >> a[1] >> a[2];|
                \item \lstinline|cout << a[0] << " " << a[2] << endl;|
            \end{itemize}

    \end{itemize}
\end{frame}
%------------------------------------------------------------

%------------------------------------------------------------
\begin{frame}[fragile]
    \frametitle{例 9.3:成绩修改 I}

    \alt<2>{
        \lstinputlisting[basicstyle=\ttfamily\scriptsize,language=C++,name=grade_change1]{ch09/grade_change1.cc}
    }{
        \begin{exampleblock}{编程题}

            \begin{itemize}
                \item 已知 6 年 A 班的小测成绩如下,其中,$k$ 号同学发现自己试卷错判,实际分数应该高 $x$ 分。\\
                    编写程序,输入两个整数 $k$ ($0 \le k < 8$) 和 $x$ ($0 \le x \le 100$),请修改并输出 $k$ 号同学的分数。

                    \begin{tikzpicture}
                        [nodes in empty cells, nodes={minimum width=1.1cm, minimum height=.7cm}, row sep=-\pgflinewidth, column sep=-\pgflinewidth]
                        \matrix(A) [matrix of nodes, ampersand replacement=\&, row 1/.style={nodes={draw=none}}, nodes={draw, anchor=center}]{
                            \lstinline|a[0]| \& \lstinline|a[1]| \& \lstinline|a[2]| \& \lstinline|a[3]| \& \lstinline|a[4]| \& \lstinline|a[5]| \& \lstinline|a[6]| \& \lstinline|a[7]| \\
                            \lstinline|71|   \& \lstinline|80|   \& \lstinline|62|   \& \lstinline|91|   \& \lstinline|99|   \& \lstinline|82|   \& \lstinline|43|   \& \lstinline|53|   \\
                        };
                    \end{tikzpicture}

                \item 样例输入

                    \lstinline|2 6|

                \item 样例输出

                    \lstinline|68|

            \end{itemize}

        \end{exampleblock}
    }
\end{frame}
%------------------------------------------------------------

%------------------------------------------------------------
\begin{frame}[fragile]
    \frametitle{例 9.4:成绩修改 II}

    \alt<2>{
        \lstinputlisting[basicstyle=\ttfamily\scriptsize,language=C++,name=grade_change2]{ch09/grade_change2.cc}
    }{
        \begin{exampleblock}{编程题}

            \begin{itemize}
                \item 已知 6 年 A 班的小测成绩如下,其中,$x$ 号与 $y$ 号同学的成绩登记反了。\\
                    编写程序,输入两个整数 $x$ ($0 \le x < 8$) 和 $y$ ($0 \le y \le 8$),请交换他们的成绩,并输出正确的成绩表,以空格间隔。

                    \begin{tikzpicture}
                        [nodes in empty cells, nodes={minimum width=1.1cm, minimum height=.7cm}, row sep=-\pgflinewidth, column sep=-\pgflinewidth]
                        \matrix(A) [matrix of nodes, ampersand replacement=\&, row 1/.style={nodes={draw=none}}, nodes={draw, anchor=center}]{
                            \lstinline|a[0]| \& \lstinline|a[1]| \& \lstinline|a[2]| \& \lstinline|a[3]| \& \lstinline|a[4]| \& \lstinline|a[5]| \& \lstinline|a[6]| \& \lstinline|a[7]| \\
                            \lstinline|71|   \& \lstinline|80|   \& \lstinline|62|   \& \lstinline|91|   \& \lstinline|99|   \& \lstinline|82|   \& \lstinline|43|   \& \lstinline|53|   \\
                        };
                    \end{tikzpicture}

                \item 样例输入

                    \lstinline|3 4|

                \item 样例输出

                    \lstinline|71 80 52 99 91 82 43 53|

            \end{itemize}

        \end{exampleblock}
    }
\end{frame}
%------------------------------------------------------------

%------------------------------------------------------------
\begin{frame}[fragile]
    \frametitle{例 9.5:计算书费 I}

    \alt<2>{
        \lstinputlisting[basicstyle=\ttfamily\scriptsize,language=C++,name=buy_book1]{ch09/buy_book1.cc}
    }{
        \begin{exampleblock}{编程题}
            \begin{itemize}
                \item 编写程序,输入两个整数 $k$ ($0 \le k < 5$) 和 $x$ ($0 \le x \le 100$),表示购买编号为 $k$ 的图书 $x$ 本,输出消费的金额,结果保留两位小数,图书的单价如下。

                    \begin{tikzpicture}
                        [nodes in empty cells, nodes={minimum width=1.1cm, minimum height=.7cm}, row sep=-\pgflinewidth, column sep=-\pgflinewidth]
                        \matrix(B) [matrix of nodes, ampersand replacement=\&, row 1/.style={nodes={draw=none}}, nodes={draw, anchor=center}]{
                            \lstinline|b[0]| \& \lstinline|b[1]| \& \lstinline|b[2]| \& \lstinline|b[3]| \& \lstinline|b[4]| \\
                            \lstinline|28.9| \& \lstinline|78.5| \& \lstinline|35.4| \& \lstinline|43.6| \& \lstinline|56|   \\
                        };
                    \end{tikzpicture}

                \item 样例输入

                    \lstinline|3 5|

                \item 样例输出

                    \lstinline|218.00|

            \end{itemize}

        \end{exampleblock}
    }
\end{frame}
%------------------------------------------------------------

%------------------------------------------------------------
\begin{frame}[fragile]
    \frametitle{例 9.6:计算书费 II}

    \alt<2>{
        \lstinputlisting[basicstyle=\ttfamily\scriptsize,language=C++,name=buy_book2]{ch09/buy_book2.cc}
    }{
        \begin{exampleblock}{编程题}

            \begin{itemize}
                \item 编写程序,输入一个整数 $n$ ($1 \le n \le 100$),表示购买次数。\\
                    然后输入 $n$ 行,每行两个整数 $k$ ($0 \le k < 5$) 和 $x$ ($0 \le x \le 100$),表示购买编号为 $k$ 的图书 $x$ 本,分别输出 n 次购书的消费金额,结果保留两位小数,图书的单价如下。

                    \begin{tikzpicture}
                        [nodes in empty cells, nodes={minimum width=1.1cm, minimum height=.7cm}, row sep=-\pgflinewidth, column sep=-\pgflinewidth]
                        \matrix(B) [matrix of nodes, ampersand replacement=\&, row 1/.style={nodes={draw=none}}, nodes={draw, anchor=center}]{
                            \lstinline|b[0]| \& \lstinline|b[1]| \& \lstinline|b[2]| \& \lstinline|b[3]| \& \lstinline|b[4]| \\
                            \lstinline|28.9| \& \lstinline|78.5| \& \lstinline|35.4| \& \lstinline|43.6| \& \lstinline|56|   \\
                        };
                    \end{tikzpicture}
            \end{itemize}

            \begin{multicols}{2}
                \begin{itemize}
                    \item 样例输入

                        \lstinline|2|\\
                        \lstinline|3 4|\\
                        \lstinline|0 5|

                    \item 样例输出

                        \lstinline|174.40|\\
                        \lstinline|144.50|
                \end{itemize}
            \end{multicols}

        \end{exampleblock}
    }
\end{frame}
%------------------------------------------------------------

%------------------------------------------------------------
\begin{frame}[fragile]
    \frametitle{思考}

    \begin{block}{}
        \vspace{.5cm}
        \begin{center}
            如何求 $n$ 次购书的总额呢?
        \end{center}
        \vspace{.5cm}
    \end{block}
\end{frame}
%------------------------------------------------------------


\section{数组的遍历}

%------------------------------------------------------------
\begin{frame}[fragile]
    \frametitle{数组的遍历}

    \begin{overlayarea}{\textwidth}{.66\textheight}
        \begin{itemize}
            \item 遍历:对数组中所有元素逐个访问一遍的过程称为遍历

            \item 遍历输入数组

                \begin{itemize}
                    \item<1-> 存储在下标 $0 \sim n - 1$ 中
                    \item<1->
                        \lstinline|for (int i = 0; i < n; i++) {|\\
                        \lstinline|  cin >> a[i];|\\
                        \lstinline|}|
                    \item<2-> 存储在下标 $1 \sim n$ 中
                    \item<2->
                        \lstinline|for (int i = 1; i <= n; i++) {|\\
                        \lstinline|  cin >> a[i];|\\
                        \lstinline|}|
                \end{itemize}

        \end{itemize}
    \end{overlayarea}
\end{frame}
%------------------------------------------------------------

%------------------------------------------------------------
\begin{frame}[fragile]
    \frametitle{数组的遍历}

    \begin{overlayarea}{\textwidth}{.66\textheight}
        \begin{itemize}
            \item 遍历:对数组中所有元素逐个访问一遍的过程称为遍历

            \item 遍历输出数组

                \begin{itemize}
                    \item<1-> 存储在下标 $0 \sim n - 1$ 中
                    \item<1-> \lstinline|for (int i = 0; i < n; i++) {|\\
                        \lstinline|  cout << a[i] << " ";|\\
                        \lstinline|}|\\
                        \lstinline|cout << endl;|\\
                    \item<2-> 存储在下标 $1 \sim n$ 中
                    \item<2-> \lstinline|for (int i = 1; i <= n; i++) {|\\
                        \lstinline|  cout << a[i] << " ";|\\
                        \lstinline|}|\\
                        \lstinline|cout << endl;|\\
                \end{itemize}

        \end{itemize}
    \end{overlayarea}
\end{frame}
%------------------------------------------------------------

%------------------------------------------------------------
\begin{frame}[fragile]
    \frametitle{例 9.7:倒序输出数组元素}

    \alt<2>{
        \lstinputlisting[basicstyle=\ttfamily\scriptsize,language=C++,name=array_rev_output]{ch09/array_rev_output.cc}
    }{
        \begin{exampleblock}{编程题}

            \begin{itemize}
                \item 编写程序,输入一个整数 $n$ ($1 \le n \le 100$),表示有 $n$ 个整数,接下来输入 $n$ 个整数存储在数组中,要求倒序输出数组元素。

                \item 样例输入

                    \lstinline|6|\\
                    \lstinline|1 4 2 8 5 7|

                \item 样例输出

                    \lstinline|7 5 8 2 4 1|

            \end{itemize}

        \end{exampleblock}
    }
\end{frame}
%------------------------------------------------------------

%------------------------------------------------------------
\begin{frame}[fragile]
    \frametitle{例 9.8:交替输出数组元素}

    \alt<5>{
        \lstinputlisting[basicstyle=\ttfamily\scriptsize,language=C++,name=array_alt_output]{ch09/array_alt_output.cc}
    }{
        \begin{exampleblock}{编程题}

            \begin{itemize}
                \item 编写程序,输入一个偶数 $n$ ($1 \le n \le 100$),接下来输入 $n$ 个整数,然后交替输出。\\
                    即按照:第 $1$ 个数、倒数第 $1$ 个数、第 $2$ 个数、倒数第 $2$ 个数 $\dots$ 的顺序输出,其中每行输出两个数字,数字与数字之间有空格隔开。
            \end{itemize}

            \begin{columns}[onlytextwidth,T]
                \column{.5\textwidth}
                \begin{itemize}
                    \item 样例输入

                        \lstinputlisting[basicstyle=\ttfamily\small,numbers=none,name=array_alt_output_in]{ch09/array_alt_output.in}

                \end{itemize}

                \column{.5\textwidth}
                \begin{itemize}
                    \item 样例输出

                        \lstinputlisting[basicstyle=\ttfamily\small,numbers=none,name=array_alt_output_out]{ch09/array_alt_output.out}

                        \begin{tikzpicture}[remember picture, overlay, scale=1.25]
                            \uncover<2>{\redbox{array_alt_output_in}{2}{1}{2}{1};}
                            \uncover<2>{\redbox{array_alt_output_in}{2}{11}{2}{11};}
                            \uncover<2>{\redbox{array_alt_output_out}{1}{1}{1}{3};}
                            \uncover<3>{\redbox{array_alt_output_in}{2}{3}{2}{3};}
                            \uncover<3>{\redbox{array_alt_output_in}{2}{9}{2}{9};}
                            \uncover<3>{\redbox{array_alt_output_out}{2}{1}{2}{3};}
                            \uncover<4>{\redbox{array_alt_output_in}{2}{5}{2}{5};}
                            \uncover<4>{\redbox{array_alt_output_in}{2}{7}{2}{7};}
                            \uncover<4>{\redbox{array_alt_output_out}{3}{1}{3}{3};}
                        \end{tikzpicture}

                \end{itemize}
            \end{columns}

        \end{exampleblock}
    }
\end{frame}
%------------------------------------------------------------

%------------------------------------------------------------
\begin{frame}[fragile]
    \frametitle{思考}

    \begin{block}{}
        \vspace{.5cm}
        \begin{center}
            两个数组间可以像变量一样直接赋值吗?
        \end{center}
        \vspace{.5cm}
    \end{block}
\end{frame}
%------------------------------------------------------------

%------------------------------------------------------------
\begin{frame}[fragile]
    \frametitle{数组间的赋值}

    \alt<5>{
        \lstinputlisting[basicstyle=\ttfamily\scriptsize,language=C++,name=array_copy_ac]{ch09/array_copy_ac.cc}
    }{
        \alt<3-4>{
            \begin{itemize}
                \item<3-> 数组之间的赋值不可以像变量那样直接进行赋值

                    \begin{itemize}
                        \item
                            \alt<3>{
                                \lstinline|b = a;|
                            }{
                                \redout{{\lstinline|b = a;|}}
                            }
                    \end{itemize}

                \item<4-> 只能通过遍历依次进行赋值

                    \begin{itemize}
                        \item
                            \lstinline|for (int i = 1; i <= n; i++) \{|\\
                            \lstinline|\ \ b[i] = a[i];|\\
                            \lstinline|\}|
                    \end{itemize}

            \end{itemize}
        }{
            \lstinputlisting[basicstyle=\ttfamily\scriptsize,language=C++,name=array_copy_ce]{ch09/array_copy_ce.cc}

            \uncover<2>{
                \begin{itemize}
                    \item 编译会报错,数组之间不能直接赋值
                \end{itemize}
            }

            \begin{tikzpicture}[remember picture, overlay]
                \uncover<2>{\redbox{array_copy_ce}{13}{3}{13}{8};}
            \end{tikzpicture}
        }
    }
\end{frame}
%------------------------------------------------------------


\section{数组遍历的应用}

%------------------------------------------------------------
\begin{frame}[fragile]
    \frametitle{例 9.9:求数组元素的和}

    \alt<2>{
        \lstinputlisting[basicstyle=\ttfamily\scriptsize,language=C++,name=array_sum]{ch09/array_sum.cc}
    }{
        \begin{exampleblock}{编程题}

            \begin{itemize}
                \item 编写程序,输入一个整数 $n$ ($1 \le n \le 100$),表示有 $n$ 个整数,接下来输入 $n$ 个整数存储在数组中,求这 $n$ 个数组元素之和。

                \item 样例输入

                    \lstinline|6|\\
                    \lstinline|1 4 2 8 5 7|

                \item 样例输出

                    \lstinline|27|

            \end{itemize}

        \end{exampleblock}
    }
\end{frame}
%------------------------------------------------------------

%------------------------------------------------------------
\begin{frame}[fragile]
    \frametitle{例 9.10:求数组元素的最大值}

    \alt<2>{
        \lstinputlisting[basicstyle=\ttfamily\scriptsize,language=C++,name=array_max]{ch09/array_max.cc}
    }{
        \begin{exampleblock}{编程题}

            \begin{itemize}
                \item 编写程序,输入一个整数 $n$ ($1 \le n \le 100$),表示有 $n$ 个整数,接下来输入 $n$ 个整数存储在数组中,求这 $n$ 个数组元素中的最大值。

                \item 样例输入

                    \lstinline|6|\\
                    \lstinline|1 4 2 8 5 7|

                \item 样例输出

                    \lstinline|8|

            \end{itemize}

        \end{exampleblock}
    }
\end{frame}
%------------------------------------------------------------

%------------------------------------------------------------
\begin{frame}[fragile]
    \frametitle{例 9.11:求数组元素的最大值及位置}

    \alt<2-3>{
        \lstinputlisting[basicstyle=\ttfamily\scriptsize,language=C++,name=array_max_with_pos]{ch09/array_max_with_pos.cc}

        \begin{tikzpicture}[remember picture, overlay]
            \uncover<3>{\redbox{array_max_with_pos}{15}{9}{15}{33};}
        \end{tikzpicture}
    }{
        \begin{exampleblock}{编程题}

            \begin{itemize}
                \item 编写程序,输入一个整数 $n$ ($1 \le n \le 100$),表示有 $n$ 个整数,接下来输入 $n$ 个整数存储在数组中,求这 $n$ 个数组元素中的最大值及位置(保证最大值是唯一的),输出以空格间隔。

                \item 样例输入

                    \lstinline|6|\\
                    \lstinline|1 4 2 8 5 7|

                \item 样例输出

                    \lstinline|8 4|

            \end{itemize}

        \end{exampleblock}
    }
\end{frame}
%------------------------------------------------------------


\section{总结}

%------------------------------------------------------------
\begin{frame}[fragile]
    \frametitle{总结}

    \begin{itemize}
        \item 数组的概念
        \item 数组的声明
        \item 数组的初始化
        \item 数组元素的访问
        \item 数组的应用
            \begin{itemize}
                \item 数组求和
                \item 求最值及其下标
            \end{itemize}
    \end{itemize}
\end{frame}
%------------------------------------------------------------

%------------------------------------------------------------
\begin{frame}
    \begin{center}
        {\Huge Thank you!}
    \end{center}
\end{frame}
%------------------------------------------------------------

\end{document}
