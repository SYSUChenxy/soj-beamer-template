%------------------------------------------------------------
\title[12 - 数组的综合应用]
{12 - 数组的综合应用}

\subtitle{C++ 程序设计基础}

\author[Beiyu Li]
{Beiyu Li\\
\texttt{<sysulby@gmail.com>}}

% \institute[SOJ]
% {Sicily Online Judge}

\date[\today]
{\number\year 年 \number\month 月 \number\day 日}
%------------------------------------------------------------


\begin{document}

\author[sysulby]
{SOJ 信息学竞赛教练组}

\begin{frame}
    \titlepage
\end{frame}
\setcounter{framenumber}{0} % 标题页不编号


\section{复习回顾}

%------------------------------------------------------------
\begin{frame}[fragile]
    \frametitle{一维数组}

    \begin{itemize}[<+->]
        \item 一维数组的声明
        
            \begin{itemize}
                \item \textbf{元素类型 \enspace 数组名[数组大小];}
                \item \lstinline|int a[105]; // 定义了可以存放 105 个整数的数组 a|
            \end{itemize}

        \item 一维数组的元素访问
        
            \begin{itemize}
                \item \textbf{数组名称[下标]}
            \end{itemize}

        \item 一维数组的遍历
        
            \begin{itemize}
                \item 使用单重循环遍历一维数组
            \end{itemize}

    \end{itemize}
\end{frame}
%------------------------------------------------------------

%------------------------------------------------------------
\begin{frame}[fragile]
    \frametitle{多维数组}

    \begin{itemize}[<+->]
        \item 多维数组的声明
        
            \begin{itemize}
                \item \textbf{元素类型 \enspace 数组名称[第一维大小][第二维大小]...;}
                \item \lstinline|int a[100][100]; // 定义了可以存放 10000 个整数的二维数组 a|
            \end{itemize}

        \item 多维数组的元素访问
        
            \begin{itemize}
                \item \textbf{数组名称[第一维大小][第二维大小]...}
            \end{itemize}

        \item 多维数组的遍历
        
            \begin{itemize}
                \item 使用多重循环遍历多维数组
            \end{itemize}

    \end{itemize}
\end{frame}
%------------------------------------------------------------


\section{一维数组的应用}

%------------------------------------------------------------
\begin{frame}[fragile]
    \frametitle{例 12.1:统计人数 I}

    \alt<8-9>{
        \lstinputlisting[language=C++,name=cnt_1]{ch12/cnt_1.cc}
        \begin{tikzpicture}[remember picture, overlay]
            \uncover<9>{\redbox{cnt_1}{12}{5}{12}{16};}
        \end{tikzpicture}
    }{
        \alt<2-7> {
            % 样例解释
            \begin{itemize}
                \item 样例说明: 
            
                \lstinputlisting[numbers=none,name=cnt_1_io]{ch12/cnt_1_io.txt}
                    \begin{tikzpicture}[remember picture, overlay,scale=1]
                        \uncover<3>{\redbox{cnt_1_io}{5}{1}{5}{1};}
                        \uncover<3>{\redbox{cnt_1_io}{3}{1}{3}{2};}
                        % \uncover<3>{\redbox{cnt_1_io}{3}{0}{3}{0};}
        
                        \uncover<4>{\redbox{cnt_1_io}{5}{3}{5}{3};}
                        \uncover<4>{\redbox{cnt_1_io}{3}{4}{3}{5};}
                        \uncover<4>{\redbox{cnt_1_io}{3}{1}{3}{2};}
        
                        \uncover<5>{\redbox{cnt_1_io}{5}{5}{5}{5};}
                        \uncover<5>{\redbox{cnt_1_io}{3}{7}{3}{8};}
                        \uncover<5>{\redbox{cnt_1_io}{3}{1}{3}{5};}
        
                        \uncover<6>{\redbox{cnt_1_io}{5}{7}{5}{7};}
                        \uncover<6>{\redbox{cnt_1_io}{3}{10}{3}{12};}
                        \uncover<6>{\redbox{cnt_1_io}{3}{1}{3}{8};}
        
                        \uncover<7>{\redbox{cnt_1_io}{5}{9}{5}{9};}
                        \uncover<7>{\redbox{cnt_1_io}{3}{14}{3}{15};}
                        \uncover<7>{\redbox{cnt_1_io}{3}{1}{3}{12};}
        
                    \end{tikzpicture}

                \begin{itemize}
                    \item<3-> 第 $1$ 个同学的左侧没有其他同学,输出 $0$
                    \item<4-> 第 $2$ 个同学的左侧有 $1$ 个同学 $98$,输出 $1$
                    \item<5-> 第 $3$ 个同学的左侧有 $2$ 个同学 $98,90$,输出 $2$
                    \item<6-> 第 $4$ 个同学的左侧有 $3$ 个同学 $98,90,79$,输出 $0$
                    \item<7-> 第 $5$ 个同学的左侧有 $4$ 个同学 $98,90,79,100$,输出 $3$
                \end{itemize}
                
            \end{itemize}
            
        }{
            \begin{exampleblock}{编程题}

                \begin{itemize}
                    \item 编写程序,输入一个整数 $n$ ($1 \le n \le 100$) ,表示有 $n$ 名学生,
                    从左向右依次输入每个学生的成绩(成绩是 $0$ 到 $100$ 之间的整数)。
                    统计每个学生的左边有几个成绩比自己好的同学。
                    
                    \item 格式要求:输出的每个数据之间以一个空格间隔。
        
                    \item 样例输入
        
                        \lstinline|5|\\
                        \lstinline|98 90 79 100 87|
        
                    \item 样例输出
        
                        \lstinline|0 1 2 0 3|
        
                \end{itemize}
        
            \end{exampleblock}
        }
    }

\end{frame}
%------------------------------------------------------------

%------------------------------------------------------------
\begin{frame}[fragile]
    \frametitle{例 12.2:统计人数 II}

    \alt<8-9>{
        \lstinputlisting[language=C++,name=cnt_2]{ch12/cnt_2.cc}
        \begin{tikzpicture}[remember picture, overlay]
            \uncover<9>{\redbox{cnt_2}{16}{14}{16}{27};}
        \end{tikzpicture}
    }{
        \alt<2-7> {
            % 样例解释
            \begin{itemize}
                \item 样例说明: 
            
                \lstinputlisting[numbers=none,name=cnt_2_io]{ch12/cnt_2_io.txt}

                    \begin{tikzpicture}[remember picture, overlay,scale=1]
                        \uncover<3-7>{\redbox{cnt_2_io}{4}{1}{4}{20};}

                        \uncover<3>{\redbox{cnt_2_io}{6}{1}{6}{1};}
                        \uncover<3>{\redbox{cnt_2_io}{2}{1}{2}{2};}
        
                        \uncover<4>{\redbox{cnt_2_io}{6}{3}{6}{3};}
                        \uncover<4>{\redbox{cnt_2_io}{2}{4}{2}{5};}

                        \uncover<5>{\redbox{cnt_2_io}{6}{5}{6}{5};}
                        \uncover<5>{\redbox{cnt_2_io}{2}{7}{2}{8};}
        
                        \uncover<6>{\redbox{cnt_2_io}{6}{7}{6}{7};}
                        \uncover<6>{\redbox{cnt_2_io}{2}{10}{2}{12};}
        
                        \uncover<7>{\redbox{cnt_2_io}{6}{9}{6}{9};}
                        \uncover<7>{\redbox{cnt_2_io}{2}{14}{2}{15};}
        
                    \end{tikzpicture}

                \begin{itemize}
                    \item<3-> 对于 $A$ 班第 $1$ 个同学,$B$ 班成绩比他好的同学有 $99$,输出 $1$
                    \item<4-> 对于 $A$ 班第 $2$ 个同学,$B$ 班成绩比他好的同学有 $99,92$,输出 $2$
                    \item<5-> 对于 $A$ 班第 $3$ 个同学,$B$ 班成绩比他好的同学有 $89,99,85,92,90$,输出 $5$
                    \item<6-> 对于 $A$ 班第 $4$ 个同学,$B$ 班没有成绩比他好的同学,输出 $0$
                    \item<7-> 对于 $A$ 班第 $5$ 个同学,$B$ 班成绩比他好的同学有 $89,99,92,90$,输出 $4$
                \end{itemize}
                
            \end{itemize}

        }{
            \begin{exampleblock}{编程题}

                \begin{itemize}
                    \item 编写程序,输入一个整数 $n$ ($1 \le n \le 100$) ,表示 $A$ 班有 $n$ 名学
                    生,接下来输入 $A$ 班 $n$ 名学生的成绩。
                    输入一个整数 $m$ ($1 \le m \le 100$) ,表示 $B$ 班有 $m$ 名学生,接下
                    来输入 $B$ 班 $m$ 名学生的成绩。(成绩均为 $0~100$ 之间的整数)
                    对 $A$ 班的每个学生, 统计在 $B$ 班的 $m$ 个人里有几个同学成绩比
                    他好,输出以空格间隔。
                    
                    \item 格式要求:输出的每个数据之间以一个空格间隔。
                \end{itemize}

                \begin{columns}[onlytextwidth,T]
    
                    \column{.5\textwidth}

                        \begin{itemize}
                            \item 样例输入
    
                            \lstinline|5|\\
                            \lstinline|98 90 79 100 87|\\
                            \lstinline|7|\\
                            \lstinline|70 89 99 85 92 60 90|

                        \end{itemize}

                    \column{.5\textwidth}

                        \begin{itemize}
                            \item 样例输出
                            
                            \lstinline|1 2 5 0 4|

                        \end{itemize}
                        
                \end{columns}
        
            \end{exampleblock}
        }
    }

\end{frame}
%------------------------------------------------------------


\section{二维数组的应用}

%------------------------------------------------------------
\begin{frame}[fragile]
    \frametitle{例 12.3:计算两个矩阵的和}

    \alt<2-10>{
        \alt<2-8>{
            \begin{itemize}
                \item 两个 $n$ 行 $m$ 列的二维数组的和 也是一个 $n$ 行 $m$ 列的二维数组
                \item 结果中第 i 行第 j 列的元素 等于 原来两个数组中第 i 行第 j 列的元素之和。
                
                \lstinputlisting[basicstyle=\ttfamily\large,numbers=none,name=matrix_add_input]{ch12/matrix_add_input.txt}

                \begin{tikzpicture}[remember picture, overlay, scale=1.5]
                    \uncover<3>{
                        \redbox{matrix_add_input}{2}{1}{2}{1}; \redbox{matrix_add_input}{2}{14}{2}{14}; \redbox{matrix_add_input}{2}{27}{2}{27};}
                    \uncover<4>{
                        \redbox{matrix_add_input}{2}{3}{2}{3}; \redbox{matrix_add_input}{2}{16}{2}{16}; \redbox{matrix_add_input}{2}{29}{2}{29};}
                    \uncover<5>{
                        \redbox{matrix_add_input}{2}{5}{2}{5}; \redbox{matrix_add_input}{2}{18}{2}{18}; \redbox{matrix_add_input}{2}{31}{2}{31};}
                    \uncover<6>{
                        \redbox{matrix_add_input}{3}{1}{3}{1}; \redbox{matrix_add_input}{3}{14}{3}{14}; \redbox{matrix_add_input}{3}{27}{3}{27};}
                    \uncover<7>{
                        \redbox{matrix_add_input}{3}{3}{3}{3}; \redbox{matrix_add_input}{3}{16}{3}{16}; \redbox{matrix_add_input}{3}{29}{3}{29};}
                    \uncover<8>{
                        \redbox{matrix_add_input}{3}{5}{3}{5}; \redbox{matrix_add_input}{3}{18}{3}{18}; \redbox{matrix_add_input}{3}{31}{3}{31};}
                \end{tikzpicture}

            \end{itemize}
        }{
            \alt<9>{
                \lstinputlisting[language=C++,name=matrix_add,firstline=1,lastline=21]{ch12/matrix_add.cc}
            }{
                \lstinputlisting[language=C++,name=matrix_add,firstline=22,lastline=37,firstnumber=22]{ch12/matrix_add.cc}
            }
        }
    }{
        \begin{exampleblock}{编程题}

            \begin{itemize}
                \item 编写程序,输入两个整数 $n$ 和 $m$ ($1 \le n, m \le 100$),接下来输入两个 $n$ 行 $m$ 列、整数类型的二维数组 $a$ 和 $b$,求这两个二维数组的和。

                \item 样例输入

                    \lstinline|2 3|\\
                    \lstinline|6 3 1|\\
                    \lstinline|3 2 2|\\
                    \lstinline|1 6 4|\\
                    \lstinline|3 2 3|

                \item 样例输出

                    \lstinline|7 9 5|\\
                    \lstinline|6 4 5|

            \end{itemize}

        \end{exampleblock}
    }
\end{frame}
%------------------------------------------------------------

%------------------------------------------------------------
\begin{frame}[fragile]
    \frametitle{例 12.4:二维数组的旋转}

    \alt<6>{
        \lstinputlisting[language=C++,name=rotate]{ch12/rotate.cc}
    }{
        \alt<2-5> {
            % 样例解释
            \begin{columns}[T]
                \column{.5\textwidth}
                    \begin{itemize}
                        \item 当前数组: 
                    
                        \lstinputlisting[numbers=none,name=rotate_io]{ch12/rotate_io.txt}
        
                            \begin{tikzpicture}[remember picture, overlay,scale=1]
                                \uncover<3>{\redbox{rotate_io}{1}{1}{3}{6} node[below,xshift=-.6cm]{\lstinline|第1列|};}
                                \uncover<4>{\redbox{rotate_io}{1}{8}{3}{13} node[below,xshift=-.6cm]{\lstinline|第2列|};}
                                \uncover<5>{\redbox{rotate_io}{1}{15}{3}{20} node[below,xshift=-.6cm]{\lstinline|第3列|};}
                            \end{tikzpicture}
        
                        % \begin{itemize}
                        %     \item
                        % \end{itemize}
                        
                    \end{itemize}

                \column{.5\textwidth}
                    \begin{itemize}
                        \item 当前输出:\\
                            
                            \lstinline| |\\
                            \uncover<3->{\lstinline|8(2,1) 1(1,1)|}\\
                            \uncover<4->{\lstinline|4(2,2) 3(1,2)|}\\
                            \uncover<5->{\lstinline|6(2,3) 7(1,3)|}

                    \end{itemize}

            \end{columns}

            \begin{itemize}
                \item 输出旋转后的数组时,行坐标 i 和 列坐标 j 如何变化?
            \end{itemize}

        }{
            \begin{exampleblock}{编程题}

                \begin{itemize}
                    \item 首先输入两个整数 $n,m$ ($1 \le n,m \le 100$),表示接下来需要输
                    入 $n$ 行 $m$ 列的正整数(正整数均小于 $1000$),这些整数组成了
                    一个 $n$ 行 $m$ 列的矩阵,请将矩阵顺时针翻转 $90$ 度后输出。
                    
                    \item 样例输入
                    
                    \lstinline|2 3|\\
                    \lstinline|1 3 7|\\
                    \lstinline|8 4 6|

                    \item 样例输出
                    
                    \lstinline|8 1|\\
                    \lstinline|4 3|\\
                    \lstinline|6 7|

                \end{itemize}

            \end{exampleblock}
        }
    }

\end{frame}
%------------------------------------------------------------

%------------------------------------------------------------
\begin{frame}[fragile]
    \frametitle{例 12.5:计算小矩阵的和}

    \alt<7>{
        \lstinputlisting[language=C++,name=small_matrix]{ch12/small_matrix.cc}
    }{
        \alt<4-6> {
            % 样例解释
            \begin{columns}[T]
                \column{.5\textwidth}
                    \begin{itemize}
                        \item 当前输入: 
                    
                        \lstinputlisting[numbers=none,name=small_matrix_i]{ch12/small_matrix_i.txt}

                                \begin{tikzpicture}[remember picture,overlay]
                                    \uncover<4->{\redbox{small_matrix_i}{2}{5}{4}{7} node[right,xshift=.3cm,yshift=.8cm]{\lstinline|第\ x 行,第 y 列|};}
                                    \uncover<4->{\redbox{small_matrix_i}{6}{1}{6}{3} node[below,xshift=.1cm]{\lstinline|起点:(x,y)|};}
                                    \uncover<5->{\redbox{small_matrix_i}{2}{5}{4}{7} node[right,xshift=.3cm,yshift=.2cm]{\lstinline|第\ ? 行,第 ? 列|};}
                                    \uncover<5->{\redbox{small_matrix_i}{6}{5}{6}{7} node[right,xshift=.3cm,yshift=.2cm]{\lstinline|小矩阵:r 行 c 列|};}
                                \end{tikzpicture}
                        
                    \end{itemize}

                \column{.5\textwidth}

            \end{columns}

            \begin{itemize}
                \item<6-> 遍历小矩阵时,小矩阵的终点行、终点列应当如何计算?
            \end{itemize}

        }{
            \begin{exampleblock}{编程题}

                \begin{itemize}
                    \item 首先输入两个整数 $n,m$ ($1 \le n,m \le 100$),表示接下来需要输
                    入 $n$ 行 $m$ 列的正整数(正整数均小于 $1000$),最后再输入四个
                    整数 $x,y,r,c$ ($1 \le x,x+r-1 \le n$),($1 \le y,y+c-1 \le m$),请
                    输出原矩阵中以 ($x,y$) 为左上角的 $r$ 行 $c$ 列的小矩阵元素的总和。
                \end{itemize}    
                
                \begin{columns}[onlytextwidth,T]

                    \column{.5\textwidth}
                        \begin{itemize}
                            \item 样例输入

                                \lstinputlisting[numbers=none,name=small_matrix_i]{ch12/small_matrix_i.txt}

                                \begin{tikzpicture}[remember picture,overlay]
                                    \uncover<2>{\redbox{small_matrix_i}{2}{5}{4}{7} node[right,xshift=.3cm,yshift=.8cm]{\lstinline|第一行,第三列|};}
                                    \uncover<2>{\redbox{small_matrix_i}{6}{1}{6}{3} node[below,xshift=.1cm]{\lstinline|起点:(1,3)|};}
                                    \uncover<3>{\redbox{small_matrix_i}{2}{5}{4}{7} node[right,xshift=.3cm,yshift=.2cm]{\lstinline|第三行,第五列|};}
                                    \uncover<3>{\redbox{small_matrix_i}{6}{5}{6}{7} node[below,xshift=.1cm]{\lstinline|小矩阵:3 行 2 列|};}
                                \end{tikzpicture}
                        \end{itemize}

                    \column{.5\textwidth}
                        \begin{itemize}
                            
                        \item 样例输出
                        
                            \lstinputlisting[numbers=none,name=small_matrix_o]{ch12/small_matrix_o.txt}

                        \end{itemize}
                \end{columns}

            \end{exampleblock}
        }
    }

\end{frame}
%------------------------------------------------------------

%------------------------------------------------------------
\begin{frame}[fragile]
    \frametitle{例 12.6:出现最频繁的点}

    \alt<2>{
        \lstinputlisting[language=C++,name=points_count]{ch12/points_count.cc}
    }{
        \begin{exampleblock}{编程题}

            \begin{itemize}
                \item 给定 $n$ ($1 \le n \le 100$) 个二维空间的点,每个点的坐标 ($x,y$) 都是整数 ($0 \le x,y \le 99$),求其中出现最频繁的点的出现次数。                

                \item 样例输入

                    \lstinline|4|\\
                    \lstinline|1 2|\\
                    \lstinline|2 3|\\
                    \lstinline|2 3|\\
                    \lstinline|5 6|

                \item 样例输出

                    \lstinline|2|

            \end{itemize}

        \end{exampleblock}
    }
\end{frame}
%------------------------------------------------------------


\section{总结}

%------------------------------------------------------------
\begin{frame}[fragile]
    \frametitle{多维数组}

    \begin{itemize}[<+->]
        \item 一维数组的应用
        
            \begin{itemize}
                \item 一维数组部分求和
                \item 一维数组遍历比较
            \end{itemize}

        \item 二维数组的应用
        
            \begin{itemize}
                \item 二维数组的旋转
                \item 计算小矩阵的和
                \item 二维数组的计数应用
            \end{itemize}

    \end{itemize}
\end{frame}
%------------------------------------------------------------

%------------------------------------------------------------
\begin{frame}
    \begin{center}
        {\Huge Thank you!}
    \end{center}
\end{frame}
%------------------------------------------------------------

\end{document}

