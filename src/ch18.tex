%------------------------------------------------------------
\title[06 - string 类型]
{06 - string 类型}

\subtitle{C++ 程序设计进阶}

\author[Beiyu Li]
{Beiyu Li\\
\texttt{<sysulby@gmail.com>}}

% \institute[SOJ]
% {Sicily Online Judge}

\date[\today]
{\number\year 年 \number\month 月 \number\day 日}
%------------------------------------------------------------


\begin{document}

\author[sysulby]
{SOJ 信息学竞赛教练组}

\begin{frame}
    \titlepage
\end{frame}
\setcounter{framenumber}{0} % 标题页不编号


\section{复习回顾}

%------------------------------------------------------------

\begin{frame}[fragile]
    \frametitle{复习回顾}

    \only<1>{
        \begin{itemize}
            \item C 风格字符串
            
            \lstinputlisting[basicstyle=\ttfamily\footnotesize,language=C++,name=review]{ch18/review.cc}
    
        \end{itemize}
    }

    \only<2> {
        \begin{itemize}
            \item \lstinline|strlen(str)| : 获取长度
            \item \lstinline|strcpy(str1, str2)| : 复制
            \item \lstinline|strcat(str1, str2)| : 拼接
            \item \lstinline|strcmp(str1, str2)| : 比较字典序
        \end{itemize}
    }
    
\end{frame}
%------------------------------------------------------------

%------------------------------------------------------------
\begin{frame}[fragile]
    \frametitle{C 风格字符串的优缺点}

    \begin{itemize}
        \item 优点
        
        \begin{itemize}
            \item 本质是字符数组,简单,接近系统底层,效率高
        \end{itemize}

        \item 缺点
        
        \begin{itemize}
            \item 容易出错(数组越界),操作不便(提取子串、替换子串等需要自己实现)
        \end{itemize}

    \end{itemize}

\end{frame}
%------------------------------------------------------------


\section{string 类型的基本用法}

%------------------------------------------------------------
\begin{frame}[fragile]
    \frametitle{string 类型}

    \only<1>{
        \begin{itemize}
            \item 为了克服 C 风格字符串的缺点,C++ 引入 string 类型
            \item 用 string 类型处理字符串的主要优点有:
            
            \begin{itemize}
                \item string 类型在 C 风格字符串的基础上进行了包装和加工,自带了丰富的函数和运算符可以直接使用,可以很方便地操作字符串
                \item 存储字符串的内存空间是系统自动分配的,字符串的大小会根据实际存储内容进行自动调整,不需要一开始声明大小
            \end{itemize}
    
        \end{itemize}
    }
    
    \only<2>{
        \begin{itemize}
            \item 添加头文件 \textcolor{red}{\lstinline|<string>|}
            \item 声明与初始化
            
            \lstinputlisting[basicstyle=\ttfamily\footnotesize,language=C++,name=init_string]{ch18/init_string.cc}

            \item 整体输出
            
            \lstinputlisting[basicstyle=\ttfamily\footnotesize,language=C++,name=cout_string]{ch18/cout_string.cc}

        \end{itemize}
    }

    \only<3>{
        \begin{itemize}
            \item \lstinline|cin| 输入
            
            \begin{itemize}
                \item 当遇到空格、换行等空字符时会结束输入
                \lstinputlisting[basicstyle=\ttfamily\footnotesize,language=C++,name=cin_string]{ch18/cin_string.cc}
            \end{itemize}

            \item \lstinline|getline| 整行输入
            
            \begin{itemize}
                \item 当遇到换行时会结束输入,会把空格读入,但不会读入换行符
                \lstinputlisting[basicstyle=\ttfamily\footnotesize,language=C++,name=getline_string]{ch18/getline_string.cc}
            \end{itemize}

        \end{itemize}
    }

    \only<4-5>{
        \begin{itemize}
            \item 访问单个字符\\
            
            \lstinputlisting[basicstyle=\ttfamily\footnotesize,language=C++,name=single_ch]{ch18/single_ch.cc}
            
            \item 遍历字符串\\
            
            \lstinputlisting[basicstyle=\ttfamily\footnotesize,language=C++,name=tra_string]{ch18/tra_string.cc}

            \begin{tikzpicture}[remember picture, overlay]
                \uncover<5>{\redbox{tra_string}{4}{16}{4}{16};}
            \end{tikzpicture}

        \end{itemize}
    }

    \only<6>{
        \begin{itemize}
            \item \lstinline|s.length()/s.size()|
            
            \begin{itemize}
                \item 返回一个整数,表示字符串的长度
                
                \lstinputlisting[basicstyle=\ttfamily\footnotesize,language=C++,name=size_string]{ch18/size_string.cc}
                % \lstinputlisting[basicstyle=\ttfamily\footnotesize,language=C++,name=length_string]{ch18/length_string.cc}
                
            \end{itemize}

        \end{itemize}
    }

\end{frame}
%------------------------------------------------------------

%------------------------------------------------------------
\begin{frame}[fragile]
    \frametitle{字符串转小写}

    \lstinputlisting[basicstyle=\ttfamily\footnotesize,language=C++,name=lower.cc]{ch18/lower.cc}
            
\end{frame}
%------------------------------------------------------------


\section{string 类型的运算符}

%------------------------------------------------------------
\begin{frame}[fragile]
    \frametitle{string 类型的运算符}

    \only<1> {
        \begin{block}{}
            \vspace{.5cm}
            \begin{center}
                C 风格字符串不可以用 = 直接进行赋值,\\ 那 string 类型可以吗?
            \end{center}
            \vspace{.5cm}
        \end{block}
    }

    \only<2> {
        \begin{itemize}
            \item 赋值
            
            \lstinputlisting[basicstyle=\ttfamily\footnotesize,language=C++,name=assign_string]{ch18/assign_string.cc}

            \item 字典序比较
            
            \lstinline|>、<、>=、<=、==、!=| \\
            
            \lstinputlisting[basicstyle=\ttfamily\footnotesize,language=C++,name=compare_string]{ch18/compare_string.cc}
        
        \end{itemize}
    }

    \only<3> {
        \begin{itemize}
            \item 拼接
            
            \lstinputlisting[basicstyle=\ttfamily\footnotesize,language=C++,name=splicing_string]{ch18/splicing_string.cc}
        
        \end{itemize}
    }
            
\end{frame}
%------------------------------------------------------------

%------------------------------------------------------------
\begin{frame}[fragile]
    \frametitle{string 类型的常用函数}

    \begin{itemize}
        \item \lstinline|s.substr(pos, len)|
        
        \begin{itemize}
            \item 返回一个字符串,内容为字符串 $s$ 中从下标 $pos$ 开始,长度为 $len$ 的子串
            
            \lstinputlisting[basicstyle=\ttfamily\footnotesize,language=C++,name=substr_string]{ch18/substr_string.cc}

            \begin{tikzpicture}[remember picture, overlay]
                \uncover<2>{\redbox{substr_string}{4}{20}{4}{21} node[below,xshift=-.2cm]{注意 pos 不能越界,越界会导致 RE};}
                \uncover<3>{\redbox{substr_string}{4}{25}{4}{26} node[below,xshift=-.2cm]{如果 len 越界了,substr 会返回从 pos 开始直到字符串末尾的子字符串};}
            \end{tikzpicture}

        \end{itemize}
    \end{itemize}
            
\end{frame}
%------------------------------------------------------------

%------------------------------------------------------------
\begin{frame}[fragile]
    \frametitle{例 6.1:枚举子串}

    \alt<2-3> {
        \lstinputlisting[basicstyle=\ttfamily\scriptsize,language=C++,name=substring]{ch18/substring.cc}

        \begin{tikzpicture}[remember picture, overlay]
            \uncover<3>{\redbox{substring}{8}{19}{8}{27};}
        \end{tikzpicture}
    } {
        \begin{exampleblock}{编程题}
            \begin{itemize}
                \item 输入一个字符串 $s$(只包含小写字母),输出字符串 $s$ 中所有长度为 $3$ 的子串
                    
                \item 样例输入
    
                    \lstinline|hello|
    
                \item 样例输出
                
                    \lstinline|hel|\\
                    \lstinline|ell|\\
                    \lstinline|llo|
    
            \end{itemize}
        \end{exampleblock}
    }

\end{frame}
%------------------------------------------------------------

%------------------------------------------------------------
\begin{frame}[fragile]
    \frametitle{string 类型的常用函数}

    \begin{itemize}
        \item \lstinline|s.find(s1, pos)|
        
        \begin{itemize}
            \item 返回一个整数,表示在字符串 $s$ 中,从 $pos$ 这个下标开始,第一次查找到的字符串 $s1$ 的位置
            \item 找不到时函数返回 \lstinline|string::npos|
            \item 省略 $pos$ 时,\lstinline|find| 默认从下标 $0$ 开始查找
            
            \lstinputlisting[basicstyle=\ttfamily\footnotesize,language=C++,name=find_string]{ch18/find_string.cc}

        \end{itemize}
    \end{itemize}
            
\end{frame}
%------------------------------------------------------------

%------------------------------------------------------------
\begin{frame}[fragile]
    \frametitle{C 风格字符串与 string 类型对比}

    \begin{table}[!ht]
        \centering
        \renewcommand{\arraystretch}{1.5} % 设置垂直内边距为1.5倍默认行高
        \begin{tabular}{p{2.2cm}p{3.5cm}p{4cm}}
        \hline
            \textbf{操作}   & \textbf{C 风格字符串}   & \textbf{string 类型} \\ \hline
            输入、输出       & \lstinline|cin / cout|            & \lstinline|cin / cout| \\ 
            整行输入         & \lstinline|fgets(s, len, stdin)|   & \lstinline|getline(cin, s)| \\ 
            第 $i$ 个字符    & \lstinline|s[i]|                  & \lstinline|s[i]| \\ 
            求长度          & \lstinline|strlen(s)|            & \lstinline|s.size() / s.length()| \\
            字符串比较       & \lstinline|strcmp(s1, s2)|        & \lstinline|s1 > s2, s1 < s2| ... \\
            字符串复制       & \lstinline|strcpy(s1, s2)|        & \lstinline|s1 = s2| \\ 
            字符串拼接       & \lstinline|strcat(s1, s2)|        & \lstinline|s1 = s1 + s2| \\\hline
        \end{tabular} 
    \end{table}
            
\end{frame}
%------------------------------------------------------------


\section{string 数组}

%------------------------------------------------------------
\begin{frame}[fragile]
    \frametitle{string 数组}

    \only<1> {
        \begin{itemize}
            \item 声明
            
            \lstinputlisting[basicstyle=\ttfamily\footnotesize,language=C++,name=init_arr_string]{ch18/init_arr_string.cc}
    
            \item 访问单个字符串
            
            \lstinputlisting[basicstyle=\ttfamily\footnotesize,language=C++,name=single_si]{ch18/single_si.cc}
    
            \item 输入
            
            \lstinputlisting[basicstyle=\ttfamily\footnotesize,language=C++,name=cin_arr_string]{ch18/cin_arr_string.cc}
    
        \end{itemize}
    }

    \only<2>{
        \begin{itemize}
            \item 输出
            
            \lstinputlisting[basicstyle=\ttfamily\footnotesize,language=C++,name=init_arr_string]{ch18/cout_arr_string.cc}
    
            \begin{tikzpicture}
                [nodes in empty cells, nodes={minimum width=1.2cm, minimum height=.7cm}, row sep=-\pgflinewidth, column sep=-\pgflinewidth]
                \matrix(A) [matrix of nodes, ampersand replacement=\&, row 1/.style={nodes={draw=none}}, column 1/.style={nodes={draw=none}}, nodes={draw, anchor=center}]{
                    \\
                    \lstinline|s[1]| \& \lstinline|h| \& \lstinline|e| \& \lstinline|l| \& \lstinline|l| \& \lstinline|o| \\
                    \lstinline|s[2]| \& \lstinline|h| \& \lstinline|i| \\
                    \lstinline|s[3]| \& \lstinline|t| \& \lstinline|u| \& \lstinline|r| \& \lstinline|i| \& \lstinline|n| \& \lstinline|g| \\
                    \lstinline|...| \& \lstinline|...| \\
                };
            \end{tikzpicture}

        \end{itemize}
    }
            
\end{frame}
%------------------------------------------------------------

%------------------------------------------------------------
\begin{frame}[fragile]
    \frametitle{string 数组}

    \only<1> {
        \begin{block}{}
            \vspace{.5cm}
            \begin{center}
                string 数组和什么数组很像?
            \end{center}
            \vspace{.5cm}
        \end{block}
    }

    \only<2> {
        \begin{itemize}
            \item 声明
            
            \lstinputlisting[basicstyle=\ttfamily\footnotesize,language=C++,name=init_arr_ch]{ch18/init_arr_ch.cc}
    
            \item 访问单个字符串
            
            \lstinputlisting[basicstyle=\ttfamily\footnotesize,language=C++,name=single_si_2]{ch18/single_si_2.cc}
    
            \item 输入
            
            \lstinputlisting[basicstyle=\ttfamily\footnotesize,language=C++,name=cin_arr_string]{ch18/cin_arr_string.cc}
    
        \end{itemize}
    }

    \only<3>{
        \begin{itemize}
            \item 输出
            
            \lstinputlisting[basicstyle=\ttfamily\footnotesize,language=C++,name=init_arr_string]{ch18/cout_arr_string.cc}
    
            \begin{tikzpicture}
                [nodes in empty cells, nodes={minimum width=1.2cm, minimum height=.7cm}, row sep=-\pgflinewidth, column sep=-\pgflinewidth]
                \matrix(A) [matrix of nodes, ampersand replacement=\&, row 1/.style={nodes={draw=none}}, column 1/.style={nodes={draw=none}}, nodes={draw, anchor=center}]{
                    \\
                    \lstinline|s[1]| \& \lstinline|h| \& \lstinline|e| \& \lstinline|l| \& \lstinline|l| \& \lstinline|o| \& \lstinline|'\\0'| \& \\
                    \lstinline|s[2]| \& \lstinline|h| \& \lstinline|i| \& \lstinline|'\\0'| \& \& \& \& \\
                    \lstinline|s[3]| \& \lstinline|t| \& \lstinline|u| \& \lstinline|r| \& \lstinline|i| \& \lstinline|n| \& \lstinline|g| \& \lstinline|'\\0'| \\
                    \lstinline|...|  \& \lstinline|...| \& \& \& \& \& \& \\
                };
            \end{tikzpicture}

        \end{itemize}
    }
            
\end{frame}
%------------------------------------------------------------

%------------------------------------------------------------
\begin{frame}[fragile]
    \frametitle{例 6.2:字符串频率}

    \alt<2> {
        \lstinputlisting[basicstyle=\ttfamily\scriptsize,language=C++,name=fre_string]{ch18/fre_string.cc}
    } {
        \begin{exampleblock}{编程题}
            \begin{itemize}
                \item 输入 $n$ ($1 \le n \le 200$) 个字符串(只包含大小写字母),
                再输入一个字符串 $t$,统计 $n$ 个字符串里与字符串 $t$ 相同的字符串的个数
                    
                \item 样例输入
    
                    \lstinline|3|\\
                    \lstinline|hello HELLO hello|\\
                    \lstinline|hello|
    
                \item 样例输出
                
                    \lstinline|2|
    
            \end{itemize}
        \end{exampleblock}
    }
            
\end{frame}
%------------------------------------------------------------

%------------------------------------------------------------
\begin{frame}[fragile]
    \frametitle{string 数组小结}

    \begin{itemize}
        \item 声明
        
        \begin{itemize}
            \item<2-> \lstinline|string s[210]; // 声明 210 个字符串|
        \end{itemize}

        \item \lstinline|s[i]| 的含义
        
        \begin{itemize}
            \item<3-> 表示的是第 $i$ 个字符串,不是第 $i$ 个字符
        \end{itemize}

        \item 使用场景
        
        \begin{itemize}
            \item<4-> 需要存储多个字符串时
        \end{itemize}

    \end{itemize}
            
\end{frame}
%------------------------------------------------------------


\section{总结}

%------------------------------------------------------------
\begin{frame}[fragile]
    \frametitle{总结}

    \begin{itemize}
        \item string 类型
        
        \begin{itemize}
            \item 声明与初始化、输入、输出、元素访问与遍历
        \end{itemize}

        \item 运算符
        
        \begin{itemize}
            \item =、+、+=、>、<、>=、<=、==、!=
        \end{itemize}

        \item 函数
        
        \begin{itemize}
            \item \lstinline|s.length()|、\lstinline|s.substr(pos, len)|、\lstinline|s.find(s1, pos)|
        \end{itemize}

        \item string 数组

    \end{itemize}
            
\end{frame}
%------------------------------------------------------------

\end{document}