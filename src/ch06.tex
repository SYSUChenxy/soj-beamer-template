%------------------------------------------------------------
\title[06 - 循环结构:for 循环]
{06 - 循环结构:for 循环}

\subtitle{C++ 程序设计基础}

\author[Beiyu Li]
{Beiyu Li\\
\texttt{<sysulby@gmail.com>}}

% \institute[SOJ]
% {Sicily Online Judge}

\date[\today]
{\number\year 年 \number\month 月 \number\day 日}
%------------------------------------------------------------


\begin{document}

\author[sysulby]
{SOJ 信息学竞赛教练组}

\begin{frame}
    \titlepage
\end{frame}
\setcounter{framenumber}{0} % 标题页不编号


\section{复习回顾}

%------------------------------------------------------------
\begin{frame}[fragile]
    \frametitle{while 循环}

    \begin{overlayarea}{\textwidth}{.7\textheight}
        \begin{itemize}

                \lstinputlisting[language=C++,name=while]{ch05/while.cc}

            \item<1-> 不断依照条件语句判定是否继续执行循环

                \begin{itemize}
                    \item<2-> 条件成立(\lstinline|true|)则执行循环体
                    \item<2-> 条件不成立(\lstinline|false|)则结束循环
                \end{itemize}

            \item<3-> 当循环体中只有一个语句时,可以省略花括号
        \end{itemize}
    \end{overlayarea}
\end{frame}
%------------------------------------------------------------

%------------------------------------------------------------
\begin{frame}[fragile]
    \frametitle{do..while 循环}

    \begin{overlayarea}{\textwidth}{.7\textheight}
        \begin{itemize}

                \lstinputlisting[language=C++,name=while]{ch05/do_while.cc}

            \item<1-> 先执行一次循环体,再依照条件语句判定是否继续执行循环

                \begin{itemize}
                    \item<2-> 条件成立(\lstinline|true|)则执行循环体
                    \item<2-> 条件不成立(\lstinline|false|)则结束循环
                \end{itemize}

            \item<3-> \lstinline|do...while| 循环的小括号后必须要有一个分号

            \item<4-> 当循环体中只有一个语句时,可以省略花括号
        \end{itemize}
    \end{overlayarea}
\end{frame}
%------------------------------------------------------------

%------------------------------------------------------------
\begin{frame}[fragile]
    \frametitle{while 和 do..while 的区别}

    \begin{itemize}
        \item \lstinline|while| 循环:先循环再判断
        \item \lstinline|do..while| 循环:先执行再判断,\textbf{至少执行一次}
    \end{itemize}

\end{frame}
%------------------------------------------------------------

%------------------------------------------------------------
\begin{frame}[fragile]
    \frametitle{问题回顾:输出从 n 到 1 之间的整数}

    \alt<2>{
        \lstinputlisting[language=C++,name=while_n_to_1]{ch05/while_n_to_1.cc}
    }{
        \begin{exampleblock}{编程题}

            \begin{itemize}
                \item 编写程序,输入一个整数 $n$ ($1 \le n \le 1000$),输出从 $n$ 到 $1$ 之间的整数,每个数字单独占一行。

                \item 样例输入

                    \lstinline|3|

                \item 样例输出

                    \lstinline|3|\\
                    \lstinline|2|\\
                    \lstinline|1|

            \end{itemize}

        \end{exampleblock}
    }
\end{frame}
%------------------------------------------------------------

%------------------------------------------------------------
\begin{frame}[fragile]
    \frametitle{问题展开:输出从 1 到 n 之间的整数}

    \alt<2-3>{
        \lstinputlisting[language=C++,name=while_1_to_n]{ch06/while_1_to_n.cc}

        \begin{tikzpicture}[remember picture,overlay]
            \uncover<3>{\redbox{while_1_to_n-8}{2}{10}}
            \uncover<3>{\redbox{while_1_to_n-9}{8}{8}}
            \uncover<3>{\redbox{while_1_to_n-11}{4}{4}}
        \end{tikzpicture}
    }{
        \begin{exampleblock}{编程题}

            \begin{itemize}
                \item 编写程序,输入一个整数 $n$ ($1 \le n \le 1000$),输出从 $1$ 到 $n$ 之间的整数,每个数字单独占一行。

                \item 样例输入

                    \lstinline|3|

                \item 样例输出

                    \lstinline|3|\\
                    \lstinline|2|\\
                    \lstinline|1|

            \end{itemize}

        \end{exampleblock}
    }
\end{frame}
%------------------------------------------------------------

%------------------------------------------------------------
\begin{frame}[fragile]
    \frametitle{循环计数}

    \begin{itemize}
        \item<1-> 倒序计数

            \begin{itemize}
                \item 从 $n$ 开始倒序计数,直到 $1$ 进行最后一次循环
                \item 循环条件:\lstinline|n > 0|
                \item 循环体中对循环变量的修改:\lstinline|n--|
            \end{itemize}

        \item<2-> 正序计数

            \begin{itemize}
                \item 从 $1$ 开始正序计数,直到 $n$ 进行最后一次循环
                \item 循环条件:\lstinline|i <= n|
                \item 循环体中对循环变量的修改:\lstinline|i++|
            \end{itemize}

    \end{itemize}
\end{frame}
%------------------------------------------------------------


\section{for 循环}

%------------------------------------------------------------
\begin{frame}[fragile]
    \frametitle{for 循环}

    \begin{overlayarea}{\textwidth}{.9\textheight}
        \begin{itemize}
            \item \lstinline|for| 循环:通过循环变量控制循环的进程

                \uncover<3->{
                    \lstinputlisting[language=C++,name=for]{ch06/for.cc}
                }
        \end{itemize}

        \uncover<2->{
            \vspace*{-1.5em}
            \centering
            \begin{tikzpicture}[node distance=1.2cm]
                \node (start) [process] {初始动作};
                \node (decision1) [decision, below of=start] {条件语句};
                \node (process1) [process, below of=decision1] {循环体};
                \node (process2) [process, below of=process1] {循环后的动作};
                \node (process3) [process, below of=process2] {...};

                \draw [arrow] (start) -- (decision1);
                \draw [arrow] (decision1) -- node[left] {yes} (process1);
                \draw [arrow] (process1) -- (process2);
                \draw [arrow] (process2) -- ++(-2.2,0) |- (decision1);
                \draw [arrow] (decision1) -- ++(2.2,0) |- node[above, xshift=-.6cm] {no} (process3);
            \end{tikzpicture}
        }
    \end{overlayarea}
\end{frame}
%------------------------------------------------------------

%------------------------------------------------------------
\begin{frame}[fragile]
    \frametitle{for 循环}

    \begin{overlayarea}{\textwidth}{.9\textheight}
        \begin{itemize}
            \item \lstinline|for| 循环:通过循环变量控制循环的进程

                \lstinputlisting[language=C++,name=for]{ch06/for.cc}

            \item 小括号中通过两个分号,隔开三种语句(可以留空)

                \begin{itemize}
                    \item<2-> 初始动作只执行一次

                    \item<3-> 条件语句为循环继续的条件(留空视作条件成立)

                        \begin{itemize}
                            \item 条件成立(\lstinline|true|)则执行循环体
                            \item 条件不成立(\lstinline|false|)则结束循环
                        \end{itemize}

                    \item<4-> 每次循环后,必定执行循环后的动作
                \end{itemize}

            \item<5-> 当循环体中只有一个语句时,可以省略花括号
        \end{itemize}
    \end{overlayarea}
\end{frame}
%------------------------------------------------------------

%------------------------------------------------------------
\begin{frame}[fragile]
    \frametitle{示例:for 循环}

    \begin{columns}
        \column{.01\textwidth}

        \column{.64\textwidth}
        \lstinputlisting[language=C++,name=for_1_to_3]{ch06/for_1_to_3.cc}
        \begin{tikzpicture}[remember picture,overlay]
            \uncover<2>{\redbox{for_1_to_3-6}{7}{10}}
            \uncover<3,6,9,12>{\redbox{for_1_to_3-6}{18}{7}}
            \uncover<4,7,10>{\redbox{for_1_to_3-7}{4}{18}}
            \uncover<5,8,11>{\redbox{for_1_to_3-6}{26}{3}}
            \uncover<13->{\redbox{for_1_to_3-10}{2}{9}}
        \end{tikzpicture}

        \column{.35\textwidth}
        \begin{itemize}
                \uncover<2-> {
                \item 变量

                    \only<2-4>{\lstinline|i = 1|}
                    \only<5>{\textcolor{red}{\lstinline|i = 2|}}
                    \only<6-7>{\lstinline|i = 2|}
                    \only<8>{\textcolor{red}{\lstinline|i = 3|}}
                    \only<9-10>{\lstinline|i = 3|}
                    \only<11>{\textcolor{red}{\lstinline|i = 4|}}
                    \only<12>{\textcolor{red}{\lstinline|i <= 3| 不成立}}
                    \only<13->{\lstinline|i = 4|}
                }

                \uncover<4-> {
                \item 输出

                    \uncover<4->{\lstinline|1|\\}
                    \uncover<7->{\lstinline|2|\\}
                    \uncover<10->{\lstinline|3|}
                }
        \end{itemize}
    \end{columns}
\end{frame}
%------------------------------------------------------------

%------------------------------------------------------------
\begin{frame}[fragile]
    \frametitle{例 6.1:输出从 1 到 n 之间的整数}

    \alt<2>{
        \lstinputlisting[language=C++,name=for_1_to_n]{ch06/for_1_to_n.cc}
    }{
        \begin{exampleblock}{编程题}

            \begin{itemize}
                \item 编写程序,输入一个整数 $n$ ($1 \le n \le 1000$),输出从 $1$ 到 $n$ 之间的整数,每个数字单独占一行。

                \item 样例输入

                    \lstinline|3|

                \item 样例输出

                    \lstinline|1|\\
                    \lstinline|2|\\
                    \lstinline|3|

            \end{itemize}

        \end{exampleblock}
    }
\end{frame}
%------------------------------------------------------------


\section{for 循环输出}

%------------------------------------------------------------
\begin{frame}[fragile]
    \frametitle{例 6.2:输出从 0 到 n - 1 之间的整数}

    \alt<2-5>{
        \alt<2-3>{
            \lstinputlisting[language=C++,name=for_0_to_n_minus_1]{ch06/for_0_to_n_minus_1.cc}
        }{
            \lstinputlisting[language=C++,name=rep_n]{ch06/rep_n.cc}
        }
        \begin{tikzpicture}[remember picture,overlay]
            \uncover<3>{\redbox{for_0_to_n_minus_1-8}{7}{22}}
            \uncover<5>{\redbox{rep_n-8}{7}{17}}
        \end{tikzpicture}
    }{
        \begin{exampleblock}{编程题}

            \begin{itemize}
                \item 编写程序,输入一个整数 $n$ ($1 \le n \le 1000$),输出从 $0$ 到 $n-1$ 之间的整数,每个数字单独占一行。

                \item 样例输入

                    \lstinline|3|

                \item 样例输出

                    \lstinline|0|\\
                    \lstinline|1|\\
                    \lstinline|2|

            \end{itemize}

        \end{exampleblock}
    }
\end{frame}
%------------------------------------------------------------

%------------------------------------------------------------
\begin{frame}[fragile]
    \frametitle{例 6.3:输出从 L 到 R 之间的整数}

    \alt<2-3>{
        \lstinputlisting[language=C++,name=for_l_to_r]{ch06/for_l_to_r.cc}
        \begin{tikzpicture}[remember picture,overlay]
            \uncover<3>{\redbox{for_l_to_r-8}{7}{18}}
        \end{tikzpicture}
    }{
        \begin{exampleblock}{编程题}

            \begin{itemize}
                \item 编写程序,输入两个整数 $L$ 和 $R$ ($1 \le L < R \le 1000$),输出从 $L$ 到 $R$ 之间的整数,每个数字单独占一行。

                \item 样例输入

                    \lstinline|3 7|

                \item 样例输出

                    \lstinline|3 4 5 6 7|

            \end{itemize}

        \end{exampleblock}
    }
\end{frame}
%------------------------------------------------------------

%------------------------------------------------------------
\begin{frame}[fragile]
    \frametitle{例 6.4:输出从 1 到 n 之间的奇数}

    \alt<2-3>{
        \lstinputlisting[language=C++,name=for_1_to_n_step_2]{ch06/for_1_to_n_step_2.cc}
        \begin{tikzpicture}[remember picture,overlay]
            \uncover<3>{\redbox{for_1_to_n_step_2-8}{26}{6}}
        \end{tikzpicture}
    }{
        \begin{exampleblock}{编程题}

            \begin{itemize}
                \item 编写程序,输入一个整数 $n$ ($1 \le n \le 1000$),保证 $n$ 为奇数,输出从 $1$ 到 $n$ 之间的奇数,每个数字之间用空格间隔。

                \item 样例输入

                    \lstinline|7|

                \item 样例输出

                    \lstinline|1 3 5 7|

            \end{itemize}

        \end{exampleblock}
    }
\end{frame}
%------------------------------------------------------------

%------------------------------------------------------------
\begin{frame}[fragile]
    \frametitle{随堂练习}

    \begin{exampleblock}{填空题}

        \begin{enumerate}
                \only<1-2>{
                \item[1.] 阅读程序写结果
                    \lstinputlisting[language=C++,name=rof_n_to_1]{ch06/rof_n_to_1.cc}

                    输入:\lstinline|5|\\
                    输出:\uncover<2>{\textcolor{red}{\lstinline|5 4 3 2 1|}}
                }

                \only<3-4>{
                \item[2.] 阅读程序写结果
                    \lstinputlisting[language=C++,name=rof_r_to_l]{ch06/rof_r_to_l.cc}

                    输入:\lstinline|15 20|\\
                    输出:\uncover<4>{\textcolor{red}{\lstinline|20 19 18 17 16 15|}}
                }
        \end{enumerate}

    \end{exampleblock}
\end{frame}
%------------------------------------------------------------


\section{for 循环应用}

%------------------------------------------------------------
\begin{frame}[fragile]
    \frametitle{示例:输出从 1 到 n 的整数之和}

    \lstinputlisting[language=C++,name=for_sum_1_to_n]{ch06/for_sum_1_to_n.cc}

    \begin{tikzpicture}[remember picture,overlay]
        \uncover<2>{\redbox{for_sum_1_to_n-8}{2}{12}}
        \uncover<2>{\redbox{for_sum_1_to_n-10}{4}{9}}
    \end{tikzpicture}
\end{frame}
%------------------------------------------------------------

%------------------------------------------------------------
\begin{frame}[fragile]
    \frametitle{例 6.5:分数数列求和}

    \alt<2-3>{
        \lstinputlisting[language=C++,name=for_sum_frac_1_i]{ch06/for_sum_frac_1_i.cc}

        \begin{tikzpicture}[remember picture,overlay]
            \uncover<3>{\redbox{for_sum_frac_1_i-9}{2}{15}}
            \uncover<3>{\redbox{for_sum_frac_1_i-11}{4}{15}}
        \end{tikzpicture}
    }{
        \begin{exampleblock}{编程题}

            \begin{itemize}
                \item 已知 $S_n = 1 + \frac{1}{2} + \frac{1}{3} + \dots + \frac{1}{n}$。\\
                    输入整数 $n$,输出 $S_n$,结果保留两位小数。

                \item 样例输入

                    \lstinline|3|

                \item 样例输出

                    \lstinline|1.83|

            \end{itemize}

        \end{exampleblock}
    }
\end{frame}
%------------------------------------------------------------

%------------------------------------------------------------
\begin{frame}[fragile]
    \frametitle{例 6.6:对输入的 n 个整数求和}

    \alt<2-3>{
        \lstinputlisting[language=C++,name=for_sum_x]{ch06/for_sum_x.cc}

        \begin{tikzpicture}[remember picture,overlay]
            \uncover<3>{\redbox{for_sum_x-8}{2}{12}}
            \uncover<3>{\redbox{for_sum_x-11}{4}{9}}
        \end{tikzpicture}
    }{
        \begin{exampleblock}{编程题}

            \begin{itemize}
                \item 编写程序,第一行输入一个整数 $n$ ($1 \le n \le 1000$),第二行输入 $n$ 个整数 $x$ ($1 \le x \le 10^6$),输出 $x$ 的和。

                \item 样例输入

                    \lstinline|3|\\
                    \lstinline|2 5 8|

                \item 样例输出

                    \lstinline|15|

            \end{itemize}

        \end{exampleblock}
    }
\end{frame}
%------------------------------------------------------------

%------------------------------------------------------------
\begin{frame}[fragile]
    \frametitle{例 6.7:对输入的 n 个整数求乘积}

    \alt<2-3>{
        \lstinputlisting[language=C++,name=for_mul_x]{ch06/for_mul_x.cc}

        \begin{tikzpicture}[remember picture,overlay]
            \uncover<3>{\redbox{for_mul_x-8}{2}{18}}
            \uncover<3>{\redbox{for_mul_x-11}{4}{9}}
        \end{tikzpicture}
    }{
        \begin{exampleblock}{编程题}

            \begin{itemize}
                \item 编写程序,第一行输入一个整数 $n$ ($1 \le n \le 10$),第二行输入 $n$ 个整数 $x$ ($0 \le x \le 20$),输出 $x$ 的乘积。

                \item 样例输入

                    \lstinline|3|\\
                    \lstinline|2 5 8|

                \item 样例输出

                    \lstinline|80|

            \end{itemize}

        \end{exampleblock}
    }
\end{frame}
%------------------------------------------------------------


\section{循环结构综合归纳}

%------------------------------------------------------------
\begin{frame}[fragile]
    \frametitle{不同循环结构的对比}

    \begin{itemize}
        \item<1-> \lstinline|while| 循环:先判断再执行

            \only<1> {
                \lstinputlisting[language=C++,name=while]{ch05/while.cc}
            }

        \item<2-> \lstinline|do...while| 循环:先执行再判断,至少执行一次

            \only<2> {
                \lstinputlisting[language=C++,name=do_while]{ch05/do_while.cc}
            }

        \item<3-> \lstinline|for| 循环:执行初始动作后,先判断再执行

            \only<3> {
                \lstinputlisting[language=C++,name=for]{ch06/for.cc}
            }

    \end{itemize}
\end{frame}
%------------------------------------------------------------

%------------------------------------------------------------
\begin{frame}[fragile]
    \frametitle{不同循环结构的适用场景}

    \begin{itemize}
        \item<1-> 循环次数未知

            \begin{itemize}
                \item \lstinline|while| 循环
                \item \lstinline|do...while| 循环:至少执行一次
            \end{itemize}

        \item<2-> 循环次数已知

            \begin{itemize}
                \item \lstinline|for| 循环
            \end{itemize}

    \end{itemize}
\end{frame}
%------------------------------------------------------------

%------------------------------------------------------------
\begin{frame}[fragile]
    \frametitle{示例:输出从 L 到 R 之间的整数}

    \begin{columns}[T]
        \column{.04\textwidth}

        \column{.4\textwidth}
        \lstinputlisting[language=C++,name=while_l_to_r]{ch06/while_l_to_r.cc}

        \column{.56\textwidth}
        \lstinputlisting[language=C++,name=for_l_to_r]{ch06/for_l_to_r.cc}
    \end{columns}
\end{frame}
%------------------------------------------------------------


\section{总结}

%------------------------------------------------------------
\begin{frame}[fragile]
    \frametitle{总结}

    \begin{itemize}
        \item \lstinline|for| 循环的语法语义

        \item \lstinline|for| 循环的应用

            \begin{itemize}
                \item 循环输出
                \item 循环累加
                \item 循环累乘
            \end{itemize}

        \item 循环结构综合归纳
    \end{itemize}
\end{frame}
%------------------------------------------------------------

%------------------------------------------------------------
\begin{frame}
    \begin{center}
        {\Huge Thank you!}
    \end{center}
\end{frame}
%------------------------------------------------------------

\end{document}
