%------------------------------------------------------------
\title[05 - 循环结构:while 循环]
{05 - 循环结构:while 循环}

\subtitle{C++ 程序设计基础}

\author[Beiyu Li]
{Beiyu Li\\
\texttt{<sysulby@gmail.com>}}

% \institute[SOJ]
% {Sicily Online Judge}

\date[\today]
{\number\year 年 \number\month 月 \number\day 日}
%------------------------------------------------------------


\begin{document}

\author[sysulby]
{SOJ 信息学竞赛教练组}

\begin{frame}
    \titlepage
\end{frame}
\setcounter{framenumber}{0} % 标题页不编号


\section{复习回顾}

%------------------------------------------------------------
\begin{frame}[fragile]
    \frametitle{顺序结构与分支结构}

    \begin{itemize}[<+->]
        \item 顺序结构:程序按照从上到下的顺序逐条执行各个语句
        \item 分支结构:可以根据不同情况,执行不同的语句
    \end{itemize}
\end{frame}
%------------------------------------------------------------

%------------------------------------------------------------
\begin{frame}[fragile]
    \frametitle{顺序结构与分支结构的局限性}

    \begin{itemize}[<+->]
        \item 只能处理小规模的问题

            \begin{itemize}
                \item 输出一行 \lstinline|"Hello world"|,如果要求输出 $100$ 行呢?
                \item 判断年份是否是闰年,如果有 $1000$ 个年份要判断呢?
            \end{itemize}

    \end{itemize}
\end{frame}
%------------------------------------------------------------

%------------------------------------------------------------
\begin{frame}[fragile]
    \frametitle{循环结构}

    \begin{itemize}
        \item<1-> 循环结构:能够\textbf{反复}执行一段指令

        \item<2-> 循环结构的主要语法

            \begin{itemize}
                \item \lstinline|while| 循环
                \item \lstinline|do...whie| 循环
                \item \lstinline|for| 循环
            \end{itemize}

    \end{itemize}
\end{frame}
%------------------------------------------------------------


\section{while 循环}

%------------------------------------------------------------
\begin{frame}[fragile]
    \frametitle{while 循环}

    \begin{columns}
        \column{.66\textwidth}
        \begin{overlayarea}{\textwidth}{.6\textheight}
            \begin{itemize}
                \item \lstinline|while| 循环:依照条件语句判定是否继续执行循环
            \end{itemize}

            \uncover<2->{
                \vspace*{1em}
                \centering
                \begin{tikzpicture}[node distance=1.25cm]
                    \node (start) [process] {...};
                    \node (decision1) [decision, below of=start] {条件语句};
                    \node (process1) [process, below of=decision1] {循环体};
                    \node (stop) [process, below of=process1] {...};

                    \draw [arrow] (start) -- (decision1);
                    \draw [arrow] (decision1) -- node[left] {yes} (process1);
                    \draw [arrow] (decision1) -- ++(2.2,0) |- node[above, xshift=-.6cm] {no} (stop);
                    \draw [arrow] (process1) -- ++(-2.2,0) |- (decision1);
                \end{tikzpicture}
            }
        \end{overlayarea}

        \column{.34\textwidth}
        \begin{overlayarea}{\textwidth}{.3\textheight}
            \uncover<3->{
                \lstinputlisting[basicstyle=\ttfamily\scriptsize,language=C++,name=while]{ch05/while.cc}
            }
        \end{overlayarea}
    \end{columns}
\end{frame}
%------------------------------------------------------------

%------------------------------------------------------------
\begin{frame}[fragile]
    \frametitle{while 循环}

    \begin{columns}
        \column{.66\textwidth}
        \begin{overlayarea}{\textwidth}{.6\textheight}
            \begin{itemize}
                \item \lstinline|while| 循环:依照条件语句判定是否继续执行循环

                \item 条件语句通常是布尔类型的表达式

                    \begin{itemize}
                        \item<2-> 条件成立(\lstinline|true|)则执行循环体
                        \item<3-> 条件不成立(\lstinline|false|)则结束循环
                    \end{itemize}

                \item<4-> 当循环体中只有一个语句时,可以省略花括号
            \end{itemize}
        \end{overlayarea}

        \column{.34\textwidth}
        \begin{overlayarea}{\textwidth}{.3\textheight}
            \lstinputlisting[basicstyle=\ttfamily\scriptsize,language=C++,name=while]{ch05/while.cc}
        \end{overlayarea}
    \end{columns}
\end{frame}
%------------------------------------------------------------

%------------------------------------------------------------
\begin{frame}[fragile]
    \frametitle{示例:while 循环}

    \begin{columns}
        \column{.01\textwidth}

        \column{.69\textwidth}
        \lstinputlisting[basicstyle=\ttfamily\scriptsize,language=C++,name=while_3_hello]{ch05/while_3_hello.cc}
        \begin{tikzpicture}[remember picture,overlay]
            \uncover<2>{\redbox{while_3_hello}{6}{3}{6}{12};}
            \uncover<3,6,9,12>{\redbox{while_3_hello}{7}{3}{7}{17};}
            \uncover<4,7,10>{\redbox{while_3_hello}{8}{5}{8}{28};}
            \uncover<5,8,11>{\redbox{while_3_hello}{9}{5}{9}{8};}
            \uncover<13->{\redbox{while_3_hello}{12}{3}{12}{11};}
        \end{tikzpicture}

        \column{.3\textwidth}
        \begin{itemize}
                \uncover<2-> {
                \item 变量

                    \only<2-4>{\lstinline|n = 3|}
                    \only<5>{\textcolor{red}{\lstinline|n = 2|}}
                    \only<6-7>{\lstinline|n = 2|}
                    \only<8>{\textcolor{red}{\lstinline|n = 1|}}
                    \only<9-10>{\lstinline|n = 1|}
                    \only<11>{\textcolor{red}{\lstinline|n = 0|}}
                    \only<12>{\textcolor{red}{\lstinline|n > 0| 不成立}}
                    \only<13->{\lstinline|n = 0|}
                }

                \uncover<4-> {
                \item 输出

                    \uncover<4->{\lstinline|hello|\\}
                    \uncover<7->{\lstinline|hello|\\}
                    \uncover<10->{\lstinline|hello|}
                }
        \end{itemize}
    \end{columns}
\end{frame}
%------------------------------------------------------------

%------------------------------------------------------------
\begin{frame}[fragile]
    \frametitle{例 5.1:输出 n 行 hello}

    \alt<2>{
        \lstinputlisting[basicstyle=\ttfamily\scriptsize,language=C++,name=while_n_hello]{ch05/while_n_hello.cc}
    }{
        \begin{exampleblock}{编程题}

            \begin{itemize}
                \item 输入一个整数 $n$ ($1 \le n \le 1000$),输出 $n$ 行 hello。

                \item 样例输入

                    \lstinline|2|

                \item 样例输出

                    \lstinline|hello|\\
                    \lstinline|hello|

            \end{itemize}

        \end{exampleblock}
    }
\end{frame}
%------------------------------------------------------------


\section{while 循环的应用}

%------------------------------------------------------------
\begin{frame}[fragile]
    \frametitle{例 5.2:输出从 n 到 1 之间的整数}

    \alt<2>{
        \lstinputlisting[basicstyle=\ttfamily\scriptsize,language=C++,name=while_n_to_1]{ch05/while_n_to_1.cc}
    }{
        \begin{exampleblock}{编程题}

            \begin{itemize}
                \item 输入一个整数 $n$ ($1 \le n \le 1000$),输出从 $n$ 到 $1$ 之间的整数,每个数字单独占一行。

                \item 样例输入

                    \lstinline|3|

                \item 样例输出

                    \lstinline|3|\\
                    \lstinline|2|\\
                    \lstinline|1|

            \end{itemize}

        \end{exampleblock}
    }
\end{frame}
%------------------------------------------------------------

%------------------------------------------------------------
\begin{frame}[fragile]
    \frametitle{例 5.3:输出从 n 到 1 的整数之和}

    \alt<2-3>{
        \lstinputlisting[basicstyle=\ttfamily\scriptsize,language=C++,name=while_sum_n_to_1]{ch05/while_sum_n_to_1.cc}

        \begin{tikzpicture}[remember picture,overlay]
            \uncover<3>{\redbox{while_sum_n_to_1}{8}{3}{8}{14};}
            \uncover<3>{\redbox{while_sum_n_to_1}{10}{5}{10}{13};}
        \end{tikzpicture}
    }{
        \begin{exampleblock}{编程题}

            \begin{itemize}
                \item 输入一个整数 $n$ ($1 \le n \le 1000$),输出从 $n$ 到 $1$ 之间的整数之和。

                \item 样例输入

                    \lstinline|5|

                \item 样例输出

                    \lstinline|15|

                \item 样例说明

                    $5 + 4 + 3 + 2 + 1 = 15$

            \end{itemize}

        \end{exampleblock}
    }
\end{frame}
%------------------------------------------------------------

%------------------------------------------------------------
\begin{frame}[fragile]
    \frametitle{例 5.4:对输入的 n 个整数求和}

    \alt<2-3>{
        \lstinputlisting[basicstyle=\ttfamily\scriptsize,language=C++,name=while_sum_x]{ch05/while_sum_x.cc}

        \begin{tikzpicture}[remember picture,overlay]
            \uncover<3>{\redbox{while_sum_x}{8}{3}{8}{14};}
            \uncover<3>{\redbox{while_sum_x}{10}{5}{11}{13};}
        \end{tikzpicture}
    }{
        \begin{exampleblock}{编程题}

            \begin{itemize}
                \item 第一行输入一个整数 $n$ ($1 \le n \le 1000$),第二行输入 $n$ 个整数 $x$ ($-10^6 \le x \le 10^6$),输出 $x$ 的和。

                \item 样例输入

                    \lstinline|3|\\
                    \lstinline|2 5 8|

                \item 样例输出

                    \lstinline|15|

            \end{itemize}

        \end{exampleblock}
    }
\end{frame}
%------------------------------------------------------------

%------------------------------------------------------------
\begin{frame}[fragile]
    \frametitle{例 5.5:输出 n 天的学习时长}

    \alt<2>{
        \lstinputlisting[basicstyle=\ttfamily\scriptsize,language=C++,name=while_n_days_study]{ch05/while_n_days_study.cc}
    }{
        \begin{exampleblock}{编程题}

            \begin{itemize}
                \item 输入一个整数 $n$ ($1 \le n \le 1000$),表示有 $n$ 天。\\
                    接下来 $n$ 行,每行输入两个整数 $a$, $b$ ($0 \le a, b \le 6$),分别表示每天校内的学习时长和校外的学习时长。\\
                    对于每一天,输出一行,表示这天的总学习时长。

                \item 样例输入

                    \lstinline|2|\\
                    \lstinline|5 3|\\
                    \lstinline|3 2|

                \item 样例输出

                    \lstinline|8|\\
                    \lstinline|5|

            \end{itemize}

        \end{exampleblock}
    }
\end{frame}
%------------------------------------------------------------

%------------------------------------------------------------
\begin{frame}[fragile]
    \frametitle{思考}

    \begin{block}{}
        \vspace{.5cm}
        \begin{center}
            前面输入的数据的数量都是确定的,\\
            如果输入的数据的数量不确定该怎么做呢?
        \end{center}
        \vspace{.5cm}
    \end{block}
\end{frame}
%------------------------------------------------------------

%------------------------------------------------------------
\begin{frame}[fragile]
    \frametitle{例 5.6:求出若干个数的总和}

    \alt<2-3>{
        \lstinputlisting[basicstyle=\ttfamily\scriptsize,language=C++,name=while_cin_sum]{ch05/while_cin_sum.cc}

        \begin{tikzpicture}[remember picture,overlay]
            \uncover<3>{\redbox{while_cin_sum}{9}{10}{9}{27};}
        \end{tikzpicture}
    }{
        \begin{exampleblock}{编程题}

            \begin{itemize}
                \item 输入若干个整数 $x$ ($1 \le x \le 1000$, 不超过 $100$ 个),当输入为 $0$ 时结束,求输入的数的总和。

                \item 样例输入

                    \lstinline|5 1 3 4 0|

                \item 样例输出

                    \lstinline|13|

                \item 样例说明

                    $5 + 1 + 3 + 4 = 13$

            \end{itemize}

        \end{exampleblock}
    }
\end{frame}
%------------------------------------------------------------


\section{do..while 循环}

%------------------------------------------------------------
\begin{frame}[fragile]
    \frametitle{do..while 循环}

    \begin{columns}
        \column{.66\textwidth}
        \begin{overlayarea}{\textwidth}{.7\textheight}
            \begin{itemize}
                \item \lstinline|do..while| 循环:先执行一次循环体,再依照条件语句判定是否继续执行循环
            \end{itemize}

            \uncover<2->{
                \vspace*{1em}
                \centering
                \begin{tikzpicture}[node distance=1.25cm]
                    \node (start) [process] {...};
                    \node (process1) [process, below of=start] {循环体};
                    \node (decision1) [decision, below of=process1] {条件语句};
                    \node (stop) [process, below of=decision1] {...};

                    \draw [arrow] (start) -- (process1);
                    \draw [arrow] (process1) -- (decision1);
                    \draw [arrow] (decision1) -- ++(-2.2, 0) |- node[right, yshift = -.6cm] {yes} (process1);
                    \draw [arrow] (decision1) -- node[left] {no} (stop);
                \end{tikzpicture}
            }
        \end{overlayarea}

        \column{.34\textwidth}
        \begin{overlayarea}{\textwidth}{.3\textheight}
            \uncover<3->{
                \lstinputlisting[basicstyle=\ttfamily\scriptsize,language=C++,name=do_while]{ch05/do_while.cc}
            }
        \end{overlayarea}
    \end{columns}
\end{frame}
%------------------------------------------------------------

%------------------------------------------------------------
\begin{frame}[fragile]
    \frametitle{do..while 循环}

    \begin{columns}
        \column{.66\textwidth}
        \begin{overlayarea}{\textwidth}{.7\textheight}
            \begin{itemize}
                \item \lstinline|do..while| 循环:先执行一次循环体,再依照条件语句判定是否继续执行循环

                \item 条件语句通常是布尔类型的表达式

                    \begin{itemize}
                        \item<2-> 条件成立(\lstinline|true|)则执行循环体
                        \item<3-> 条件不成立(\lstinline|false|)则结束循环
                    \end{itemize}

                \item<4-> \lstinline|do...while| 循环的小括号后必须要有一个分号

                \item<5-> 当语句块中只有一个语句时,可以省略花括号
            \end{itemize}
        \end{overlayarea}

        \column{.34\textwidth}
        \begin{overlayarea}{\textwidth}{.3\textheight}
            \lstinputlisting[basicstyle=\ttfamily\scriptsize,language=C++,name=do_while]{ch05/do_while.cc}
        \end{overlayarea}
    \end{columns}
\end{frame}
%------------------------------------------------------------

%------------------------------------------------------------
\begin{frame}[fragile]
    \frametitle{示例:do..while 循环}

    \lstinputlisting[basicstyle=\ttfamily\scriptsize,language=C++,name=do_while_3_hello]{ch05/do_while_3_hello.cc}
\end{frame}
%------------------------------------------------------------

%------------------------------------------------------------
\begin{frame}[fragile]
    \frametitle{例 5.7:判断非负整数的位数}

    \alt<2-6>{
        \alt<2-4>{
            \lstinputlisting[basicstyle=\ttfamily\scriptsize,language=C++,name=while_count_digit]{ch05/while_count_digit.cc}
        }{
            \lstinputlisting[basicstyle=\ttfamily\scriptsize,language=C++,name=do_while_count_digit]{ch05/do_while_count_digit.cc}
        }
        \uncover<4> {
            \begin{itemize}
                \item 如果输入为 $0$ 呢?
            \end{itemize}
        }

        \begin{tikzpicture}[remember picture,overlay]
            \uncover<3-4>{\redbox{while_count_digit}{8}{3}{8}{14};}
            \uncover<3-4>{\redbox{while_count_digit}{11}{5}{11}{12};}
            \uncover<6->{\redbox{do_while_count_digit}{9}{3}{9}{6};}
            \uncover<6->{\redbox{do_while_count_digit}{12}{3}{12}{19};}
        \end{tikzpicture}
    }{
        \begin{exampleblock}{编程题}

            \begin{itemize}
                \item 输入一个非负整数 $n$ ($1 \le n \le 10^9$),输出 $n$ 的位数。

                \item 样例输入

                    \lstinline|16|

                \item 样例输出

                    \lstinline|2|

            \end{itemize}

        \end{exampleblock}
    }
\end{frame}
%------------------------------------------------------------

%------------------------------------------------------------
\begin{frame}[fragile]
    \frametitle{例 5.8:反向输出非负整数}

    \alt<2-3>{
        \lstinputlisting[basicstyle=\ttfamily\scriptsize,language=C++,name=do_while_reverse_digit]{ch05/do_while_reverse_digit.cc}

        \begin{tikzpicture}[remember picture,overlay]
            \uncover<3>{\redbox{do_while_reverse_digit}{9}{13}{9}{18};}
        \end{tikzpicture}
    }{
        \begin{exampleblock}{编程题}

            \begin{itemize}
                \item 输入一个整数 $n$ ($0 \le n \le 10^9$),对 $n$ 进行拆分数位并反向输出每一位数,数字之间以空格间隔。

                \item 样例输入

                    \lstinline|856|

                \item 样例输出

                    \lstinline|6 5 8|

            \end{itemize}

        \end{exampleblock}
    }
\end{frame}
%------------------------------------------------------------

%------------------------------------------------------------
\begin{frame}[fragile]
    \frametitle{while 和 do..while 的区别}

    \begin{itemize}
        \item \lstinline|while| 循环:先循环再判断
        \item \lstinline|do..while| 循环:先执行再判断,\textbf{至少执行一次}
    \end{itemize}

\end{frame}
%------------------------------------------------------------


\section{总结}

%------------------------------------------------------------
\begin{frame}[fragile]
    \frametitle{总结}

    \begin{itemize}
        \item \lstinline|while| 循环
        \item \lstinline|do...while| 循环
        \item 累加器 \lstinline|sum| 和 计数器 \lstinline|cnt| 需要初始化为 $0$
    \end{itemize}
\end{frame}
%------------------------------------------------------------

%------------------------------------------------------------
\begin{frame}
    \begin{center}
        {\Huge Thank you!}
    \end{center}
\end{frame}
%------------------------------------------------------------

\end{document}
