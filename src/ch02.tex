%------------------------------------------------------------
\title[02 - 变量与输入输出]
{02 - 变量与输入输出}

\subtitle{C++ 程序设计基础}

\author[Beiyu Li]
{Beiyu Li\\
\texttt{<sysulby@gmail.com>}}

% \institute[SOJ]
% {Sicily Online Judge}

\date[\today]
{\number\year 年 \number\month 月 \number\day 日}
%------------------------------------------------------------


\begin{document}

\author[sysulby]
{SOJ 信息学竞赛教练组}

\begin{frame}
    \titlepage
\end{frame}
\setcounter{framenumber}{0} % 标题页不编号


\section{复习回顾}

%------------------------------------------------------------
\begin{frame}[fragile]
    \frametitle{程序的基本结构}

    \lstinputlisting[basicstyle=\ttfamily\scriptsize,language=C++,name=structure]{ch01/structure1.cc}
\end{frame}
%------------------------------------------------------------

%------------------------------------------------------------
\begin{frame}[fragile]
    \frametitle{输出语句}

    \alt<2>{
        \lstinputlisting[basicstyle=\ttfamily\scriptsize,language=C++,name=output2]{ch02/output2.cc}
    }{
        \lstinputlisting[basicstyle=\ttfamily\scriptsize,language=C++,name=output1]{ch02/output1.cc}
    }
\end{frame}
%------------------------------------------------------------

%------------------------------------------------------------
\begin{frame}[fragile]
    \frametitle{格式控制}

    \lstinputlisting[basicstyle=\ttfamily\scriptsize,language=C++,name=output3]{ch02/output3.cc}

    \begin{itemize}
        \item<2-> 输出结果:3.14
    \end{itemize}
\end{frame}
%------------------------------------------------------------


\section{变量}

%------------------------------------------------------------
\begin{frame}[fragile]
    \frametitle{变量}

    \begin{itemize}
        \item<1-> 变量的概念

            \begin{itemize}
                \item 变量就是在程序运行过程中,其值可以被改变的量
            \end{itemize}

        \item<2-> 变量的作用

            \begin{itemize}
                \item 程序运行过程中\textbf{存储数据}的“盒子”
            \end{itemize}

        \item<3-> 变量的声明

            \begin{itemize}
                \item \textbf{类型 \enspace 变量名}
                \item \lstinline|int a;|
            \end{itemize}

    \end{itemize}
\end{frame}
%------------------------------------------------------------

%------------------------------------------------------------
\begin{frame}[fragile]
    \frametitle{变量}

    \begin{itemize}
        \item<1-> 变量的初始化

            \begin{itemize}
                \item 变量在声明的同时设置一个初始的数值
                \item \textbf{类型 \enspace 变量名 = 值}
                \item \lstinline|int a = 2;|
            \end{itemize}

        \item<2-> 变量的赋值

            \begin{itemize}
                \item 在变量声明之后,可以通过 \textcolor{red}{=} 修改变量存储的数据
                \item \lstinline|a = 3; // 变量 a 中存储的数据会从 2 变为 3|
            \end{itemize}

    \end{itemize}
\end{frame}
%------------------------------------------------------------

%------------------------------------------------------------
\begin{frame}[fragile]
    \frametitle{变量的命名规则}

    \begin{itemize}

        \item<1-> 变量名只能由字母、数字或下划线构成,但不能以数字开头

            \begin{itemize}
                \item \lstinline|dis2| 或 \lstinline|_dis| 是合法变量名
                \item \lstinline|2dis| 是非法变量名
            \end{itemize}

        \item<2-> 变量名区分大小写

            \begin{itemize}
                \item \lstinline|Dis| 与 \lstinline|dis| 视作不同变量名
            \end{itemize}

        \item<3-> 变量名不能与 C++ 的关键字重名

            \begin{itemize}
                \item \lstinline|std| 和 \lstinline|return| 是非法变量名
            \end{itemize}

    \end{itemize}
\end{frame}
%------------------------------------------------------------

%------------------------------------------------------------
\begin{frame}[fragile]
    \frametitle{变量的输入与输出}

    \begin{itemize}
        \item<1-> 当我们希望变量的值不是预先写入程序,而是通过用户输入决定时,应该怎么办?

        \item<2-> 使用 \lstinline|cin| 语句输入变量,在输入流符号 \lstinline|>>| 后写上变量名

            \begin{itemize}
                \item
                    \lstinline|int a;|\\
                    \lstinline|cin >> a;|
            \end{itemize}

        \item<3-> 使用 \lstinline|cout| 语句输出变量,在输出流符号 \lstinline|<<| 后写上变量名

            \begin{itemize}
                \item \lstinline|cout << a << endl;|
            \end{itemize}

    \end{itemize}
\end{frame}
%------------------------------------------------------------

%------------------------------------------------------------
\begin{frame}[fragile]
    \frametitle{多个变量的使用}

    \begin{itemize}
        \item<1-> 可以同时声明多个变量

            \begin{itemize}
                \item \lstinline|int a, b;|
            \end{itemize}

        \item<2-> 同时输入多个变量

            \begin{itemize}
                \item \lstinline|cin >> a >> b;|
            \end{itemize}

        \item<3-> 同时输出多个变量

            \begin{itemize}
                \item \lstinline|cout << a << " " << b << endl;|
            \end{itemize}

    \end{itemize}
\end{frame}
%------------------------------------------------------------

%------------------------------------------------------------
\begin{frame}[fragile]
    \frametitle{例 2.1:输出长方形的长和宽 I}

    \alt<2>{
        \lstinputlisting[basicstyle=\ttfamily\scriptsize,language=C++,name=example1]{ch02/example1.cc}
    }{
        \begin{exampleblock}{编程题}

            \begin{itemize}
                \item 编写程序,由用户输入两个整数 $a$, $b$,代表长方形的长和宽,再输出长和宽。\\
                    输出分两行,第一行输出长,第二行输出宽。

                \item 样例输入

                    \lstinline|5 3|

                \item 样例输出

                    \lstinline|length of rectangle: 5|\\
                    \lstinline|width of rectangle: 3|

            \end{itemize}

        \end{exampleblock}
    }
\end{frame}
%------------------------------------------------------------


\section{数据类型}

%------------------------------------------------------------
\begin{frame}[fragile]
    \frametitle{变量的数据类型}

    \begin{block}{整数类型}

        \begin{itemize}
            \item \lstinline|int|:$-2^{31} \sim 2^{31} - 1$(近似值 $2.1 \times 10^9$)
            \item \lstinline|long long|:$-2^{63} \sim 2^{63} - 1$(近似值 $9.2 \times 10^{18}$)
        \end{itemize}

    \end{block}

    \begin{itemize}
        \item
            \lstinline|int a = 123;|\\
            \lstinline|cout << a << endl; // 输出 123|

        \item<2->
            \alt<2>{
                \lstinline|int b = 123456789123;|\\
                \lstinline|cout << b << endl; // 输出错误,整数溢出|
            }{
                \redout{{\lstinline|int b = 123456789123;|}}\\
                \redout{{\lstinline|cout << b << endl; // 输出错误,整数溢出|}}
            }

        \item<3->
            \lstinline|long long b = 123456789123;|\\
            \lstinline|cout << b << endl; // 输出 123456789123|
    \end{itemize}
\end{frame}
%------------------------------------------------------------

%------------------------------------------------------------
\begin{frame}[fragile]
    \frametitle{变量的数据类型}

    \begin{block}{浮点数类型(实数/小数)}

        \begin{itemize}
            \item \lstinline|double|:双精度浮点数
        \end{itemize}

    \end{block}

    \begin{itemize}
        \item
            \lstinline|double d = 2.33;|\\
            \lstinline|cout << d << endl; // 输出 2.33|
    \end{itemize}
\end{frame}
%------------------------------------------------------------

%------------------------------------------------------------
\begin{frame}[fragile]
    \frametitle{变量的数据类型}

    \begin{block}{字符类型}

        \begin{itemize}
            \item \lstinline|char|:键盘上的字母、数字及符号
        \end{itemize}

    \end{block}

    \begin{itemize}
        \item
            \lstinline|char c = '!';|\\
            \lstinline|cout << c << endl; // 输出 !|
    \end{itemize}
\end{frame}
%------------------------------------------------------------

%------------------------------------------------------------
\begin{frame}[fragile]
    \frametitle{变量的数据类型}

    \begin{block}{布尔类型}

        \begin{itemize}
            \item \lstinline|bool|:true / false(真 / 假)
        \end{itemize}

    \end{block}

    \begin{itemize}
        \item
            \lstinline|bool t = true;|\\
            \lstinline|cout << t << endl; // 输出 1|

        \item
            \lstinline|bool f = false;|\\
            \lstinline|cout << f << endl; // 输出 0|
    \end{itemize}
\end{frame}
%------------------------------------------------------------

%------------------------------------------------------------
\begin{frame}[fragile]
    \frametitle{例 2.2:输出长方形的长和宽 II}

    \alt<2>{
        \lstinputlisting[basicstyle=\ttfamily\scriptsize,language=C++,name=example2]{ch02/example2.cc}
    }{
        \begin{exampleblock}{编程题}

            \begin{itemize}
                \item 编写程序,由用户输入两个实数 $a$, $b$,代表长方形的长和宽,再输出长和宽。\\
                    输出分两行,第一行输出长,第二行输出宽。结果保留 $2$ 位小数。

                \item 样例输入

                    \lstinline|5.727 3.472|

                \item 样例输出

                    \lstinline|length of rectangle: 5.73|\\
                    \lstinline|width of rectangle: 3.47|

            \end{itemize}

        \end{exampleblock}
    }
\end{frame}
%------------------------------------------------------------


\section{常量}

%------------------------------------------------------------
\begin{frame}[fragile]
    \frametitle{常量}

    \begin{itemize}
        \item<1-> 常量的概念

            \begin{itemize}
                \item 常量就是在程序运行过程中,其值不能被改变的量
            \end{itemize}

        \item<2-> 常量的作用

            \begin{itemize}
                \item 存储在程序运行过程中不能被修改的数据,更易于维护
                \item 例如:圆周率 $PI$
            \end{itemize}

        \item<3-> 常量的声明

            \begin{itemize}
                \item 声明常量与声明变量类似
                \item 但需要在数据类型前加 \lstinline|const| 关键字
                \item \lstinline|const double PI = 3.14159;|
            \end{itemize}

    \end{itemize}
\end{frame}
%------------------------------------------------------------

%------------------------------------------------------------
\begin{frame}[fragile]
    \frametitle{注意事项}

    \begin{itemize}
        \item<1-> 常量必须在声明的同时初始化

        \item<2-> 不能被重新赋值

            \begin{itemize}
                \item
                    \alt<2>{
                        \lstinline|const double PI;|\\
                        \lstinline|PI = 3.14159;|
                    }{
                        \redout{{\lstinline|const double PI;|}}\\
                        \redout{{\lstinline|PI = 3.14159;|}}
                    }
            \end{itemize}

        \item<4-5> 也不能输入

            \begin{itemize}
                \item
                    \alt<4>{
                        \lstinline|const double PI;|\\
                        \lstinline|cin >> PI;|
                    }{
                        \redout{{\lstinline|const double PI;|}}\\
                        \redout{{\lstinline|cin >> PI;|}}
                    }
            \end{itemize}

    \end{itemize}
\end{frame}
%------------------------------------------------------------

%------------------------------------------------------------
\begin{frame}[fragile]
    \frametitle{字面量}

    \begin{itemize}
        \item<1-> 字面量的概念

            \begin{itemize}
                \item C++ 源代码中的固定数值,是预先写入程序中的数据
                \item \lstinline|int a = 2;|
            \end{itemize}

        \item<2-> 字面量的数据类型

            \begin{itemize}
                \item \lstinline|cout << 3 << endl;   // 3 默认是整数类型|
                \item \lstinline|cout << 3.0 << endl; // 3.0 默认是浮点数类型|
            \end{itemize}

    \end{itemize}
\end{frame}
%------------------------------------------------------------


\section{类型转换}

%------------------------------------------------------------
\begin{frame}[fragile]
    \frametitle{类型转换}

    \lstinputlisting[basicstyle=\ttfamily\scriptsize,language=C++,name=cast]{ch02/cast.cc}

    \begin{itemize}
        \item<2-> 以上程序能编译通过吗?
        \item<2-> 运行结果是两行 \lstinline|123456789123| 吗?
    \end{itemize}
\end{frame}
%------------------------------------------------------------

%------------------------------------------------------------
\begin{frame}[fragile]
    \frametitle{类型转换}

    \begin{itemize}
        \item 任何基本数据类型之间都可以相互转换,常见的有:

            \begin{itemize}
                \item \lstinline|int <=> long long|
                \item \lstinline|int <=> double|
                \item \lstinline|int <=> bool|
            \end{itemize}

    \end{itemize}
\end{frame}
%------------------------------------------------------------

%------------------------------------------------------------
\begin{frame}[fragile]
    \frametitle{类型转换}

    \begin{itemize}
        \item<1-> \lstinline|int| 和 \lstinline|long long| 的数据范围不同(容量不同)

        \item<2-> \lstinline|int => long long|

            \begin{itemize}
                \item 数据正常转换
            \end{itemize}

        \item<3-> \lstinline|long long => int|

            \begin{itemize}
                \item 超过 \lstinline|int| 类型上限($2.1 \times 10^9$)时会\textbf{溢出},导致数据不准确
                \item 其它情况下,数据正常转换
            \end{itemize}

    \end{itemize}
\end{frame}
%------------------------------------------------------------

%------------------------------------------------------------
\begin{frame}[fragile]
    \frametitle{类型转换}

    \begin{itemize}
        \item<1-> \lstinline|double| 和 \lstinline|int| 的区别在于是否存在小数部分

        \item<2-> \lstinline|double => int|

            \begin{itemize}
                \item 删除小数部分
                \item \lstinline|2.9 => 2|
            \end{itemize}

        \item<3-> \lstinline|int => double|

            \begin{itemize}
                \item 添加小数部分
                \item \lstinline|5 => 5.0|
            \end{itemize}

    \end{itemize}
\end{frame}
%------------------------------------------------------------

%------------------------------------------------------------
\begin{frame}[fragile]
    \frametitle{类型转换}

    \begin{itemize}
        \item<1-> \lstinline|bool| 类型表示 true / false,实际存储的是数值 $1$ / $0$

        \item<2-> \lstinline|bool => int|

            \begin{itemize}
                \item \lstinline|false => 0|
                \item \lstinline|true => 1|
            \end{itemize}

        \item<3-> \lstinline|int => bool|

            \begin{itemize}
                \item \lstinline|0 => false|
                \item \lstinline|非 0 => true|
            \end{itemize}

    \end{itemize}
\end{frame}
%------------------------------------------------------------

%------------------------------------------------------------
\begin{frame}[fragile]
    \frametitle{类型转换}

    \begin{itemize}
        \item<1-> 直接赋值时,会产生隐式类型转换

            \begin{itemize}
                \item
                    \lstinline|int a = 2.9;|\\
                    \lstinline|cout << a << endl; // 输出 2|
            \end{itemize}

        \item<2-> 也可以通过 \textbf{(类型)变量名} 的形式进行强制类型转换

            \begin{itemize}
                \item
                    \lstinline|double b = 2.9;|\\
                    \lstinline|cout << b << endl;      // 输出 2.9|\\
                    \lstinline|cout << (int)b << endl; // 输出 2|
            \end{itemize}

    \end{itemize}
\end{frame}
%------------------------------------------------------------


\section{总结}

%------------------------------------------------------------
\begin{frame}[fragile]
    \frametitle{总结}

    \begin{itemize}
        \item 变量与常量
            \begin{itemize}
                \item 声明、初始化
                \item 赋值
                \item 输入输出
            \end{itemize}

        \item 数据类型与类型转换
    \end{itemize}
\end{frame}
%------------------------------------------------------------

%------------------------------------------------------------
\begin{frame}
    \begin{center}
        {\Huge Thank you!}
    \end{center}
\end{frame}
%------------------------------------------------------------

\end{document}
